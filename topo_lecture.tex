\documentclass[language=english]{TemplateLecture}

\renewcommand{\ProfName}{Stefan Schwede}
\renewcommand{\LectureName}{Topology II}
\renewcommand{\Semester}{SoSe 2025}
\renewcommand{\mName}{Jan Malmström}

\begin{document}

\setcounter{chapter}{0}
\chapter{Cohomology}

\newLecture{07.04.2025}

\section{Last Term}

In last term, we discussed
\begin{itemize}
    \item CW-complexes
    \item higher homotopy groups
    \item Whitehead theorem 
    \item Singular homology
    \item cellular homology
\end{itemize}

In the very end, cohomology was started. Remeber
\[\begin{tikzcd}
    & & & \mathrm{Ab} \\
    \mathrm{TOP} \ar[rrru, bend left, "H_n(X;A)"] \ar[rrrd, bend right, "H^n(X;A)"] \ar[r, "\cS"] & (\mathrm{simpl. Sets}) \ar[r, "{C(\_, \IZ)}"] & (\mathrm{Chains}) \ar[ru, "H_n(\_ \otimes A)"] \ar[rd, "{H^n(\Hom(\_, A))}"] & \\
    & & & \mathrm{Ab}^{op} \\
\end{tikzcd}\]

\section{Cup-product}


Let \(X\) be a simplicial set, and \(R\)\footnote{A ring is not necessarily commutative, but has a unit} a ring\index{ring}.
\[C^n(X, R) = \mathrm{maps}(X_n, R)\]
is an abelian group under pointwise addition. There is a differential
\[d^n\colon C^n(X,R) \to C^{n+1}(X,R)\]
given by
\[d^n(f)(y) = \sum_{i = 0}^{n+1}(-1)^{i} f(d_i^*(y))\]
with \(f\colon X_n \to R, y \in X_{n+1}\)


\begin{construction}[Cup product/Alexander Withney map]
    \index{cup-product}\index{Alexander Withney map}
    The cup prodcut/Alexander Withney map
    \[\cup\colon C^n(X,R) \times C^m(X,R) \to C^{m+n}(X,R)\]
    with \(n,m \geq 0\) is defined by
    \[(f\cup g)(x)\coloneq f(d_{front}^*(x)) \cdot g(d_{back}^*(x))\]
    with \(f\colon X_n \to R, g\colon X_m \to R\), \(x \in X_{n+m}\).

    Where we use \([n+m] = \set{0,1,\dots, n+m}\) and \(d_{front} \colon [n] \to [n+m], d_{back}\colon [m]\to [n+m]\) are given by \(d_{front}(i) = i\), \(d_{back}(i) = n+i\). Note, that \(d_{front}\) and \(d_{back}\) respectively suppress in their noation \(n\) and \(m\).
\end{construction}

\begin{thm}{fundamental properties of cup product}{fpcp}
    The cup-product satisfies the following properties.
    \begin{enumerate}
        \item The AW-map is biadditive and satisfies a boundary formula:
        \[d(f\cup g) = (df) \cup g + (-1)^n f \cup (dg) \in C^{m+n+1}(X,R)\]
        \item Associativity: For \(h \in C^k(X, R)\), \((f \cup g) \cup h = f \cup (g\cup h) \in C^{n+m+k}(X,R)\).
        
        Let \(1 \in C^0(X,R)\) be the constant function \(1\colon X_0 \to R\) with value 1. Then \(1 \cup f = f\cup 1 = f\).
        \item Naturality: Let \(\alpha \colon Y \to X\) be a morphism of symplicial sets. Then
        \[\alpha^*(f \cup g) = \alpha^*(f) \cup \alpha^*(g), \quad \alpha^*(1) = 1.\]
        where \(\alpha^*\colon C^n(X,R) \to C^n(Y,R), \quad f \mapsto f\circ \alpha_n\).
    \end{enumerate}
\end{thm}

\begin{proof}\leavevmode
    \begin{itemize}
        \item Let \(d_{front}\colon [n] \to [n+m]\), \(d_{back}\colon [m] \to [n+m]\) be as in the definition of \(\cup\). Then
        \[d_i \circ d_{front} = \begin{cases}
            d_{front} \circ d_i & 0 \leq i \leq n+1 \\
            d_{front} & n+1 \leq i \leq n+m+1 \\
        \end{cases}\]
        and
        \[d_i \circ d_{back} = \begin{cases}
            d_{back} \circ d_i & 0 \leq i \leq n \\
            d_{back} \circ d_{i-n} & n \leq i \leq n+m+1 \\
        \end{cases}\]
        Note, that for \(n+1\) and \(n\) respectively the cases are the same.

        now
        \[\begin{split}
            d(f\cup g)(x) &= \sum_{i = 0}^{n+m+1}(-1)^{i} (f\cup g)(d_i^*(x)) \\
            &= \sum_{i = 0}^{n+m+1} (-1)^{i} \cdot f(d_{front}^*(x)) \cdot g(d_{back}^*(d_i^*(x))) \\
            = \sum_{i = 0}^{n} (-1)^{i} &\cdot f(d_{front}^*(d_i^*(x))) \cdot g(d_{back}^*(d_i^*(x))) + \sum_{j = 1}^{m+1} (-1)^{n+j} \cdot f(d_{front}^*(d_{j+n}^*(x))) \cdot g(d_{back}^*(d_{j+n}^*(x))) \\
            &= \sum_{i = 0}^{n+1}(-1)^{i} \cdot f(d_i^*(d_{front}^*(x))) \cdot g(d_{back}^*(x)) + \sum_{j = 0}^{m+1}(-1)^{n+j} f(d_{front}^*(x)) \cdot g(d_j^*(d_{back}^*(x))) \\
            &= d(f)(d_{front}^*(x)) \cdot g(d_{back}^*(x)) + (-1)^n \cdot f(d_{front}^*(x)) \cdot d(g)(d_{back}^*(x)) \\
            &= ((df) \cup g)(x) + (-1)^n \cdot (f\cup dg)(x)\\
            & = ((df) \cup g + (-1)^n \cdot f\cup (dg))(x)
        \end{split}\]

        \item For \(x \in X_{n+m+k}\) we see
        \[\begin{split}
            ((f\cup g) \cup h)(x) &= (f\cup g) (d_{front}^*(x)) \cdot h(d_{back}^*(x)) \\
            &= f(d_{front}^*(d_{front}^*(x))) \cdot g(d_{back}^*(d_{front}^*(x))) \cdot h(d_{back}^*(x))\\
            & = f(d_{front}^*(x)) \cdot g(d_{middle}^*(x)) \cdot h(d_{back}^*(x))
        \end{split}\]
        Note that we abuse that \(d_{front}\) suppresses the indices for which the map is the front map.
        We have in the last line
        \[d_{front}\colon[n] \to [n+m+k], d_{middle}\colon [m] \to [n+m+k], d_{back}\colon [k] \to [n+m+k]\]
        defined by
        \[d_{front}(i) = i, d_{middle}(i) = n+i, d_{back}(i) = n+m+i\]
        this is obviously associative in the inputs\footnote{for Schwede at least.}
        \item Naturality for \(\alpha\colon Y \to X\) we see
        \[\begin{split}
            (\alpha^*(f\cup g))(y) &= (f\cup g)(\alpha_{n+m}(y)) \\
            &= f(d_{front}^*(\alpha_{n+m}(y))) \cdot g(d_{back}^*(\alpha_{n+m}(y))) = f(\alpha_n(d_{front}^*(y))) \cdot g(\alpha_m(d_{back}^*(y))) \\
            &= \alpha^*(f)(d_{front}^*(y)) \cdot \alpha^*(g)(d_{back}^*(y)) \\
            &= (\alpha^*(f) \cup \alpha^*(g))(y).
        \end{split}\]
    \end{itemize}
\end{proof}

\begin{defi}{Differential graded ring}{}
    A differential graded ring (dg-ring)\index{differential graded ring}\index{graded ring} is a cochain-complex \(A = \set{A^n, d^n}_{n \in \IZ}\) equipped with biadditive maps
    \[\cdot\colon A^n \times A^m \to A^{n+m}, \quad n,m \in \IZ \]
    and a unit \(1 \in A^0\), such that;
    \begin{itemize}
        \item \(\cdot\) is associative and has \(1\) as a unit element.
        \item the Leibniz rule holds:
        \[d(a\cdot b) = (da) \cdot b + (-1)^n \cdot a \cdot (db)\]
        with \(a \in A^n, b \in A^m\).\footnote{The sign is somehow connected to a sign-rule I couldn't follow. The d moved past the a or something.}
    \end{itemize}
\end{defi}

\begin{example}
    Some Differential graded rings are:
    \begin{itemize}
        \item \(C^\cdot(X,R)\) for a simplicial set \(X\) and a ring \(R\).
        \item De Rham complex of a smooth manifold.
    \end{itemize}

\end{example}

\begin{construction}[Cup-Product on cohomology]\index{cup-product}
    Let \(A = (A^n, d, \cdot)\) be a dg-ring. We define a map
    \[\cdot \colon H^n(A) \times H^m(A) \to H^{n+m}(A), \quad [a] \cdot [b] = [a \cdot b]\]

    This is well defined:
    \[d(a\cdot b) = \underset{= 0}{(da)} \cdot b + (-1)^n @. a \cdot \underset{= 0}{(db)} = 0\]
    so \(a\cdot b\) is a cycle and we can take its homology class. Let \(x \in A^{n-1}\).
    \[(a+ dx) \cdot b = a\cdot b + (dx) \cdot b = a \cdot b + d(x\cdot b) = [(a+dx)\cdot b] = [a \cdot b]\]
    so it only depends on the cohomology class of \(a\), analogous for \(b\).

    The product on cohomology inherits associativity and unity with \(1 = [1] \in H^0(A)\). We need to see \(1\) is a cocycle:
    \[d(1) = d(1\cdot 1) = (d1) \cdot 1 + (-1)^0 1 \cdot (d1) = 2 \cdot d(1)\]
    and so \(d(1) = 0\).

    The cup product on the \(R\)-cohomology of a simplicial set \(X\) is the product induced by the cup product on \(C^*(X,R)\) in \(H^*(C(X,R)) = H^*(X,R)\).
\end{construction}

\begin{thm}{Properties of the cup-product on homology}{}
    Let \(X\) be a simplicial set and \(R\) a ring. Then
    \begin{itemize}
        \item The cup product on \(H^*(X,R)\) is associative and unital, with unit the cohomology class of the constant function \(1\colon X_0 \to R\).
        \item For a morphism of simplicial sets \(\alpha\colon Y \to X\), the relation
        \[\alpha^*([x] \cup [y]) = \alpha^*[X] \cup \alpha^*[y]\]
        holds for all \([x] \in H^n(X,R), [y] \in H^m(X,R)\).
    \end{itemize}
\end{thm}

\begin{remark}
    The cup product generalizes to relative cohomology: For \(A, B\) simplicial subsets of \(X\). We have
    \[C^n(X,A;R) = \set{f\colon X_n\to R \mid f(A_n) = \set{0}}\]
    The relative cup product is the restriciton of \(\cup\) on \(C^*(X,R)\) to
    \[C^n(X,A;R) \times C^m(X,B;R) \xrightarrow{u} C^{n+m}(X,A\cup B;R).\]
    Let \(x \in (A \cup B)_{n+m}\), then
    \[(f\cup g)(x) = f(d_{front}^*(x)) \cdot g(d_{back}^*(x))\]
    if \(x \in A_{n+m}\) then \(f(d_{front}^*(x)) = 0\) and analogous with \(B_{n+m}\), anyways the product is 0.

    This gives us biadditive well defined maps
    \[\cup \colon H^n(X,A;R) \times H^n(X,B;R) \to H^{n+m}(X,A\cup B; R)\]

    In particular for \(A = B\) we get
    \[\cup\colon H^n(X,A;R) \times H^n(X,A;R) \to H^{n+m}(X,A;R)\]
    which is well defined and associative, but not unital anymore.
\end{remark}

\section{Commutativity of the cup-product}

\begin{thm}{Commutativity of the cup-product}{}
    Let \(X\) be a simplicial set and \(R\) a commutative ring. Then for all \([x] \in H^n(X,R); [y] \in H^m(X,R)\) the realtion
    \[[x] \cup [y] = (-1)^{n\cdot m} \cdot [y] \cup [x]\]
    holds.
\end{thm}

Schwede points out, that the easy way doesn't work.
\textbf{Warning.} For \(f \in C^n(X,R), g \in C^m(Y,R)\), then in general \(f\cup g \neq (-1)^{n+m} (g\cup f)\) in \(C^{n+m}(X,R)\). The commutativity is a property we only get on homology.


\begin{construction}
    The \(\cup_1\)-product (spoken Cup-one)
    \[\cup_1\colon C^n(X,R) \times C^m(X,R) \to C^{n+m-1}(X,R)\]
    is defined by
    \[(f\cup_1 g)(x) = \sum_{i =0}^{n-1} (-1)^{(n-1)\cdot (m+1)} f((d_i^{out})^*(x)) \cdot g((d_i^{inner})^*(x))\]
    for \(f\in C^n\), \(g \in C^m\) and \(x \in X_{n+m-1}\).\footnote{There are also \(\cup_i\) for \(i \in \IN\). However, they are quite messy and combinatorical.}
    where \(d_i^{out}\colon [n] \to [n+m-1], d_i^{inner}\colon [m] \to [n+m-1]\) are the unique monotone injective maps with images \(\img(d_i^{out}) = \set{0, \dots, i} \cup \set{i+m, \dots, n+m-1}\) and \(\img(d_i^{inn}) = \set{i, \dots, i+m}\).
\end{construction}

\begin{thm}{\(\cup_1\)-Product}{}
    The \(\cup_1\)-product satisfies the following formula
    \[d(f\cup_1 g) = (df) \cup_1 g + (-1)^n \cdot f \cup_1 (dg) - (-1)^{n+m}(f\cup g) - (-1)^{n+1}{m+1}(g\cup f)\]
    for \(f \in C^n(X,R)\) and \(g \in C^m(X,R)\).
\end{thm}

\begin{remark}
    What we want to see, is that \(f\cup g\) and \(g\cup f\) are not the same but rather homotopic, and \(\cup_1\) wittnesses that homotopy.
\end{remark}

\begin{proof}
    This theorem will not be prooven, because it is quite messy. You should find a lecture-video for that.
\end{proof}

Now suppose that \(f\) and \(g\) are cocycles, i.e. \(df = 0\), \(dg = 0\). Then
\[d(f\cup_1 g) = -(-1)^{n+m}(f\cup g) - (-1)^{(n+1)(m+1)}(g\cup f) \]
and we get
\[(-1)^{n+m+1}\cdot d(f\cup_1 g) = f\cup g - (-1)^{n \cdot m} (g\cup f)\] and as such
\[0 = [(-1)^{n+m-1}] = [f] \cup [g] - (-1)^{n\cdot m} [g] \cup [f]\]

\begin{remark}
    Last term we discussed the tensor product of two chain complexes (in an exercise):
    \[(C\otimes D)_n = \bigoplus_{p+q = n} C_p \otimes D_q\]
    and differential
    \[d(x\otimes y) = (dx) \otimes y + (-1)^{\abs{x}} \cdot x \otimes (dy)\]
\end{remark}

\begin{remark}
    Reinterpretation of \(d(f \cup_1 g)\).
    The cup product yields a morphism of cochain complexes
    \[C^*(X,R) \otimes C^*(X,R) \to C^*(X,R)\]
    and we get a diagram
    \[\begin{tikzcd}
        x \otimes y \ar[d] & C^*(X,R) \otimes C^*(X,R) \ar[r, "\cup"] \ar[d] & C^*(X,R) \\
        y \otimes x & C^*(X,R) \otimes C^*(X,R) \ar[ru, "\cup"] & \\
    \end{tikzcd}\]
    that does not commute, however it does so up to cochain homotopy and \(\cup_1\) is exactly a cochain homotopy between the two maps.
\end{remark}

\end{document}