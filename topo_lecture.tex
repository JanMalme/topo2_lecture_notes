\documentclass[language=english]{TemplateLecture}

\renewcommand{\ProfName}{Stefan Schwede}
\renewcommand{\LectureName}{Topology II}
\renewcommand{\Semester}{SoSe 2025}
\renewcommand{\mName}{Jan Malmström}

\begin{document}

\setcounter{chapter}{0}
\part{Cup product and Künneth theorem}

\newLecture{07.04.2025}

In last term, we discussed
\begin{itemize}
    \item CW-complexes
    \item higher homotopy groups
    \item Whitehead theorem 
    \item Singular homology
    \item cellular homology
\end{itemize}

In the very end, cohomology was started. Remeber
\[\begin{tikzcd}
    & & & \mathrm{Ab} \\
    \mathrm{TOP} \ar[rrru, bend left, "H_n(X;A)"] \ar[rrrd, bend right, "H^n(X;A)"] \ar[r, "\cS"] & (\mathrm{simpl. Sets}) \ar[r, "{C(\_, \IZ)}"] & (\mathrm{Chains}) \ar[ru, "H_n(\_ \otimes A)"] \ar[rd, "{H^n(\Hom(\_, A))}"] & \\
    & & & \mathrm{Ab}^{op} \\
\end{tikzcd}\]

\chapter{Cup-product}


Let \(X\) be a simplicial set, and \(R\)\footnote{A ring is not necessarily commutative, but has a unit} a ring\index{ring}.
\[C^n(X, R) = \mathrm{maps}(X_n, R)\]
is an abelian group under pointwise addition. There is a differential
\[d^n\colon C^n(X,R) \to C^{n+1}(X,R)\]
given by
\[d^n(f)(y) = \sum_{i = 0}^{n+1}(-1)^{i} f(d_i^*(y))\]
with \(f\colon X_n \to R, y \in X_{n+1}\)


\begin{construction}[Cup product/Alexander Whitney map]
    \index{cup-product}\index{Alexander Whitney map}
    The cup prodcut/Alexander Withney map
    \[\cup\colon C^n(X,R) \times C^m(X,R) \to C^{m+n}(X,R)\]
    with \(n,m \geq 0\) is defined by
    \[(f\cup g)(x)\coloneq f(d_{front}^*(x)) \cdot g(d_{back}^*(x))\]
    with \(f\colon X_n \to R, g\colon X_m \to R\), \(x \in X_{n+m}\).

    Where we use \([n+m] = \set{0,1,\dots, n+m}\) and \(d_{front} \colon [n] \to [n+m], d_{back}\colon [m]\to [n+m]\) are given by \(d_{front}(i) = i\), \(d_{back}(i) = n+i\). Note, that \(d_{front}\) and \(d_{back}\) respectively suppress in their noation \(n\) and \(m\).
\end{construction}

\begin{thm}{fundamental properties of cup product}{fpcp}
    The cup-product satisfies the following properties.
    \begin{enumerate}
        \item The AW-map is biadditive and satisfies a boundary formula:
        \[d(f\cup g) = (df) \cup g + (-1)^n f \cup (dg) \in C^{m+n+1}(X,R)\]
        \item Associativity: For \(h \in C^k(X, R)\), \((f \cup g) \cup h = f \cup (g\cup h) \in C^{n+m+k}(X,R)\).
        
        Let \(1 \in C^0(X,R)\) be the constant function \(1\colon X_0 \to R\) with value 1. Then \(1 \cup f = f\cup 1 = f\).
        \item Naturality: Let \(\alpha \colon Y \to X\) be a morphism of symplicial sets. Then
        \[\alpha^*(f \cup g) = \alpha^*(f) \cup \alpha^*(g), \quad \alpha^*(1) = 1.\]
        where \(\alpha^*\colon C^n(X,R) \to C^n(Y,R), \quad f \mapsto f\circ \alpha_n\).
    \end{enumerate}
\end{thm}

\begin{proof}\leavevmode
    \begin{enumerate}
        \item We check some properties: Let \(d_{front}\colon [n] \to [n+m]\), \(d_{back}\colon [m] \to [n+m]\) be as in the definition of \(\cup\). Then
        \[d_i \circ d_{front} = \begin{cases}
            d_{front} \circ d_i & 0 \leq i \leq n+1 \\
            d_{front} & n+1 \leq i \leq n+m+1 \\
        \end{cases}\]
        and
        \[d_i \circ d_{back} = \begin{cases}
            d_{back} \circ d_i & 0 \leq i \leq n \\
            d_{back} \circ d_{i-n} & n \leq i \leq n+m+1 \\
        \end{cases}\]
        Note, that for \(n+1\) and \(n\) respectively the cases are the same.

        Now we calculate
        \[\begin{split}
            d(f\cup g)(x) &= \sum_{i = 0}^{n+m+1}(-1)^{i} (f\cup g)(d_i^*(x)) \\
            &= \sum_{i = 0}^{n+m+1} (-1)^{i} \cdot f(d_{front}^*(x)) \cdot g(d_{back}^*(d_i^*(x))) \\
            = \sum_{i = 0}^{n} (-1)^{i} &\cdot f(d_{front}^*(d_i^*(x))) \cdot g(d_{back}^*(d_i^*(x))) + \sum_{j = 1}^{m+1} (-1)^{n+j} \cdot f(d_{front}^*(d_{j+n}^*(x))) \cdot g(d_{back}^*(d_{j+n}^*(x))) \\
            &= \sum_{i = 0}^{n+1}(-1)^{i} \cdot f(d_i^*(d_{front}^*(x))) \cdot g(d_{back}^*(x)) + \sum_{j = 0}^{m+1}(-1)^{n+j} f(d_{front}^*(x)) \cdot g(d_j^*(d_{back}^*(x))) \\
            &= d(f)(d_{front}^*(x)) \cdot g(d_{back}^*(x)) + (-1)^n \cdot f(d_{front}^*(x)) \cdot d(g)(d_{back}^*(x)) \\
            &= ((df) \cup g)(x) + (-1)^n \cdot (f\cup dg)(x)\\
            & = ((df) \cup g + (-1)^n \cdot f\cup (dg))(x)
        \end{split}\]

        \item For \(x \in X_{n+m+k}\) we see
        \[\begin{split}
            ((f\cup g) \cup h)(x) &= (f\cup g) (d_{front}^*(x)) \cdot h(d_{back}^*(x)) \\
            &= f(d_{front}^*(d_{front}^*(x))) \cdot g(d_{back}^*(d_{front}^*(x))) \cdot h(d_{back}^*(x))\\
            & = f(d_{front}^*(x)) \cdot g(d_{middle}^*(x)) \cdot h(d_{back}^*(x))
        \end{split}\]
        Note that we abuse that \(d_{front}\) suppresses the indices for which the map is the front map.
        We have in the last line
        \[d_{front}\colon[n] \to [n+m+k], d_{middle}\colon [m] \to [n+m+k], d_{back}\colon [k] \to [n+m+k]\]
        defined by
        \[d_{front}(i) = i, d_{middle}(i) = n+i, d_{back}(i) = n+m+i\]
        this is obviously associative in the inputs\footnote{for Schwede at least.}
        \item Naturality for \(\alpha\colon Y \to X\) we see
        \[\begin{split}
            (\alpha^*(f\cup g))(y) &= (f\cup g)(\alpha_{n+m}(y)) \\
            &= f(d_{front}^*(\alpha_{n+m}(y))) \cdot g(d_{back}^*(\alpha_{n+m}(y))) = f(\alpha_n(d_{front}^*(y))) \cdot g(\alpha_m(d_{back}^*(y))) \\
            &= \alpha^*(f)(d_{front}^*(y)) \cdot \alpha^*(g)(d_{back}^*(y)) \\
            &= (\alpha^*(f) \cup \alpha^*(g))(y).
        \end{split}\]
    \end{enumerate}
\end{proof}

\begin{defi}{Differential graded ring}{}
    A differential graded ring (dg-ring)\index{differential graded ring}\index{graded ring} is a cochain-complex \(A = \set{A^n, d^n}_{n \in \IZ}\) equipped with biadditive maps
    \[\cdot\colon A^n \times A^m \to A^{n+m}, \quad n,m \in \IZ \]
    and a unit \(1 \in A^0\), such that;
    \begin{itemize}
        \item \(\cdot\) is associative and has \(1\) as a unit element.
        \item the Leibniz rule holds:
        \[d(a\cdot b) = (da) \cdot b + (-1)^n \cdot a \cdot (db)\]
        with \(a \in A^n, b \in A^m\).\footnote{The sign is somehow connected to a sign-rule I couldn't follow. The d moved past the a or something.}
    \end{itemize}
\end{defi}

\begin{example}
    Some Differential graded rings are:
    \begin{itemize}
        \item \(C^\cdot(X,R)\) for a simplicial set \(X\) and a ring \(R\).
        \item De Rham complex of a smooth manifold.
    \end{itemize}

\end{example}

\begin{construction}[Cup-Product on cohomology]\index{cup-product}
    Let \(A = (A^n, d, \cdot)\) be a dg-ring. We define a map
    \[\cdot \colon H^n(A) \times H^m(A) \to H^{n+m}(A), \quad [a] \cdot [b] = [a \cdot b]\]

    This is well defined:
    \[d(a\cdot b) = \underset{= 0}{(da)} \cdot b + (-1)^n @. a \cdot \underset{= 0}{(db)} = 0\]
    so \(a\cdot b\) is a cycle and we can take its homology class. Let \(x \in A^{n-1}\).
    \[(a+ dx) \cdot b = a\cdot b + (dx) \cdot b = a \cdot b + d(x\cdot b) = [(a+dx)\cdot b] = [a \cdot b]\]
    so it only depends on the cohomology class of \(a\), analogous for \(b\).

    The product on cohomology inherits associativity and unity with \(1 = [1] \in H^0(A)\). We need to see \(1\) is a cocycle:
    \[d(1) = d(1\cdot 1) = (d1) \cdot 1 + (-1)^0 1 \cdot (d1) = 2 \cdot d(1)\]
    and so \(d(1) = 0\).

    The cup product on the \(R\)-cohomology of a simplicial set \(X\) is the product induced by the cup product on \(C^*(X,R)\) in \(H^*(C(X,R)) = H^*(X,R)\).
\end{construction}

\begin{thm}{Properties of the cup-product on homology}{}
    Let \(X\) be a simplicial set and \(R\) a ring. Then
    \begin{itemize}
        \item The cup product on \(H^*(X,R)\) is associative and unital, with unit the cohomology class of the constant function \(1\colon X_0 \to R\).
        \item For a morphism of simplicial sets \(\alpha\colon Y \to X\), the relation
        \[\alpha^*([x] \cup [y]) = \alpha^*[X] \cup \alpha^*[y]\]
        holds for all \([x] \in H^n(X,R), [y] \in H^m(X,R)\).
    \end{itemize}
\end{thm}

\begin{remark}
    The cup product generalizes to relative cohomology: For \(A, B\) simplicial subsets of \(X\). We have
    \[C^n(X,A;R) = \set{f\colon X_n\to R \mid f(A_n) = \set{0}}\]
    The relative cup product is the restriciton of \(\cup\) on \(C^*(X,R)\) to
    \[C^n(X,A;R) \times C^m(X,B;R) \xrightarrow{\cup} C^{n+m}(X,A\cup B;R).\]
    Let \(x \in (A \cup B)_{n+m}\), then
    \[(f\cup g)(x) = f(d_{front}^*(x)) \cdot g(d_{back}^*(x))\]
    if \(x \in A_{n+m}\) then \(f(d_{front}^*(x)) = 0\) and analogous with \(B_{n+m}\), anyways the product is 0.

    This gives us biadditive well defined maps
    \[\cup \colon H^n(X,A;R) \times H^n(X,B;R) \to H^{n+m}(X,A\cup B; R)\]

    In particular for \(A = B\) we get
    \[\cup\colon H^n(X,A;R) \times H^n(X,A;R) \to H^{n+m}(X,A;R)\]
    which is well defined and associative, but not unital anymore.
\end{remark}

\section{Commutativity of the cup-product}

\begin{thm}{Commutativity of the cup-product}{}
    Let \(X\) be a simplicial set and \(R\) a commutative ring. Then for all \([x] \in H^n(X,R); [y] \in H^m(X,R)\) the realtion
    \[[x] \cup [y] = (-1)^{n\cdot m} \cdot [y] \cup [x]\]
    holds.
\end{thm}

Schwede points out, that the easy way doesn't work.
\textbf{Warning.} For \(f \in C^n(X,R), g \in C^m(Y,R)\), then in general \(f\cup g \neq (-1)^{n+m} (g\cup f)\) in \(C^{n+m}(X,R)\). The commutativity is a property we only get on homology.


\begin{construction}
    The \(\cup_1\)-product (spoken Cup-one)
    \[\cup_1\colon C^n(X,R) \times C^m(X,R) \to C^{n+m-1}(X,R)\]
    is defined by
    \[(f\cup_1 g)(x) = \sum_{i =0}^{n-1} (-1)^{(n-1)\cdot (m+1)} f((d_i^{out})^*(x)) \cdot g((d_i^{inner})^*(x))\]
    for \(f\in C^n\), \(g \in C^m\) and \(x \in X_{n+m-1}\).\footnote{There are also \(\cup_i\) for \(i \in \IN\). However, they are quite messy and combinatorical.}
    where \(d_i^{out}\colon [n] \to [n+m-1], d_i^{inner}\colon [m] \to [n+m-1]\) are the unique monotone injective maps with images \(\img(d_i^{out}) = \set{0, \dots, i} \cup \set{i+m, \dots, n+m-1}\) and \(\img(d_i^{inn}) = \set{i, \dots, i+m}\).
\end{construction}

\begin{thm}{\(\cup_1\)-Product}{}
    The \(\cup_1\)-product satisfies the following formula
    \[d(f\cup_1 g) = (df) \cup_1 g + (-1)^n \cdot f \cup_1 (dg) - (-1)^{n+m}(f\cup g) - (-1)^{n+1}{m+1}(g\cup f)\]
    for \(f \in C^n(X,R)\) and \(g \in C^m(X,R)\).
\end{thm}

\begin{remark}
    What we want to see, is that \(f\cup g\) and \(g\cup f\) are not the same but rather homotopic, and \(\cup_1\) wittnesses that homotopy.
\end{remark}

\begin{proof}
    This theorem will not be prooven, because it is quite messy. You should find a lecture-video for that.
\end{proof}

Now suppose that \(f\) and \(g\) are cocycles, i.e. \(df = 0\), \(dg = 0\). Then
\[d(f\cup_1 g) = -(-1)^{n+m}(f\cup g) - (-1)^{(n+1)(m+1)}(g\cup f) \]
and we get
\[(-1)^{n+m+1}\cdot d(f\cup_1 g) = f\cup g - (-1)^{n \cdot m} (g\cup f)\] and as such
\[0 = [(-1)^{n+m-1}] = [f] \cup [g] - (-1)^{n\cdot m} [g] \cup [f]\]

\begin{remark}
    Last term we discussed the tensor product of two chain complexes (in an exercise):
    \[(C\otimes D)_n = \bigoplus_{p+q = n} C_p \otimes D_q\]
    and differential
    \[d(x\otimes y) = (dx) \otimes y + (-1)^{\abs{x}} \cdot x \otimes (dy)\]
\end{remark}

\begin{remark}
    Reinterpretation of \(d(f \cup_1 g)\).
    The cup product yields a morphism of cochain complexes
    \[C^*(X,R) \otimes C^*(X,R) \to C^*(X,R)\]
    and we get a diagram
    \[\begin{tikzcd}
        x \otimes y \ar[d] & C^*(X,R) \otimes C^*(X,R) \ar[r, "\cup"] \ar[d] & C^*(X,R) \\
        y \otimes x & C^*(X,R) \otimes C^*(X,R) \ar[ru, "\cup"] & \\
    \end{tikzcd}\]
    that does not commute, however it does so up to cochain homotopy and \(\cup_1\) is exactly a cochain homotopy between the two maps.
\end{remark}


\newLecture{09.04.2025}


Only with the definition of the cup-product we cannot calculate a lot yet. Some methods to compute cup-products are:
\begin{itemize}
    \item directly from the definition
    \item cellular approximation of the diagonal (whatever that means, he gives a little intuition I failed to record.) (this might be used later)
    \item Group homology (one exapmle later today, something for AT I)
    \item Poincaré duality (later this term)
    \item Analysis on smooth manifolds together with De Rahm Cohomology 
\end{itemize}
The first two methods are not very practical.

\begin{example}
    Let \(X\) be a discrete space, Then \(\cS(X)\) is a constant simplicial set. The chain complex has the form
    \[\begin{tikzcd}
        \ar[r, "0"] & \IZ[X] \ar[r, "="] & \IZ[X] \ar[r, "0"] & \IZ[X]\\
    \end{tikzcd}\]
    And so \(H^n(X,R) = 0\) for \(n \geq 0\).
    And only for \(n = m = 0\) something nontrivial happens. for \(f\colon X_0 \to R, g\colon X_0 \to R\), we have \((f\cup g)(x) = f(d_{front}^*(x)) \cdot g(d_{back}^*(x)) = f(x) \cdot g(x)\)
    and so the cup product is just pointwise multiplication in dimension \(0\).

    More generally: \(H^0(X,R) = \mathrm{maps}(\pi_0(X),R)\) with \(\cup\)-prodcut pointwise multiplication
\end{example}

\begin{example}
    Let \(G\) be a group: Define a category \(\underline{G}\)\footnote{via geometric realization, these define interesting spaces, namely some (missed word)-Maclane spaces \(M(G,1)\), didn't catch it all} wit one object \(*\) and \(\Hom_{\underline{G}}(*,*) = G\). We then define
    \[BG = N(\underline{G})\]
    Where \(N\) is the Nerve-Functor \(\mathbf{CAT} \to \mathbf{Sset}\).
    Then
    \[(BG)_n = G^n, \quad d_i^*\colon G^n \to G^{n-1} (g_1, \dots, g_n) \mapsto \begin{cases}
        (g_2, \dots, g_n) & i = 0 \\
        (g_1, \dots, g_i \circ g_{i+1}, \dots, g_n) & 1 \leq i \leq n-1 \\
        (g_1, \dots , g_{n-1}) & i = n \\
    \end{cases}\]
    And \(s_i(g_1, \dots, g_n) = (g_1, \dots, g_i, 1, g_{i+1}, \dots, g_n)\).

    The general case of this is too hard to calculate. We take \(G = (\IF_2, +)\) and \(R = \IF_2\) and we calculate \(H^*(B\IF_2, \IF_2)\).
    We see
    \[\begin{tikzcd}
        {C^0(BG,A)} \ar[r, "d"] \ar[d, phantom, sloped, "= "] & C^1(BG,A) \ar[r, "d"] \ar[d, phantom, sloped, "="] & C^2(BG,A) \ar[r] \ar[d, phantom, sloped, "="] & \dots \\
        \mathrm{maps}(\{1\},A) \ar[r, "0"] \ar[d, sloped, phantom, "\cong"] & \mathrm{maps}(G,A) \ar[r] & \mathrm{maps}(G^2,A) & \\
        A &(f\colon G \to A) \ar[r]&(df)(g,h) & \\
    \end{tikzcd}\]
    And the map is defined by
    \[f(d_0^*(g,h)) - f(d_1^*(g,h)) + f(d_2^*(g,h)) = f(h) - f(g\cdot h) + f(g)\]
    and 
    \[df = 0 \Lra f(g,h) = f(g) + f(h)\]
    \(\implies\) 1-cocycles are the group homomorphisms from \(G\) to \(A\)
    \[H^1(BG,A) \cong \Hom(G,A)\]
    and  for \(G = (\IF_2, +)\), \(A = \IF_2\)
    
    We define
    \[0 \neq x \coloneq [\Id_{\IF_2}] \in H^1(B\IF_2, \IF_2).\]
    We will show that \(x^n = x \cup \dots \cup x\) (\(n\)-times) \(\in H^n(B\IF_2, \IF_2)\) is nonzero.
    \begin{Proposition}
        \(x^n \in H^n(B\IF_2, \IF_2)\) is represented by
        \[f_n \colon (\IF_2)^n \to \IF_2, f_n(\lambda_1, \dots, \lambda_n) = \lambda_1 \cdot \dots \cdot \lambda_n = \begin{cases}
            1 & \text{if } \lambda_1 = \lambda_2 = \dots = \lambda_n = 1 \\
            0 & \text{else}
        \end{cases}\]
    \end{Proposition}
    \begin{proof}
        By induction on \(n\). We checked for \(n = 1\).
        For \(n \geq 2\) we have
        \[\begin{split}
            x^n &= x^{n-1} \cup x = [f_{n-1}] \cup [\Id_{\IF_2}] \\
            &= [f_{n-1} \cup \Id] \\
        \end{split}\]
        Then
        \[\begin{split}
            (f_{n-1} \cup \Id)(\lambda_1 , \dots, \lambda_n) &= f_{n-1}(d_{front}^*(\lambda_1, \dots, \lambda_n)) \cdot \Id_(d_{back}^*(\lambda_1, \dots, \lambda_n)) \\
            &= f_{n-1}(\lambda_1, \dots, \lambda{n-1}) \cdot \Id(\lambda_n) \\
            &= (\lambda_1 \cdot \dots \cdot \lambda_{n-1}) \cdot \lambda_n \\
        \end{split}\]
    \end{proof}

    Claim: \(x^n \neq 0\). In the UCT for cohomology we used the evaluation pair
    \[\Phi\colon H^n(X,A) \to \Hom(H_n(X;\IZ);A), \quad [f_n\colon X_n \to A] \mapsto \set*{[\sum b_ix_i] \mapsto \sum b_i f(x_i)}\]
    for \(b_i \in \IZ, x_i \in X_n\).
    We can slightly variate that for ring coefficients:
    \[\Phi\colon H^n(X,R) \to \Hom(H_n(X,R), R)\]
    and \([f\colon X_n \to R] \mapsto \set{[\sum r_i \cdot x_i] \mapsto \sum r_i \cdot f(x_i)}\) with \(r_i \in R, x_i \in X_n\).

    With \(X = B\IF_2, R = \IF_2\), we consider
    \[y\coloneq \sum_{(\lambda_1 \dots, \lambda_n) \in (\IF_2)^n} 1 (\lambda_1, \dots, \lambda_n) \in \IF_2[(\IF_2)^n] = \IF_2[(B\IF_2)_n]\]
    Claim: \(y\) is an \(n\)-cycle in \(C_*(B\IF_2, \IF_2)\).
    \[\begin{split}
        dy &= \sum_{i = 0, \dots n} (-1)^{i} \cdot d_i^*(\sum_1\cdot (\lambda_1, \dots, \lambda_n)) \\
        &= \sum_{i = 0,\dots, n} \underbrace{\sum_{(\lambda_1, \dots, \lambda_n) \in \IF_2^n} (-1)^{i} \cdot d_i^*(\lambda_1, \dots, \lambda_n)}_{\text{cancel in pairs}} \\
        &= 0
    \end{split}\]
    Now
    \[d_0^*(0,\lambda_2, \dots, \lambda_n) = (\lambda_2, \dots \lambda_n) = d_0^*(1, \lambda_2, \dots, \lambda_n)\]
    So
    \[\Phi(x^n) \colon H_n(B\IF_2, \IF_2) \to \IF_2\]
    \[\Phi(x^n)[y] = \Phi[f_n][\sum_{(\lambda_1, \dots, \lambda_n) \in \IF_2^n}(\lambda_1, \dots, \lambda_n)] = \sum_{(\lambda_1, \dots, \lambda_n)} f_n(\lambda_1, \dots, \lambda_n) = \sum_{(\lambda_1, \dots, \lambda_n)} \lambda_1,\cdot \dots \cdot \lambda_n = 1 \neq 0\]
    and \([y] \neq 0\) in \(H_n(B\IF_2, \IF_2)\).

    We will later see, that in fact \(H^*(B\IF_2;\IF_2) = \IF_2[X]\).

    \begin{Remark}
        Let \(p\) be an odd prime. \(H^*(B\IF_p, \IF_p) = ?\).
        \[0 \neq x = [\Id_{\IF_p} \in H^1(B\IF_p; \IF_p)]\]
        still makes sense, but now there are more scalars and
        \[x^n = 0\]
        for \(n \geq 2\). The graded commutativity says:
        \[x\cup x = (-1)^{1\cdot 1} x \cup x = - x\cup x\]
        so if \(R\) is commutative, \(x \in H^n(X,R)\) and \(n\) is odd, then \(2\cdot(x\cup x) = 0\) in \(H^{2n}(X,R)\).
        And then \(2\cdot x^2 = 0 \Ra x^2 = 0\).

        Define \(h \colon \IF_p \times \IF_p \to \IF_p\) by 
        \[h(i,j) = \begin{cases}
            0 & \text{if } i + j < p \\
            1 & \text{if } i+ j \geq p \\
        \end{cases}\]
        where we write \(\IF_p = \set{0, \dots, p-1}\). Now \(h \in C^2(B\IF_p, \IF_p)\). Fact: \(dh = 0\) and \(0 \neq y \coloneq [h] \in H^2(B\IF_p, \IF_p)\).

        We then get (but do not proove)
        \[H^*(B\IF_p, \IF_p) = \Lambda(x) \otimes \IF_p[y]\] and
        \[H^{2n}(B\IF_p, \IF_p) = \IF_p \set{y^n}, \quad H^{2n+1}(B\IF_p, \IF_p) = \IF_p \set{xy^n}\]
    \end{Remark}
\end{example}

\chapter{Künneth theorem}
\index{Künneth theorem}
The Künneth theorem is an algebraic relationship between \(H^*_*(X,R), H^*_*(Y,R)\) and \(H^*_*(X\times Y, R)\)\footnote{\(H_*^*\) denotes, that Schwede was too lazy to write the statement for homology and cohomology separately}.

Here is a simplest version in homology with field coefficients:
\begin{thm}{Künneth, simple version}{}
    Let \(X\) and \(Y\) be spaces and \(k\) a field. Then
    \[H_n(X\times Y, k)\]
    is natural isomorphic to
    \[\bigoplus_{p+q = n} H_p(X,k) \otimes_k H_q(Y,k)\]
\end{thm}

\section{The Eilenberg-Zilber-theorem}

Let \(A,B\) be simplicial abelian groups. Then we get two natural chain homotopy equivalences
\[\begin{tikzcd}
    C_*(A) \otimes C_*(B) \ar[rr, bend left, "\text{Eilenberg-Zilber-Map}"] && C_*(A\otimes B) \ar[ll, bend left, "\text{Alexander Whitney map}"]
\end{tikzcd}\]

\begin{defi}{Simplicial abelian group}{}
    A \emph{simplicial abelian group} \index{simplicial abelian group} is a functor \(A\colon \Delta^{Op} \to \mathbf{Ab.Groups}\).
\end{defi}

\begin{Remark}
    Equivalently a simplicial abelian group is a colleciton of abelian groups \(A_n\), and homomorphisms \(\alpha^*\colon A_m \to A_n\) for all \(\alpha\colon [n] \to [m]\) in \(\Delta\), s.t. \((\alpha\circ \beta)^* = \beta^* \circ \alpha^*\).

    Equivalently a simplicial abelian group is a simplical set endorsed with abelian group structure on the sets of \(n\)-simplices, such that all \(\alpha^*\) are homomorphisms.
\end{Remark}


\begin{example}
    Let \(X\) be a simplicial set and \(A\) an abelian group. Then the composite
    \[\begin{tikzcd}
        \Delta^{op} \ar[rr, bend left, "{A[X]}"]\ar[r, "X"] & (\mathbf{Sets}) \ar[r, "{{A[\_]}}"] & (\mathbf{ab.grps}) \\
    \end{tikzcd}\]
    is a simplicial abelian group.
\end{example}

\begin{construction}
    Let \(A\colon \Delta^{op} \to (\mathbf{ab.grps})\) be a simplicial abelian groups. Its \emph{chain complex} \(C_*(A)\) is the chain complex with \(C_n(A) = A_n\) with differential
    \[d\colon C_n(A) = A_n \to A_{n-1} = C_{n-1}(A), \quad d(a) = \sum_{i = 0, \dots, n} (-1)^{i} d_i^*(a)\]
    And one can easily check \(d\circ d = 0\).
\end{construction}

\textbf{Note.} The following commutes
\[\begin{tikzcd}
    (\mathbf{Ssets}) \ar[rr, "X \mapsto {C_*(X,A)}"] \ar[rd, "{A[\_]}"] & & (\mathbf{Chains}) \\
    & (\mathbf{s.ab.grps}) \ar[ru, "C_*"] \\
\end{tikzcd}\]

\begin{remark}
    The tensor product of chain complexes \(C,D\) is
    \[(C\otimes D)_n \coloneq \bigoplus_{p+q = n} C_p \otimes D_q\]
    with differential
    \[d(x\otimes y) = (dx \otimes y) + (-1)^p x \otimes (dy)\]
    for \(x \in C_p, y \in D_q\).
\end{remark}

    We can also form the tensor product of simplical abelian groups\index{Tensor product of simplicial abelian groups}:
\begin{defi}{Tensor product of simplicial abelian groups}{}
    \(A,B \colon \Delta^{op} \to (\mathbf{ab.grps})\) by
    \[(A\otimes B)_n = A_n \otimes B_n, \quad \alpha^*\colon (A\otimes B)_n \to (A\otimes B)_m\]
    for \(\alpha\colon [m] \to [n]\) is defined as \(\alpha^*(a\otimes b) = \alpha^*(a) \otimes \alpha^*(b)\) and we write \(\alpha^*_{A\otimes B} \coloneq \alpha^*_A \otimes \alpha^*_B\).

    This can be equally described as the composite
    \[\Delta^{op}\xrightarrow{(A,B)} (\mathbf{ab.grps})\times (\mathbf{ab.grps}) \xrightarrow{\otimes} (\mathbf{ab.grps})\]
\end{defi}    

\textbf{Warning.} For \(A,B \in (\mathbf{SAB}) = \) simplicial abelian groups
\[C_*(A\otimes B) \neq C_*(A) \otimes C_*(B)\]
Also he did this in dimension \(n\), but I lacked time to copy.

The Eilenberg-Zilber theorem is a natural pair of chain homotopy equivalences between these two.

\begin{construction}
    Let \(A,B\) be simplicial chain groups. The \emph{Alexander-Whitney map} is the chain map
    \[AW\colon C_*(A\otimes B) \to C_*(A) \otimes C_*(B)\]
    defined by
    \[\begin{tikzcd}
        C_n(A\otimes B) \ar[d, phantom, sloped, "="]\ar[r] & \bigoplus\limits_{p+q = n, p,q \geq 0} A_p \otimes B_q \ar[d, phantom, sloped, "="] \\
        A_n \otimes B_n & C_*(A) \otimes C_*(B)
    \end{tikzcd}\]
    \[AW_n(a\otimes b) = \sum_{p+q = n} d_{front}^*(a) \otimes d_{back}^*(b)\]
    Where \([p] \xrightarrow{d_{front}} [p+q] = [n] \xleftarrow{d_{back}} [q]\).

    You may check for yourself, that this is a chain map, however Schwede didn't do that.
\end{construction}


\newLecture{14.04.2025}

\begin{Remark}
    An example for a simplicial abelian group, that is not of the form
    \[\Delta^{op} \xrightarrow{X} \mathbf{sets} \xrightarrow{A[\_]} (\mathbf{ab. grps.})\]
    is for any abelian group \(G\) the simplicial set \(BG\), that also admits structure of a simplicial abelian group.
\end{Remark}

\begin{remark}[Relation between AW-map and cup-product]
    For a simplicial set \(X\) and ring \(R\),
    \[C^*(X,R) = \Hom(C_*(X,\IZ), R) = \Hom (C_*(\IZ[X]),R)\]
    and \(C^n(X,R) = \Hom(C_n(X,\IZ), R)\).
    If \(\psi \in C^n(X,R)\) is a cocycle, i.e. \(d(\psi) = 0\), then it extends to a chain map
    \[\tilde{\psi}\colon C_*(\IZ[X]) \to R[n]\]
    where \(R[n]\) is the complex with \(R\) in dimension \(n\) and \(0\) otherwise.
    and \(\tilde{\psi}\) is \(\psi\) in dimension \(n\) and \(0\) otherwise.

    For \(f \in C^n(X,R), g\in C^m(X,R)\) cocycles, we have \(f\cup g \in C^{n+m}(X,R)\). Then \(\tilde{f\cup g}\) is the following composite
    \[\begin{tikzcd}
        {C_*(\IZ[X])} \ar[rr, "{C_*(\IZ[\text{diagonal}])}"] && {C_*(\IZ[X \times X])} \ar[r, "\cong", phantom] & {C_*(\IZ[X] \otimes \IZ[X])} \ar[dlll, "\text{AW}"] \\
        {C_*(\IZ[X]) \otimes C_*(\IZ[X])} \ar[r, "\tilde{f} \otimes \tilde{g}"] & {R[n] \otimes R[m]} \ar[r, "\text{mult}"] & {R[n+m]} \\
    \end{tikzcd}\]
\end{remark}

\begin{defi}{(p,q)-shuffle}{}
    A \((p,q)\)-shuffle\index{(p,q)-shuffle} for \(p,q \geq 0\) is a permutation \(\sigma\) of \(\set{0,1,\dots, p+q-1}\), such that the restriction of \(\sigma\) to \(\set{0,1,\dots, p-1}\) is monotone, and the restriction of \(\sigma\) to \(\set{p, \dots, p+q-1}\) is monotone.
\end{defi}

\begin{Remark}
    \enquote{Shuffles leave the first \(p\) elements in order and the last \(q\) elements in order.}
\end{Remark}

\begin{example}
    The only \((p,0)\)-shuffle or \((0,q)\)-shuffles are the identity.

    There are precisely two \((1,1)\)-shuffles, namely both permutations of \(\set{0,1}\).

    \(\sigma \in S_3\) given by \(\sigma(0) = 0 \sigma(1) = 2, \sigma(2) = 1\) is not a \((2,1)\)-shuffle, but it is a \((1,2)\)-shuffle.
\end{example}

\begin{remark}
    \((p,q)\)-shuffles biject with \(p\)-element subsets of \(\set{0,1,\dots, p+q-1}\) by \(\sigma\mapsto \set{\sigma(0), \dots, \sigma(p)}\) and also wit \(q\)-element subsets of \(\set{0,1, \dots, p+q-1}\) by \(\sigma\mapsto \set{\sigma(p),\dots, \sigma(p+q-1)}\).

    This means \(\abs{(p,q)\text{-shuffles}} = \binom{p+q}{p} = \binom{p+q}{q}\).
\end{remark}

\begin{notation}
    Let \(\sigma\) be a \((p,q)\)-shuffle. We write \(\mu_i\Coloneq \sigma(i-1)\) for \(1 \leq 1 \leq p\) and \(\nu_i \coloneq \sigma(p+i-1)\) for \(1 \leq i \leq q\).

    This means
    \(0 \leq \mu_1 \leq \dots \leq \mu_p\) and \(0 \leq \nu_1 \leq \dots \leq \nu_q \leq p+q-1\).
\end{notation}

\begin{defi}{Eilenberg-Zilber map}{}
    Let \(A,B\) be simplicial abelian groups. The Eilenberg-Zilber map \index{Eilenberg-Zilber map}/shuffle map \index{shuffle map} is
    \[EZ\colon C_*(A) \otimes C_*(B) \to C_*(A\otimes B)\]
    is the direct sum of the homomorphisms
    \[\nabla_{p,q} \colon A_p\otimes B_q \to A_{p+q} \otimes B_{p+q}\]
    given by
    \[a\otimes b \mapsto a \nabla b \coloneq \sum_{\sigma\colon (p,q)\text{-shuffle}} \sgn(\sigma) \cdot (s_{\nu_i} \circ \dots \circ s_{\nu_q})^*(a) \otimes (s_{\mu_1}\circ \dots \circ s_{\mu_p})^*(b)\]
\end{defi}

\begin{example}
    There is only one \((p,0)\)-shuffle, the identity of \(\set{0, \dots, p-1}\). Then \(\mu_i = i-1\).
    \[\nabla_{p,0}\colon A_p \otimes B_0 \to A_p \otimes B_p\]
    is defined by
    \[a\otimes b \mapsto a\nabla b = a \otimes (s_0 \circ \dots \circ s_{p-1})^*(b).\]

    For \(p = q = 1\) i didn't have the time to copy.
\end{example}

Schwede claims, that the Eilenberg-Zilber map is a chain map and he can't believe he actually did those calculations 4 years ago. He will not torture us, but you may watch the videos.

\begin{thm}{Shuffle maps form a chain map}{}
    The shuffle maps \(\nabla_{p,q}\) for varying \(p,q \geq 0\) assemble into a chain map. Furthermore, for \(a \in A_p, b \in B_q\)
    \[d(a\nabla b) = (da) \nabla b + (-1)^p a \nabla (db)\]
\end{thm}

He specifies, that the calculation takes up 8 pages of his notes.


\begin{thm}{Eilenberg-Zilber}{ElbZlb}
    Let \(A,B\) be simplicial abelian groups. Then the morphisms
    \[\begin{tikzcd}
        C_*(A) \otimes C_*(B) \ar[rr, bend left, "\text{Eilenberg-Zilber}"] && C_*(A\otimes B) \ar[ll, bend left, "AW"]
    \end{tikzcd}\]

    are mutually inverse natural chain homotopy equivalences.
\end{thm}

A first method of proof would be explicit formulas for the chain homotopies \(\text{AW} \circ EZ \sim \Id\) and \(EZ \circ AW \sim \Id\). That is however infinitely annoying and we will not do this.

For the special case, where \(A = \IZ[X], B= \IZ[Y]\) for simplicial sets \(X,Y\) we proove this via acyclic models. For that we need some category-theory:

\section{Yoneda Lemma \& Acyclic models}

\begin{thm}{Yoneda lemma}{}
    Let \(\cC\) be a category and \(c\) an object of \(\cC\). Let \(F\colon \cC \to (\mathbf{sets})\) be a functor: Then the evaluation map
    \[\Nat_{\cC \to \mathbf{sets}}(\cC(c,\_), F) \to F(c)\]
    given by
    \[(\tau\colon \cC(c,\_) \to F) \mapsto (\tau_c\colon \cC(c,c)\to F(c))(\id_c)\]
    is bijective.

    Equally: for every \(x \in F(c)\), there is a unique natural transformation \(\tau\colon (\cC(c, \_) \to F)\), such that \(\tau_c(\id_c) = x\).
\end{thm}

\begin{Remark}
    A special case of this is
    \[\Hom_{\mathbf{sset}}(\Delta^n, X) \cong X_n, \quad (f\colon \Delta
    ^n\to X) \mapsto f_n(\id_{[n]}).\]
    where \(\Delta^n = \Delta(\_, [n])\).
\end{Remark}

\begin{proof}
    We show injectivity and surjectivity.
    \begin{description}
        \item[Injectivity] Let \(\tau\colon \cC(c,\_) \to F\) be any natural transformation. Let \(d\) be another object of \(\cC\), \(f\colon c\to d\) any morphism. Then we have
        \[\tau_d\colon \cC(c,d) \to F(d)\]
        and
        \[\tau_d(f\colon c \to d) = \tau_d(\cC(c,f)(\id_c)) = F(f) (\tau_c(\id_c))\]
        where we use naturality of \(\tau\):
        \[\begin{tikzcd}
            {\cC(c,d)} \ar[r, "\tau_d"] \ar[d, "{\cC(c,g)}"] & F(d) \ar[d, "F(g)"] \\
            {\cC(c,e)} \ar[r, "\tau_e"] & F(e) \\
        \end{tikzcd}\]
        which implies the value of \(\tau\) at \(d,f\colon c\to d\) is determined by its value of \((c, \id_c)\) and the functorality of \(F\).
        \item[Surjectivity] Let \(y \in F(c)\) be given. For an object \(d\) of \(\cC\) and morphism \(f\colon c \to d\), we define
        \[\tau_d\colon \cC(c,d) \to F(d) \quad \tau_d(f) \coloneq F(f)(y).\]
        We check \(\tau_c(\id_c) = F(\id_c)(y) = y\).
        We need to check for naturality. Let \(g\colon d \to e \) be another morphism. Then
        \[\begin{split}
            F(g)(\tau_d(f)) &= F(g)(F(f)(y)) = F(g\circ f)(y) \\
            &= \tau_e(g\circ f) = \tau_e(\cC(c,g)(f)) \\
        \end{split}\]
        \end{description}
\end{proof}

Let \(\cC\) be a category, \(c\) an object of \(\cC\). We define the functor \(\IZ[\cC(c,\_)]\colon \cC \to (\mathbf{ab.grps.})\) as the composite
\[\cC \xrightarrow{\cC(c,\_)} (\mathbf{sets}) \xrightarrow{\IZ[\_]} (\mathbf{ab.grps.}).\]
In particular, \(\IZ[\cC(c, \_)] (d) = \IZ[\cC(c,d)]\).
\begin{Proposition}[Additive Yoneda lemma]
    Let \(c \in ob(\cC), F\colon \cC \to (\mathbf{ab.grps.})\) any functor. Then the evaluation map
    \[\Nat_{\cC \to (\mathbf{ab.grps.})}(\IZ[\cC(c,\_)], F) \to F(c)\]
    is bijective. \((\tau\colon \IZ[\cC(c, \_)] \to F) \mapsto \tau_c(1\cdot \id_c)\).
\end{Proposition}

\begin{proof}
    For varying objects \(d\) of \(\cC\), the bijections
    \[\Hom_{AB}(\IZ[\cC(c,d)], F(d)) \cong \Hom_{\mathbf{sets}}(\cC(c,d), F(d))\]
    assemble into a bijection\footnote{I don't know why though.}
    \[\Nat_{\cC \to \mathbf{Ab}}(\IZ[\cC(c, \_)], F) \cong \Nat_{\cC \to \mathbf{sets}}(\cC(c, \_), F) \overset{\text{Yoneda}}{\cong} F(c)\]
\end{proof}

\begin{defi}{Representable functor}{}
    A functor \(F\colon \cC \to \mathbf{Ab}\) is representable\index{representable functor} if there is an object \(c \in \cC\) and a natural isomorphism \(F\cong \IZ[\cC(c,\_)]\)
\end{defi}
\textbf{Note.} Any isomorphism \(F \cong \IZ[\cC(c,\_)]\) is determined by the \enquote{universal element} in \(F(c)\).

\begin{example}
    Let \(\cC = (\mathbf{ssets}) \times (\mathbf{ssets})\) be the product of two copies of the category of simplicial sets. Define \(f\colon (\mathbf{ssets})\times (\mathbf{ssets}) \to \mathbf{Ab}\) given by \(F(X,Y) = \IZ[X_p \times Y_q]\) for some \(p,q \geq 0\).
    \textbf{Claim.} This functor is representable by \((\Delta^p, \Delta^q)\) with natural isomorphisms.
    \[(\mathbf{ssets}\times \mathbf{ssets})((\Delta^p, \Delta^q), (X,Y)) = \mathbf{sets}(\Delta^p, X) \times \mathbf{sets}(\Delta^q,Y) \cong X_p \times Y_q\]
    Apply free abelian groups to get
    \[\IZ[(\mathbf{ssets} \times \mathbf{ssets})((\Delta^p,\Delta^q)(X,Y))] \cong \IZ[X_P \times Y_q]\]
\end{example}

\begin{notation}
    For \(F\colon \cC \to \mathbf{Chains}\) we write \(F_n = (\_)_n \circ F\colon \cC \to \mathbf{Ab}\) as the composite.
    \[\cC \xrightarrow{F} \mathbf{Chains} \xrightarrow{(\_)_n} \mathbf{Ab}\]
    and the second map sends \(C = C(n, d_n)_{n \in \IZ} \mapsto C_n\).
\end{notation}

\begin{thm}{Acyclic models}{}
    Let \(\cC\) be a category, \(F,G\colon \cC \to \mathbf{Chains}_+ = \text{non-negative grade chain complexes}\). Let \(\psi \colon F \to G\) be a natural transformation of functors. Suppose;
    \begin{enumerate}
        \item The transformation \(\psi_0 \colon F_0 \to G_0 \colon \cC \to \mathbf{Ab}\) is the zero natural transformation
        \item For every \(n \geq 1\), the functor \(F_n \colon \cC \to \mathbf{Ab}\) is isomorphic to a direct sum of representable functors, \(\bigoplus_{i \in I} IZ[\cC(c_i,\_)]\) for some family \(\set{c_i}_{i \in I}\) of \(\cC\)-objects such that \(H_n(G(c_i)) = 0\).
    \end{enumerate}
    Then \(\psi\) is naturally chain nullhomotopic.
\end{thm}


\newLecture{16.04.2025}

\begin{proof}
    For \(n \geq 0\), we will construct natural transformations
    \[s_n\colon F_n \to G_{n+1}\]
    of functors \(\cC \to \mathbf{Ab}\), such that
    \begin{equation}\tag{*}
        d_{n+1} \circ s_n + s_{n-1} \circ d_n = \psi_n
    \end{equation}
    as natural transformations (i.e. they have the chain homotopy property).

    The construction is by induction on \(n.\) We begin with \(s_0 = 0\) and \(s_{-1} = 0\). Suppose \(n \geq 1\) and that \(s_0, \dots, s_{n-1}\) have been constructed satisfying (*). Then
    \[d_n^G \circ (\psi_n - s_{n-1} \circ d_n^F) = d_n^G \circ \psi_n - d_n^G \circ s_{n-1} \circ d_n^F\]
    as \(\psi\) is a chain map,
    \[= \psi_{n-1} \circ d_n^F - d_n^G \circ s_{n-1} \circ d_n^F = (\psi_{n-1} - d_n^G \circ s_{n-1}) \circ d_n^F \overset{(*)}{=} s_{n-2} \circ d_{n-1}^F \circ d_n^F = 0.\]
    So \(\psi_n - s_{n-1} \circ d_n^F\colon F_n \to G_n\) takes values in cycles.
    By 2.,
    \[f_n = \bigoplus_{i \in I} \IZ[\cC(c, \_)]\]
    for some set \(\set{c_i}_{i \in I}\) of \(\cC\)-objects, such that \(H_n(G(c_i)) = 0\). Let \(j \in I\), write
    \[x_j \in F(c_j) = \bigoplus_{i \in I} \IZ[\cC(c_i,c_j)]\]
    be the element \(1 \cdot \id_j\) in the \(j\)-th summand. Then
    \[\psi_n^{c_j}(x_j) - s_{n-1}^{{c_j}}(d_n^{F, c_i}(x_j)) \in G_n(c_j)\]
    is a cycle. Since \(H_n(G(c_j)) = 0\), the class is a boundary in the complex \(G(c_j)\).

    Let \(y_j \in G(c_j)_{n+1}\) be a element such that
    \[d_{n+1}^{c_j}(y_j) = \psi_n^{c_j}(x_j) - s_{n-1}^{c_j}(d_n^{F, c_j}(x_j))\]
    The additive Yoneda lemma provides a unique natural transformation
    \[s_{n,j} \colon \IZ[\cC(c_j, \_)] \to G_{n+1}\]
    such that \(s_{n,j}(x_j) = s_{n,j}^{c_j}(1\cdot \id_{c_j}) = y_j \in G_{n+1}(c_j)\).

    We define the natural transformation
    \[s_n\colon F_n = \bigoplus_{i \in I} \IZ[\cC(c_i, \_)] \to G_{n+1}\]
    as \(s_n = \bigoplus_{j \in I} s_{n,j}\).

    It suffices now to show, that (*) holds on each summand \(\IZ[\cC(c_j, \_)]\). By the additive Yoneda lemma, there it suffices to check the relation on \(1\cdot \id_{c_j}\), which holds by definition.
\end{proof}

\begin{Remark}
    We only prooved \enquote{half} of the acyclic models theorem. The other half states:

    Let \(\cC\) and \(F,G\colon \cC \to \mathbf{Chains}_+\) be as before, satisfying 2.. Then any natural transformation \(\psi_0\colon F_0 \to G_0\) can be extended to a natural transformation \(\psi\colon F \to G\).
\end{Remark}

\begin{proof}[Proof of theorem \ref{thm:ElbZlb}]
    Now to actually proove the Eilenberg-Zilber-Theorem \ref{thm:ElbZlb} (at least in a special case.) Let \(A,B\) be simplicial abelian groups. We assume \(A = \IZ[X]\), \(B= \IZ[Y]\) for some simplicial sets \(X,Y\). We write \(C_*(X), C_*(Y)\). For sets \(S,T\),
    \[\begin{tikzcd}
        \IZ[S] \otimes \IZ[T] \ar[rr, bend left] && \IZ[S \times T] \ar[ll, bend left] \\
        s\otimes t \ar[rr] && (s,t)
    \end{tikzcd}\]
    is naturally isomorphic. Dimensionwise this gives \(\IZ[X] \otimes \IZ[Y] \cong \IZ[X \times Y]\).

    We want to move this further to \(C_*(X) \otimes C_*(Y) \cong C_*(X\times Y)\).

    \begin{proposition}\leavevmode
        \begin{enumerate}
            \item For all \(p\geq 0\), the simplicial set \(\Delta^q\) is simplicially contractible.
            \item For all \(p \geq 0\), the complex \(C_*(\Delta^p)\) is chain homotopy equivalent to the complex \(\IZ[0]\), the complex consisting of \(\IZ\) in dimension \(0\).
            \item For \(p,q \geq 0\), the chain complex \(C_*(\Delta^p) \otimes C_*(\Delta^q)\) is chain homotopy equivalent to \(\IZ[0]\). In particular,
            \[H_n(C_*(\Delta^p) \otimes C_*(\Delta^q)) = 0\]
            for \(n > 0\).
        \end{enumerate}
    \end{proposition}
    \begin{proof}\leavevmode
        \begin{enumerate}
            \item We define a morphism of simplicial sets \(H\colon \Delta^p \times \Delta^1 \to \Delta^p\) that contracts \(\Delta^p\) to the last vertex.\footnote{remember, that Homotopy is not symmetric in Simplicial sets. This is such an example.}
            In dimension \(n\),
            \[H_n\colon \Delta([n], [p]) \times \Delta([n],[1]) \to \Delta([n], [p])\]
            is given by
            \[H_n(\alpha, \beta)(i) = \begin{cases}
                \alpha(i) & \text{if } \beta(i) = 0 \\
                p & \text{if } \beta(i) = 1 \\
            \end{cases}\]
            for \(0 \leq i \leq n\). Let \(\gamma\colon [m] \to [n]\) be any morphism in \(\Delta\). Then
            \[H_m(\gamma^*(\alpha,\beta))(j) = H_m(\alpha\circ \gamma, \beta\circ \gamma)(j) = \begin{cases}
                \alpha(\gamma(j)) & \text{if } \beta(\gamma(j)) = 0 \\
                p & \text{if } \beta(\gamma(j)) = 1 \\
            \end{cases} = H_n(\alpha,\beta)(\gamma(j)) = \gamma^*(H_n(\alpha,\beta)(j))\]
            This means \(H\) is a homotopy from \(\Id_{\Delta^p}\) to the composite
            \[\Delta^p \to \Delta^0 \xrightarrow{p\text{-th vertex}} \Delta^p\]
            \item \(C_*\colon \mathbf{ssets} \to \mathbf{chains}\) takes simplicial homotopies to chain homotopies. So we know \(C_*(\Delta^p)\) is chain homotopy equivalent to
            \[C_*(\Delta^0) = (\dots \IZ \xrightarrow{\Id} \IZ \xrightarrow{0} \IZ)\]
            which is chain homotopy equivalent to
            \[(\dots 0 \to 0 \to 0 \to 0 \to \IZ) = \IZ[0]\]

            \item The tensor product of chain complexes preserves chain homotopy equivalences in each variable separatedly. So
            \[C_*(\Delta^p) \otimes C_*(\Delta^q) \sim \IZ[0] \otimes C_*(\Delta^1) \sim \IZ[0] \otimes \IZ[0] \cong \IZ[0].\]
        \end{enumerate}
    \end{proof}

    We now must produce natural chain homotopies from
    \[\mathrm{AW} \circ \mathrm{EZ} \colon C_*(X) \otimes C_*(Y) \to C_*(X) \otimes C_*(Y)\]
    and
    \[\mathrm{EZ} \circ \mathrm{AW}: C_*(X\times Y) \to C_*(X \times Y)\]
    to the respective identities.

    \textbf{Claim.} \(\mathbf{AW}\circ \mathbf{EZ} - \Id_{C_*(X) \otimes C_*(Y)}\colon C_*(X) \otimes C_*(Y) \to C_*(X) \otimes C_*(Y)\) satisfies the hypothesis of acyclic models.

    \begin{proof}
        \[\begin{tikzcd}
            C_0(X) \otimes C_0(Y) \ar[d, phantom, "=", sloped]\ar[r,"\cong", phantom] & \IZ[X_0] \otimes \IZ[Y_0] \ar[r, bend left, "\cong"] & \IZ[X_0 \times Y_0] \ar[l, bend left, "\cong"] \ar[d, "=", phantom, sloped]\\
            (C_*(X) \otimes C_*(Y))_0 && C_0(X\times Y)\\
        \end{tikzcd}\]
        Which means \((\mathbf{AW} \circ \mathbf{EZ})_0 = \Id\) and \((\mathbf{EZ} \circ \mathbf{AW})_0 = \Id\). which means \(\psi_0 = \) zero natural transformation.

        \[(C_*(X) \otimes C_*(Y))_n = \bigoplus_{p+q = n} C_p(X) \otimes C_q(Y) = \bigoplus_{p+q = n} \IZ[X_p] \otimes \IZ[Y_q] \cong \bigoplus_{p+q = n} \IZ[X_p \times Y_q]\]
        which is represented by \((\Delta^p, \Delta^q)\). Then
        \(H_n(C_*(\Delta^p \otimes \Delta^q)) = 0\) (I think, he erased before I could copy.)

        We consider \(\phi \colon \mathbf{EZ} \circ \mathbf{AW} - \Id_{C_*(X\times Y)} \colon C_*(X\times Y) \to C_*(X\times Y)\). We know, \(\phi_0 = 0\). We need to show, that \(\phi\) satisfies the hypothesis of acyclic models.
        \[C_n(X\times Y) = \IZ[X_n \times Y_n]\]
        is representable by \((\Delta^n, \Delta^n)\).
        \[H_n(C_*(\Delta^n\times \Delta^n)) \cong H_n(\Delta^0 \times \Delta^0) = H_n(\Delta^0) = 0\]
        for \(n > 0\), where we used \(\Delta^n \sim \Delta^0\) and so \(\Delta^n \times \Delta^n \sim \Delta^0 \times \Delta^0\). So acyclic models produces a natural chain nullhomotopy of \(\phi\).
    \end{proof}

    This concludes the proof of the Künneth theorem.
\end{proof}

\section{Revisiting Commutativity of the cup-product}

The symmetry isomorphism of chain complexes \(C,D\) is the morphism.
\[\tau_{C,D} \colon C\otimes D \xrightarrow{\cong} D\otimes C\]
is given by
\[\begin{tikzcd}
    {\tau_{C,D}}_n & \colon (C\otimes D)_n & (D\otimes C)_n \\
    & \bigoplus_{p+q = n} C_p \otimes D_q & \bigoplus_{q+p= n} D_q \otimes C_p \\
    & c\otimes d & (-1)^{pq} \cdot d\otimes c
\end{tikzcd}\]
me not being able to keep up.

\textbf{Fact.}
\[\begin{tikzcd}
    C_*(X) \otimes C_*(Y) \ar[d, "\tau"] \ar[r, "\mathbf{EZ}"] & C_*(X,Y) \ar[d, "C_*(flip)"] \\
    C_*(Y) \otimes C_*(X) \ar[r, "\mathbf{EZ}"] & C_*(Y\otimes X) \\
\end{tikzcd}\]
commutes. where \(flip \colon X \times Y \to Y \times X, \; (x,y) \mapsto (y,x)\).
Hence, \enquote{The Eilenberg-Zilber map is symmetric}.

But however for AW the same diagram does NOT commute. Another diagram is missing.

%%Diagram

Howeveer it does so up to natural chain homotopy by applying the acyclic models to the differenc of the two composites. He explains, why we can apply acyclic models.

Let \(X\) be a simplicial set. The diagonal \(\Delta\colon X \to X \times X\) is flip-invariant, i.e.
\[\begin{tikzcd}
    X \ar[r, "\Delta"]\ar[rd, "\Delta"] & X\times X \ar[d, "flip"] \\
    & X \times X
\end{tikzcd}\]
We draw a diagram:
\[\begin{tikzcd}
    C_*(X) \ar[r, "C_*(\Delta)"] \ar[rd, "C_*(\Delta)"] & C_*(X\times X) \ar[d, "C_*(flip)"]\ar[r, "\mathbf{AW}"] &C_*(X) \otimes C_*(X) \ar[d, "\tau"]\\
    & C_*(X\times X) \ar[r, "\mathbf{AW}"]& C_*(X) \otimes C_*(X) \\
\end{tikzcd}\]
that commutes up to homotopy. We apply the functor \(\Hom(\_, R)\) to get a new diagram and my speed at copying was not capable of keeping up. You may want to have a look at the videos for this.


\newLecture{23.04.2025}


The Plan for today is to show the Künneth theorem for homology. The rough approximation is, that product of spaces goes to Tensorproducts of abelian groups.

\section{Algebraic Künneth theorem.}

If \(X,Y\) are simplicial sets, then by EZ we have \(H_*(X\times Y; R) = H_*(C_*(X\times Y; R)) \cong H_*((C_*(X,R)) \otimes_R C_*(Y; R))\) and we want to see how that relates to \(H_*(X,R) \otimes_R H_*(Y; R)\). This will be the algebraic Künneth theorem.

In the following \(R\) is a commutative ring (have integers and fields in mind).

\begin{defi}{Tensor Product of \(R\)-chains}{}
Let \(C,D\) be chain complexes of \(R\)-modules. We define a new complex of \(R\)-modules \(C \otimes_R D\):
\[(C\otimes_R D)_n = \bigoplus_{p+q = n} C_p \otimes_R D_q\]
with differential
\[d(x\otimes y) = dx \otimes y + (-1)^{pq} x\otimes dy.\]
\end{defi}

Note that \(R \otimes \IZ[S] \cong R[S]\) for \(S\) a simplicial set. And \(R[S] \otimes_R R [T] \cong R[S\times T]\) for \(S,T\) simplicial sets.

For \(X,Y\) simplicial sets, we have
\[R\otimes C_*(X,\IZ) \otimes C_*(Y,\IZ) \xrightarrow{R\otimes \mathbf{EZ}} R\otimes C_*(X\times Y; \IZ) \cong C_*(X\otimes Y; R)\]
and for \(R\otimes C_*(X; \IZ) \otimes C_*(Y; \IZ) \cong (R\otimes C_*(X;Z)) \otimes_R (R \otimes C_*(Y; \IZ)) = C_*(X,R) \otimes_R C_*(Y,R)\), so we get a Eilenberg-Zilber map
\[C_*(X,R)\otimes_R C_*(Y,R) \xrightarrow{\mathbf{EZ}} C_*(X\times Y; R)\]

\textbf{Aim.} relate \(H_*(C\otimes_R D)\) to \(H_*(C), H_*(D)\). Our hope is to have a map
\[\bigoplus_{p+q = n} H_p(C) \otimes_R H_q(D) \xrightarrow{???} H_n(C\otimes_R D)\]

For example taking \(R = \IZ\) and \(C = D = (\dots ,\ra 0\ra \IZ \xrightarrow{\cdot 2}\IZ \ra 0)\). Then
\[H_n(C) = H_n(D) = \begin{cases}
    \IZ/2 & n = 0 \\
    0 & n \neq 0 \\
\end{cases}\]
but \(C \otimes D = (0\ra 0 \ra \IZ \ra \IZ^2 \ra \IZ \ra 0)\). And 
\[H_1(C\otimes D) = \set{(x,-x) \in \IZ}/\set{(2y, -2y) \mid y\in \IZ} \cong \IZ/2 \neq 0\]

\begin{defi}{Projective \(R\)-modules}{}
    An \(R\)-module \(P\) is \emph{projective}\index{Projective Module} if for every epimorphism \(\varepsilon \colon M\to N\) of \(R\)-modules, the map
    \[\Hom(P, \varepsilon) \colon \Hom(P, M) \to \Hom(P, N)\]
    is surjective.
    \[\begin{tikzcd}
        & M \ar[d, "\varepsilon", two heads] \\
        P \ar[ru, dotted] \ar[r, "f"] & N \\
    \end{tikzcd}\]
\end{defi}

\textbf{Fact.} \(P\) is projective iff \(P\) is a direct summand of a free module iff there exists a \(R\)-module \(Q\) and a set \(S\), such that
\[P \oplus Q \cong R[S].\]

\begin{proof}
    Free modulse are projective:
    \[\begin{tikzcd}
        & M \ar[d, "\varepsilon", two heads] \\
        R[S] \ar[r, "f"] \ar[ru, dotted, "g"] & N \\
    \end{tikzcd}\]
    for every \(s \in S\) choose \(m_s \in M\) \(\varepsilon(m_s) = f(s)\). Then there is a unique homomorphism \(g\colon R[S] \to M\) such that \(g(s) = m_s\).
    
    Let \(P\) be projective and \(Q\) a summand of \(P\). For reasons I couldn't copy, then \(Q\) is also projective.

Let \(P\) be a projective \(R\)-modulo. Consider the epimorphism
\[\begin{split}
    R[P ] &\to P \\
    p &\mapsto p
\end{split}\]

Then we have
\[\begin{tikzcd}
    & R[P] \ar[d, two heads] \\
    p \ar[ru, dotted, "g"] \ar[r, "\id"] & P \\
\end{tikzcd}\]
So \(P\) is a direct summand of \(R[P]\).

\end{proof}

\begin{itemize}
    \item If \(R\) is a field, then all modules are free, hence projective.
    \item \(R = \IZ/6\), \(P = \IZ/2, Q = \IZ/3\). Then \(\IZ/6 \cong \IZ/2\oplus \IZ/3\), so, as \(\IZ/6\) is free, \(\IZ/2\) and \( \IZ/3\) are projective, but not free.
\end{itemize}

\begin{proposition}
    Let \(R\) be a commutative ring, and
    \[0 \ra I \xrightarrow{\alpha} M \xrightarrow{\beta} N \to 0\]
    be a short exact sequence of \(R\)-modules.

    Then for every \(R\)-module \(P\), the sequence
    \[P\otimes_R I \xrightarrow{P\otimes_R \alpha} P \otimes_R M \xrightarrow{p\otimes_R \beta} P\otimes_R N \to 0\]
    is exact. (\enquote{\(P \otimes_R \_\) is right exact}). If moreover \(P\) is projective, then it is also exact with a \(0\) on the left, i.e. \(P\otimes_R \alpha\) is injective. (\enquote{projective modules are flat}).
\end{proposition}

\begin{proof}
    \[p\otimes_R \beta) \circ (p\otimes_R) \alpha = P\otimes_r (\beta\circ \alpha) = P\otimes_R 0 = 0\]
    so \(\Img(P\otimes_R \alpha) \subseteq \ker(P\otimes_R \beta)\) so we get an induced homomorphism
    \[\gamma \frac{P\otimes_R M}{\Img(P\otimes_R \alpha)} \to P\otimes_R N\]
    exactness is equivalent to \(\delta\) being an isomorphism.
    We define a homomorphism \(\delta\colon P\otimes_R N \to \frac{P\otimes_R M}{\Img(P\otimes_R \alpha)}\)given by \((p,n) \in P\otimes N\) choose \(\tilde n \in M\), such that \(\beta(\tilde n) = n\).

    \textbf{Claim.} \(\delta(p\otimes n) = p\otimes \tilde n + \Img(P\otimes_R \alpha)\) is independent of choice of \(\tilde n\)
    \begin{proof}
        Let \(\tilde\tilde n \in M\) also satisfy \(\beta(\tilde{\tilde n}) = n\). Then \(\beta(\tilde{\tilde n} - \tilde n) = 0\), so there is \(i \in I\) s.t. \(\alpha(i) = \tilde{\tilde n} - \tilde n\). \(p \otimes \tilde{\tilde n} - p\otimes \tilde n = p\otimes (\tilde{\tilde n} - \tilde n) = p\otimes \alpha(i) \in \Img(P\otimes_R \alpha)\).
    \end{proof}

    \textbf{Claim.} The assignment of \(\delta\) is biadditive and sends \((rp,n)\) and \((p, rn)\) to the same element.

    Then this extends to a well defined \(R\)-linear map
    \[P\otimes_R N \to \frac{P\otimes_R M}{\img(P\otimes_R \alpha)}\]
    which is isomorphic.

    Now let \(P\) be projective. We show that then \(P \otimes_R \alpha\) is injective.

    \begin{description}
        \item[Case 1] \(P = R[S]\) free, \(S\) some set. Then
        \[P \otimes_R M = R[S] \otimes_R M \cong \bigoplus{s \in S} M\]
        we have a natural isomorphism of \(R\)-modules in \(M\).

        From this we get a commutative square of \(R\)-modules:
        \[\begin{tikzcd}
            P\otimes_R I \ar[d, "\cong", sloped, phantom] \ar[r, "P\otimes_R \alpha"] & P\otimes_R M \ar[d, "\cong", phantom, sloped] \\
            \bigoplus_{s\in S} I \ar[r, "\bigoplus_{s \in S} \alpha"] & \bigoplus_{s \in S} M \\
        \end{tikzcd}\]
        where the bottom map is injective.
        \item[General case] \(P\) projective is a summand of a free module \(F\), i.e. there are homomorphisms
        \[P \xrightarrow{\lambda} F \xrightarrow{\mu} P\]
        s.t. \(\mu\circ \lambda = \Id_P\). We consider thee commutative square
        \[\begin{tikzcd}
            P\otimes_R I \ar[d, "\lambda\otimes_R I"]\ar[r, "P\otimes_R \alpha"] & P\otimes_R N \ar[d, "\lambda\otimes_R N"]\\
            F \otimes_R I \ar["F \otimes_R \alpha", r] & F\otimes_R N \\
        \end{tikzcd}\]
        where the bottom map is injective by Case 1 and \(\lambda \otimes_R I\) is injective, as it admits a retraction.
    \end{description}
\end{proof}

\begin{defi}{Global dimension of rings}{}
    A commutative ring \(R\)\index{Global dimension} has global dimension \(\leq 1\) if every submodule of a projective module is projective.
\end{defi}

\begin{example} Some rings with global dimension \(\leq 1\) are
    \begin{itemize}
        \item fields 
        \item the ring of integers \(\IZ\) (subgroups of free abelian groups are free).
        \item every PID\footnote{no zero divisors and every ideal is generated by a single element.} is of this form. See for example \(k[x]\) for \(k\) a field or \(\IZ[i]\) the gaussian integers
        \item \(\IZ_p\) the \(p\)-adic integers.
    \end{itemize}
\end{example}

\begin{defi}{Tor of nice rings}{}
    Let \(R\) be a commutative ring of global dimension \(\leq 1\). Let \(M,N\) be \(R\)-modules. Choose an epimorphism \(p\colon P \to N\) of \(R\)-modules with \(P\) projective. Define
    \[\Tor^R(M,N) = \Ker(M\otimes_R N \xrightarrow{M\otimes_R incl} M\otimes_R P)\]
\end{defi}

\textbf{Facts.} This is independent up to preferred isomorphism of the choice of \(p\colon P \to N\).

It is symmetric, i.e. we can resolve \(M\) instead of \(N\).

If \(P\) is projective, then \(\Tor^R(P,N) = 0 = \Tor^R(M,P)\).

\begin{construction}
    For \(R\) a commutative ring, \(C,D\) complexes of \(R\)-modules. We define a natural homomorphism
    \[\Phi \colon H_p(C) \otimes_R H_q(D) \to H_{p+q}(C\otimes_R D)\]
    via \([x] \otimes [y] \mapsto [x\otimes y]\)
\end{construction}

We can check this is well defined.

\begin{thm}{Algebraic Künneth theorem}{}
    Let \(R\) be a commutative ring of global dimension \(\leq 1\). Let \(C, D\) be complexes of projective \(R\)-modules. Then the following map is \(R\)-linearly split injective
    \[\bigoplus \Phi \colon \bigoplus_{p+q = n} H_p(C) \otimes_R H_q(D) \to H_n(C \otimes_R D)\]
    Moreover the cokernel is naturally isomorphic to
    \[\bigoplus_{p+q = n-1} \Tor^R(H_p(C), H_q(D)).\]

    Equivalently, there is a natural and split short exact sequence
    \[0 \to \bigoplus_{p+q = n} H_p(C) \otimes_R H_q(D) \xrightarrow{\Phi} H_n(C\otimes D) \to \bigoplus_{p+q = n-1} \Tor^R(H_p(C), H_q(D)) \to 0\]
\end{thm}

\begin{proof}
    We let \(Z = \set{Z_q}_{q\in \IZ}\) be the comples of \(R\) modules with \(d = 0\) where \(Z_q = \Ker(d\colon D_q \to D_{q-1})\), let \(B = \set{B_q}\) be the complex with \(d = 0\) where \(B_q = \Img(D\colon D_{q+1} \to D_q)\). We have a short exact sequence of complexes of \(R\)-modules
    \[0 \to Z \xrightarrow{incl} D \xrightarrow{d} B[1] \to 0\]
    where \(B[1]\) is the complex \(B\) shilfted up by 1.

    We have \(B_q \subseteq Z_q \subseteq D_q\) projective by hypothesis. Since \(R\) has global dimension \(\leq 1\), \(B_q\) and \(Z_q\) are also projective.
    \[0 \to Z_q \to D_q \xrightarrow{d} B_{q-1} \to 0\]
    is hort exact, \(B_{q-1}\) is projective, so the sequence splits.

    For every \(R\)-module \(N\), the sequence
    \[0 \to N\otimes_R Z_p \to N\otimes_R D_q \to N\otimes_R B_{q-1} \to 0\]
    is exact.

    This means we get a short exact sequence of complexes
    \[0 \to C\otimes_R Z \to C\otimes_R D \to C\otimes_R B[1] \to 0\]

    This means we get a long exact homology sequence
    \[\to H_n(C\otimes_R Z) \xrightarrow{H_n(C\otimes_R incl)} H_n(C\otimes D) \xrightarrow{H_n(C\otimes d)} H_{n-1}(C\otimes_R B) \xrightarrow{\partial} H_{n-1}(C\otimes_R Z) \to \dots\]
    Since \(Z\) has trivial differential:
    \[H_n(C\otimes_R Z) = H_n(\bigoplus_{q \in \IZ} C[q] \otimes Z_q) \cong \bigoplus_{q \in \IZ} H_n(C[q]\otimes Z_q) \cong \bigoplus_{q \in \IZ} H_n(C[q]) \otimes_R Z_q = \bigoplus_{p \in \IZ} H_{n-q}(C) \otimes_R Z_q\]
    where we use that \(Z_q\) is projective.

    Similarly \(H_n(C\otimes_R B) \cong \bigoplus_{q\in \IZ} H_{n-q}(C) \otimes B_q\).

    This gives us a long exact sequence
    \[\dots \to \bigoplus_{p+q = n} H_p(C) \otimes_R Z_q \to H_n(C\otimes_R D) \to \bigoplus_{p+q = n-1} H_p(C) \otimes B_q \to \bigoplus_{p+q = n-1} H_p(C) \otimes Z_q\]
    This splits up into short exact sequences
    \[0 \to \bigoplus_{p+q = n} \Coker(H_p(C) \otimes B_q \to H_p(C) \otimes_R Z_p) \to H_n(C\otimes_R D) \to \bigoplus_{p+q =n-1} \Ker(H_p(C) \otimes_R B_q \to H_p(C) \otimes Z_q) \to 0\]
    We know \(0 \to B_q \to Z_q \to H_q(D)\) is a projective resolution of \(H_q(D)\).

    This means for all \(R\)-modules \(N\),
    \[\Tor^R(N; H_q(D)) = \Ker(N\otimes_R B_q \to N\otimes_R Z_q)\]
    \[N\otimes_R H_q(D) \cong \Coker(N\otimes_R B_q \to N\otimes_R Z_q)\]
    So we get:
    \[0 \to \bigoplus_{p+q = n} H_p(C) \otimes H_q(D) \xrightarrow{\Phi} H_n(C\otimes_R D) \to \bigoplus_{p+q = n-1} \Tor^R(H_p(C), H_q(D)) \to 0\]

    for next lecture remains, that \(\Phi\) has a \(R\)-linear retraction!

\newLecture{28.04.2025}

    For the \(R\)-linear spitting.

    Because \(B_q\) is projective, the following s.e.s. splits:
    \[0 \to Z_q \xrightarrow{\incl} D_q \xrightarrow{d}  B_q \to 0\]
    and the map \(Z_q\) to \(D_q\) admits a retraction. We choose a retraction \(r_q\colon D_q \to Z_q\) to the inclusion.

    Then
    \[\begin{tikzcd}
        D_q+1 \ar[d, "d"] \ar[rrdd, bend left, "0"] & & \\
        B_q  \ar[d, phantom, sloped, "\subseteq "] \ar[rrd, "0"] \ar[dr, "="] & & \\
        D_q \ar[r, "r_q"] & Z_q  \ar[r] & H_q(D) \\
    \end{tikzcd}\]
    the retraction \(\set{r_q}_{q \in \IZ}\) for a morphism of chain complexes
    \[r\colon D \to \set{H_q(D), d = 0}_q\]
    that induces the identity on homology.

    \(H_q(r) \cong H_q(D) \to H_q(H_*(D), d= 0) = H_q(D)\).
    Similalry, there is a chain map \(\rho\colon C \to \set{H_p(C), d = 0}\) that is the identity on homology. This gives a chain mpa \(\rho \otimes_R r\colon C \otimes_R D \to (H_*(C) \otimes_R H_*(D), d= 0)\) which on homology
    \[H_n(\rho\otimes_r r) \colon H_n(C \otimes_R D) \to H_n(H_*(C) \otimes RH_*(D), d= 0) = \bigoplus_{p+q = n} H_n(C) \otimes_R H_n(D)\]
    which is a retraction to
    \[\Psi \colon \bigoplus_{p+q = n} H_p(C) \otimes_R H_q(D) \to H_n(C\otimes R D)\]

\end{proof}

\begin{example}
    Let \(R\) be a field. Then every module is free, hence projective, and
    \[\Tor^R(M,N) = 0\]
    for all \(R\)-modules M,N. For all complexes of \(R\)-modules C,D;
    \[\psi\colon \bigoplus_{p+q = n} H_p(C) \otimes_R H_q(D) \xrightarrow{\cong} H_n(C\otimes_R D).\]
    is an isomorphism.

    If \(R = \IZ\). Let \(C,D\) be a complex of free abelian groups. Then there is a split s.e.s.
    \[0 \to \bigoplus_{p+q = n} H_p(C) \otimes H_q(D) \to H_n(C\otimes D) \to \bigoplus_{p+q = n-1} \Tor(H_p(C), H_q(D)) \to 0\]
\end{example}

\section{Homological Künneth theorem}

\begin{construction}[Homology exterior pairing]
    Let \(X,Y\) be simplicial sets. Let \(R\) be a commutative ring. We define
    \[\times \colon H_p(X,R) \otimes_R H_q(Y,R) \to H_{p+q} (X\times Y, R)\]
    as the composite
    \[H_p(C_*(X,R)) \otimes_R H_q(C_*(Y; R)) \xrightarrow{\Phi} H_{p+q}(C_*(X, R) \otimes C_*(Y,R)) \xrightarrow{H_{p+q}(\mathrm{EZ})} H_{p+q}(C_*(X\times Y, R))\]
    
    For topological spaces \(A,B\) we Define
    \[\times \colon H_p(A; R) \otimes _R H_q(B,R) \to H_{p+q}(A\times B, R)\]
    as the composite
    \[H_p(\cS(A),R) \otimes_R H_q(\cS(B),R) \xrightarrow{\times} H_{p+q}(\cS(A) \otimes \cS(B), R) \cong H_{p+q}(\cS(A\times B); R)\]
    where the isomorphism is given by the fact, that simplical complex commutes with products. The isomorphism is the canonical map
    \[\cS(A) \times \cS(B) \xleftarrow{(\cS(p_A), \cS(p_B))} \cS(A\times B)\]
\end{construction}

\begin{thm}{Künneth theorem for homology with field coefficients}{}
    Let \(R\) be a field. Let \(X,Y\) be simplicial sets or spaces. Then the homology external product
    \[\times \colon \bigoplus_{p+q = n} H_p(X,R) \otimes_R H_q(X,R) \to H_n(X\times Y; R)\]
    is an isomorphism.
\end{thm}

\begin{proof}
    Follows directly from algebraic Künneth + Eilenberg-Zilber
\end{proof}

\begin{thm}{Künneth theorem for homology}{}
    Let \(X,Y\) be spaces or simplicial sets. Then there is a natural and split s.e.s. 
    \[0 \to \bigoplus_{p+q = n} H_p(X, \IZ) \otimes H_q(Y, \IZ) \to H_n(X\times Y; \IZ) \to \bigoplus_{p+q = n-1} \Tor(H_p(X,\IZ) , H_q(Y, \IZ)) \to 0\]
\end{thm}

\textbf{Special Case.} Let \(X, Y\) be spaces or simplical sets. Suppose that \(H_n(X, \IZ)\) is free for all \(n \geq 0\). Then
\[\bigoplus_{p+q = n} H_p(X, \IZ) \otimes H_q(Y, \IZ) \xrightarrow{\Phi} H_n(X\times Y; \IZ)\]
is an isomorphism.

\section{Künneth theorem for cohomology}

Next we want to show the Künneth theorem for cohomology. The strategy:
\begin{itemize}
    \item EZ provides a chain homotopy equilvalence \(C_*(X, \IZ) \otimes C_*(Y, \IZ)\) to \(C_*(X\times Y, \IZ)\).
    \item \(\Hom(\_, R) \colon \mathbf{Chains} \to \mathbf{coChains}_R\) preserves chain homotopies, so
    \[\Hom(C_*(X, \IZ), R) \otimes \Hom(C_*(Y, \IZ), R) \cong \Hom((C_*(X\times Y), \IZ), R)\]
    \item in favorable cases we can relate
    \[H^*(\Hom(C, R) \otimes_R \Hom(D, R)) \text{ to } H^*(\Hom(C,R)) \otimes_R H^*(\Hom(D,R))\]
    \item apply the algebraic Künneth theorem.
\end{itemize}

Step 3 is the hard step. For that we study relations between \(\Hom\) and tensors.

Let \(A\) be an abelian group and \(R\) an commutative ring. We make the set \(\Hom(A,R)\) of group homomorphisms into an \(R\) module by pointwise addition and skalar multiplication. So \(f,g \in \Hom(A,R)\), \(r \in R\). then
\[(f+g)(a) = f(a) + g(a), \quad ((r\cdot f)(a) = r\cdot f(a))\]
Let \(B\) be another abelian group. Then
\[\bullet \colon \Hom(A,R) \times \Hom(B,R) \to \Hom(A\otimes B, R)\]
by \((f\bullet g)(a\otimes b) = f(a) \cdot g(b)\). This is additive in \(f\) and \(g\).
\[(f+f') \bullet g = (f\bullet g) + (f' \bullet g)\]
and
\[(rf) \bullet g = r \cdot (f\bullet g) = f \bullet (r\cdot g)\]
for all \(r \in R\).
This means this extends to a well-defined \(R\)-linear map
\[\Hom(A,R) \otimes_R \Hom(B,R) \to \Hom(A\otimes B, R)\]
\begin{proposition}
    Let \(A,B\) be abelian groups and \(R\) a commutative ring. If \(A\) is finitely generated and free, then
    \[\Hom(A, R) \otimes_R \Hom(B,R) \to \Hom(A\otimes B, R)\]
    is an isomorphism of \(R\)-modules.
\end{proposition}

\begin{proof}
    For \(A = \IZ\):
    \[\begin{tikzcd}
        {\Hom(\IZ,R) \otimes_R \Hom(B,R)} \ar[r, "\bullet"] \ar[d, "{\mathrm{ev} \otimes_R \Hom(B,R)}"] & {\Hom(\IZ\otimes B, R)} \ar[d, "{\cong \; \Hom(k,R)}"]\\
        {R \otimes_R \Hom(B,R)} \ar[r, "\cong", "r\otimes g \mapsto r\cdot g"'] & {\Hom(B,R)}
    \end{tikzcd}\]
    where we have \(k \colon B \to \IZ \otimes B\) with \(b\mapsto 1 \otimes b\).

    Suppose the claim holds for \(A\) and \(A'\). Then it holds for \(A \oplus A'\).
    \[\begin{tikzcd}
        {\Hom(A\oplus A*, R) \otimes_R \Hom(B,R)} \ar[r, "\bullet"] \ar[d, phantom, sloped, "\cong"]& {\Hom((A\oplus A') \otimes B, R)} \ar[d, phantom, sloped, "\cong"] \\
        {(\Hom(A,R) \oplus \Hom(A',R)) \otimes_R \Hom(B,R)} \ar[d, phantom, sloped, "\cong"] & {\Hom((A\otimes B) \oplus (A' \otimes B), R)} \ar[d, phantom, sloped, "\cong"]\\
        {(\Hom(A,R) \otimes_R \Hom(B,R)) \bigoplus (\Hom(A',R) \otimes_R \Hom(B,R))} \ar[r, "\cong \; \text{by assumption}"]&  {\Hom(A\otimes B, R) \oplus \Hom(A' \otimes B, R)} &\\
    \end{tikzcd}\]

    The claim holfs for \(A = \IZ^k\), \(k \in \IN\). any finitely generated free abelian group is isomorphic to \(\IZ^k\).
\end{proof}

\begin{example}
    \begin{description}
        \item[\(R = \IF_2\)] \(A = B= \IZ[\IN]\). Then \(\Hom(\IZ[\IN], R) \cong \mathrm{maps}(\IN, R)\) by evaluation of generators. This is \(R\)-linear by the \(R\)-module structure on \(\mathrm{maps}(\IN, R)\).
        \[\begin{tikzcd}
            \Hom(A,R) \otimes \Hom(B,R) \ar[r, "\bullet"] & \Hom(A\otimes B, R) \\
            \mathrm{maps}(\IN,R) \otimes_R \mathrm{maps}(\IN, R) & \Hom(\IZ[\IN \times \IN], R) \\
            & \mathrm{maps}(\IN \times \IN, R)
        \end{tikzcd}\]
        This is however not an isomorphism.

        \item[] \(A = B = \IZ/2\) and \(R = \IZ/4\). Then \(\Hom(A,R) = \Hom(B,R) = \Hom(\IZ/2, \IZ/4)\) is cyclif of order two generate by \(i \colon \IZ/2 \to \IZ/4\), \(n +2\IZ \mapsto 2n + 4\IZ\).
        \[\begin{tikzcd}
            \Hom(A,R) \otimes_R \Hom(B,R) \ar[r, "\bullet"] \ar[d, phantom, sloped, "="] & \Hom(A\otimes B, R) \\
            \Hom(\IZ/2, \IZ/4) \otimes_{\IZ/4} \Hom(\IZ/2, \IZ/4) \ar[d, phantom, sloped, "\cong"] & \Hom(\IZ/2 \otimes \IZ/2, \IZ/4) \ar[d, phantom, sloped, "\cong"]\\
            \IZ/2 \ar[r, "0"] & \IZ/2 \\
        \end{tikzcd}\]
    \end{description}
\end{example}

This shows, that both assumptions are strictly necessary.

Now let \(C,D\) be complexes of abelian groups. Then \(\Hom(C,R), \Hom(D,R)\) are cochian complexes of \(R\)-modules.
\[\Hom(C,R)^n = \Hom(C_n,R)\]
and
\[d^n \colon \Hom(C,R)^n \to \Hom(C,R^{n+1}) = \Hom(D_{n+1}, R)\]
The sum of the \(\bigoplus\) homomorphism gives a cochain map
\[\bigoplus \colon \Hom(C,R) \otimes_R \Hom(D,R) \to \Hom(C\otimes D, R)\]
which is in dimension \(n\):
\[\bigoplus_{p+q = n} \Hom(C_p,R) \otimes_R \Hom(D_q, R) \xrightarrow{\text{sum of } \bigoplus} \Hom(\bigoplus_{p+q = n} C_p \otimes D_q, R)\]

\begin{proposition}
    Let \(C\) and \(D\) be chain complexes of abelian groups, such that \(C_n = 0 = D_n\) for \(n < 0\) and that \(C_n\) is finitely generated and free for all \(n \geq 0\). Then \(\bigoplus\) is an isomorphism.
    \[\bigoplus\colon \Hom(C,R) \otimes \Hom(D,R) \to \Hom(C\otimes D, R)\]
    is an isomorphism of cochain complexes.
\end{proposition}

\begin{proof}
    The vanishing hypothesis makes the potentially infinte sums
    \[\bigoplus_{p+q = n} \Hom(C_p, R) \otimes_R \Hom(D_q, R)\]
    finite.

    Then \(\Hom(\_, R)\) preserves sums. And
    \[\Hom(C_p, R) \otimes \Hom(D_q,R) \xrightarrow{\bigoplus} \Hom(C_p \otimes D_q; R)\]
    is an isomorphism by the previous proposition.
\end{proof}

This is not yet good enough to apply to topological spaces, as they are very not finitely generated.

\begin{proposition}
    Let \(C\) be a chain complex of free abelian groups, such that  \(C_n = 0\) for \(n < 0\). Suppose that \(H_n(C)\) is finitely generated for all \(n > 0\).

    Then there is a subcomplex \(B\) of \(C\), such that
    \begin{itemize}
        \item \(B_n\) is finitely generated and free for all \(n \geq 0\).
        \item The inclusion \(B \to C\) is a chain homotopy equivalence.
    \end{itemize}
\end{proposition}

\begin{proof}
    We construct subgroups \(B_n\) of \(C_n\) by induction on \(n \geq 0\), such that
    \begin{itemize}
        \item     \(d(B_n) \subseteq B_{n-1}\)
        \item the inclusions of \(0 \to B_n \xrightarrow{d} \to B_{n-1} \xrightarrow{d} \dots \to B_0 \to 0\)
        \item into \(C\) induces an isomorphism on \(H_i\) for all \(0 \leq i \leq n-1\) and an epimorphism on \(H_n\).
    \end{itemize} 

    Induction start: Let \(x_1, \dots, x_m\) be elements of \(C_0\), that generate \(H_0(C)\). Select \(B_0\) to be the subgroups of \(C_0\) generated by \(x_1, \dots, x_m\).

    Induction step: Suppose \(B_0, \dots, B_{n-1}\) have been constructed fullfilling the conditions.
    Let \(x_1, \dots, x_m\) be cycles in \(C_n\) whose homology classes generate \(H_n(C)\), which is possible because \(H_n(C)\) is finitely generated. Set
    \[Z = \Ker(d\colon B_{n-1} \to B_{n-2}) \cap \img(d\colon C_n \to C_{n-1})\]
    which is finitely generated because \(B_{n-1}\) is. Let \(z_1, \dots, z_k\) generate this intersection. Choose \(y_1, \dots, y_k \in C_n\), such that \(d(y_i) = z_i\) for \(1 \leq i \leq k\).

    Let \(B_n\) be the subgroup generated by \(x_1, \dots, x_m, y_1, \dots, y_k\). Then \(d(B_n) \subseteq B_{n-1}\).

    Let \(B_{\leq n}\) and \(B_{< n}\) be the subcomplexes of \(C\) generated by \(B_0, \dots, B_n\) and \(B_0, \dots, B_{n-1}\)

    Then \(B_{<n} \subseteq B_{\leq n} \subseteq C\) where \(B_{ < n}\) induces isomorphism on \(H_i\) for \(0 \leq i \leq n-2\) and epi on \(H_{n-1}\). Similarly \(B_{<n} \to B_{\leq n}\) is iso in dimension \(\leq n-1\).

    Then \(B_{\leq n}\) is an Isomorphism on \(H_i\) for \(0 \leq i \leq n-2\) and surjective on \(H_n\) because we include \(x_1, \dots, x_m\) that generate \(H_n(C)\).

    Let \(x \in B_{n-1}\) be any cycle whose class is in the kernel of \(H_{n-1}(B_{<n}) \to H_{n-1}(C)\). Then \(x \in Z\) so \(x\) is a linear combination of the classes \(z_1, \dots, z_k\) and hence a boundary of a linear combination of \(y_1, \dots, y_k\). So \(x = d(w)\) for some \(w \in B_n\). Then 
    \[\begin{tikzcd}
        &H_{n-1}(B_n) \ar[rd] & \\
        H_{n-1}(B_{<n}) \ar[ru] \ar[rr]& & H_{n-1}(C)
    \end{tikzcd}\]
    the class of \(x\) maps to 0 and the map becomes injective and hence an isomorphism.

    We let \(B\) be the subcomplex of \(C\) generated by all \(B_i\) for all \(i \geq 0\). Then the inclusion \(B \to C\) induces an isomorphism on \(H_i\) for all \( i \geq 0\), so it is a quasi-isomorphism.

    By the end of last term we prooved, it is already a chain homotopy equivalence!
\end{proof}


\newLecture{30.04.2025}


\begin{thm}{Algebraic Künneth theorem, cohomology}{}
    Let R be a commutative ring of global dimension \(\leq 1\). Let \(C,D\) be chain complexes of abelian groups such that \(C_n = 0 = D_n\) for \(n < 0\) and all \(C_n\) are free and \(H_n(C)\) is finitely generated free.

    Then for all \(n \geq 0\):
    \[\bigoplus_{p+q = n} H^p(\Hom(C,R)) \otimes_R H^q(\Hom(D,R)) \xrightarrow{\Phi} H^n(\Hom(C\otimes D, R))\]
    is injective and its cokernel is isomorphic to
    \[\bigoplus_{p+q = n+1} \Tor^R(H^p(\hom(C, R)), H^q(\Hom(D,R)))\]
\end{thm}

\textbf{Warning.} We do not assume, that there is a splitting.

\begin{proof}
    \enquote{Basically just putting all the hard stuff we've already done together in the right way.}
    \begin{description}
        \item[Case 1] Suppose that also \(C_n\) is finitely generated for all \(n \geq 0\). Then \(\bullet \colon \Hom(C,R) \otimes _R \Hom(D,R) \to \Hom(C\otimes D, R)\) is an isomorphism of cochain complexes. Applying the homological algebraic Künneth theorem to
        \[H^n(\Hom(C\otimes D,R)) \cong H^n(\Hom(C,R) \otimes_R \Hom(D,R))\]
        since \(C_n\) is finitely generated and free, it is isomorphic to \(\IZ^k\) for some \(k \geq 0\), so \(\Hom(C,R)^n = \Hom(C_n,R) \cong \Hom(\IZ^k, R) = R^k\) which is free hence projective as an \(R\)-module for all \(n \geq 0\).

        \textbf{Caveat 1.} we make cochain complexes into chain complexes, then apply Künneth, then come back. This turns \(n-1\) in the \(\bigoplus\) for \(\Tor R\) into \(n+1\).

        \textbf{Caveat 2.} The proof of the homological Künneth theorem (without the splitting) used only that one complex is dimensionwise projective. Hence it is no problem, that \(D\) is not projective.

        \item[General case] We choose a subcomplex \(B\) of \(C\) such that \(B_n\) is finitely generated for all \(n \geq 0\) and \(B \hookrightarrow C\) is a chain homotopy equivalence. Then
        \[\Hom(i, R) \colon \Hom(B,R) \to \Hom(C,R)\]
        is a chain homotopy equivalence of \(R\)-module complexes.\footnote{This is due to the Hom-functor being additive. Unfortunately I don't know what that means.}
        
        \textbf{Note} Additive functors preserve chain homotopy equivalences, however not quasi-isomorphisms. Because of that, quasi-Isomorphisms and chain homotopy equivalences are quite different.

        Similarly we see
        \[\Hom(i\otimes D, R) \colon \Hom(C\otimes D, R) \to \Hom(B\otimes D, R)\]
        is a chain homotopy equivalence.

        This gives a commutative square in \(\mathbf{coChains}_R\):
        \[\begin{tikzcd}
            {\bigoplus\limits_{p+q = n} H^p(\Hom(C,R)) \otimes H^q(\Hom(D,R))} \ar[r, hook, "\Phi"] \ar[d, "\cong"] & {H^n(\Hom(C\otimes D, R))} \ar[d, "\cong"] \\
            {\bigoplus\limits_{p+q = n} H^p(\Hom(B,R)) \otimes_R H^q(\Hom(D,R))} \ar[r, hook, "\Phi", "\text{by special case}"'] & {H^n(\Hom(B\otimes D, R))} \\
        \end{tikzcd}\]
    \end{description}
\end{proof}

\begin{construction}
    Let \(X,Y\) be spaces or simplicial sets. \(R\) an commutative ring. The \emph{exterior cup product} \index{exterior cup product}
    \[\times \colon H^p(X,R) \times H^q(Y,R) \to H^{p+q}(X\otimes Y, R)\]
    is defined by \((x,y) \mapsto p_1^*(x) \cup p_2^*(y)\), where \(p_1 \colon X\times Y \to X\) and \(p_2 \colon X \times Y \to Y\).
\end{construction}

\textbf{Recall.} The AW-map is
\[\mathrm{AW}\colon C_*(X\times Y) \to C_*(X) \otimes C_*(Y)\]

\begin{proposition}
    Let \(X, Y\) be simplicial sets, \(R\) commutative ring. Then the composite
    \[H^p(X,R) \otimes_R H^q(Y,R) \xrightarrow[[f]\otimes [g] \mapsto [f\otimes g]]{\Phi} H^{p+q}(\Hom(C_*(X) ,R) \otimes_R \Hom(C_*(Y),R)) \xrightarrow{H^{p+q}(\bullet)} H^{p+q} (\Hom(C_*(X) \otimes C_*(Y), R)) \xrightarrow{H^{p+q}(\mathrm{AW})} H^{p+q}(\Hom(C_*(X \otimes Y), R)) = H^{p+q}(X\otimes Y, R)\]

    equals the external cup product.
\end{proposition}

\begin{proof}
    In the notes.
\end{proof}


\begin{thm}{Künneth theorem in cohomology}{}
    Let \(R\) be a commutative ring of global dimension \(\leq 1\). Let \(X,Y\) be spaces such that \(H_n(X, \IZ)\) is finitely generated for all \(n \geq 0\). Then the total exterior cup product map
    \[\bigoplus_{p+q = n} H^p(X,R) \otimes_R H^q(Y,R) \to H^n(X \times Y, R)\]
    is injective, and its cokernel is naturally isomorphic to
    \[\bigoplus_{p+q = n+1} \Tor^R(H^p(X,R), H^q(Y,R))\]
\end{thm}

\begin{proof}
    Similar to the homological one. Use the cohomological algebraic Künneth theorem and the Eilenberg-Zilber theorem. You can read it up somewhere.
\end{proof}

\begin{remark}
    Let \(X\) be a CW-complex of finite type \index{finite type CW-complex} i.e. such that it has only finitely many cells in every dimension. (ex. \(\IR P ^\infty\)). Then
    \[C^{Cell}_*(X,\IZ)\]
    is finitely generated free in every dimension, hence \(H_n^{cell}(X,\IZ) \cong H_n(X, \IZ)\) if finitely generated, so Künneth theorem applies.
\end{remark}

\begin{construction}
    Let \(A,B\) be graded-commutative\footnote{\(a \cdot b = (-1)^{\deg(A) \cdot\deg(B)} b\cdot a\)} rings. Then \(A\otimes B\) is another graded-commutative ring by
    \[(A\otimes B)_n = \bigoplus_{p+q = n} A_p \otimes B_q\]
    with multiplication for \(a \in A_p, b \in B_q, a' \in A_{p'}, b' \in B_{q'}\).
    \[(a\otimes b) \cdot (a' \otimes b') = (-1)^{p' \cdot q} (aa') \otimes (bb')\]

    Check for well-definedness yourself.
\end{construction}

\begin{corollary}
    Let \(R\) be a field, \(X,Y\) spaces and suppose, that \(H_n(X, \IZ)\) is finitely generated for all \(n \geq 0\). Then
    \[\times\colon H^*(X,R) \otimes_R H^*(Y,R) \to H^*(X\times Y, R)\]
    is an isomorphism of graded-commutative \(R\)-algebras.
\end{corollary}

\textbf{Note.} We already knew that this is a isomorphism of abelian groups. The new information is, that this is compatible with ring structure.

\begin{proof}
    We take \(x \in H^p(X,R), x' \in H^{p'}(X,R) , y \in H^q(Y,R) , y' \in H^{q'}(Y,R)\) and then
    \[\begin{split}
        (x \cup x') \times (y \cup y') &= p_1^*(x \cup x') \cup p_2^*(y \cup y')\\
        &= (p_1^*(x) \cup p_1^*(x')) \cup (p_2^*(y) \cup p_2^*(y')) \\
        &= (-1)^{p'\cdot q} (p_1^*(x) \cup p_2^*(y)) \cup (p_1^*(x') \cup p_2^*(y')) \\
        &=(-1)^{p\cdot q'} (x \times y) \cup (x' \times y')
    \end{split}\]
\end{proof}

\begin{corollary}
    Let \(X,Y\) be spaces such that \(H_n(X, \IZ)\) is finitely genrated and free for all \(n \geq 0\). Then
    \[H^*(X, \IZ) \otimes H^*(Y,\IZ) \to H^*(X\times Y, \IZ)\]
    is an isomoprhism of graded-commutative rings.
\end{corollary}

\section{Example calculations for Cohomology-rings}

Now we are actually calculating some cohomology rings. Namely for \(S^k \times S^l\), \(S^1 \times \dots \times S^1\) and \(\IC P^2\).

Remember
\[H^n(S^k) = \begin{cases}
    \IZ & n = 0,k \\
    0 & n \neq 0,k \\
\end{cases}\]
and assume \(k \geq 1\). For dimensional reasons, the cup product on \(H^*(S^k, \IZ)\) is trivial. \(H^*(S^k, \IZ)\) is dimensionwise finitely generated free, and hence for every space \(Y\) the exterior cup product
\[H^*(S^k, \IZ) \otimes H^*(Y, \IZ) \to H^*(S^k \times Y, \IZ)\]
is an isomorphism of graded-commutative rings. Take \(Y = S^l\) for \(l \geq 1\).

Let \(e_k \in H^n(S^k; \IZ)\) be one of the two generators. Then \(H^+(S^k, \IZ) = \Lambda(e_k)\) where \(\Lambda\) denotes an exterior product. This includes \(e_k^2 = 0\). We define \(a \coloneq p_1^*(e_k) \in H^k(S^k \times S^l; \IZ)\) and \(b\coloneq p_2^*(e_l) \in H^l(S^k \times S^l; \IZ)\). Then 
\[H^*(S^k, \IZ) \otimes H^*(S^l, \IZ) \xrightarrow[x]{\cong} H^*(S^k \times S^l; \IZ) = \IZ\set{1 \times 1, 1 \times e_l, e_k \times 1, e_k \cdot e_l}\]
where we have \(1\times 1 = 1, 1 \times e_l = b, e_k \times 1 = a, e_k \cdot e_l = a\cup b\).

We look at multiplicative relations:
\[a^2 = 0, b^2 = 0\]
and so
\[a^2 = (p_1^*(e_k))^2 = p_1^*(e_k^2) = p_1^*(0) = 0\]

If \(k \) or \(l\) is even, then \(a \cup b = b\cup a\) and if both are odd, then \(a\cup b = - b\cup a\).

We summarize, if \(k\) and \(l\) are even, then
\[H^*(S^k \times S^l; \IZ) = \IZ[a,b]/(a^2 = 0, b^2 = 0)\]
and if one is odd
\[H^*(S^k \times S^l; \IZ) = \Lambda(a,b)\]
where \(\Lambda\) again denotes exterior products.


We give an inductive description of  \(H^*(S^1 \times \dots \times S^1; \IZ)\) \(n\)-times. We use, that
\[\times \colon H^*(S^1; \IZ) \otimes H^*(\underbrace{S^1 \times \dots \times S^1}_{n-1 \text{ times}}) \cong H^*(S^1\times \dots \times S^1, \IZ)\]
 we define \(a_i = p_i^*(e_1) \in H^1(\underbrace{S^1 \times \dots \times S^1}_{n}; \IZ)\), where \(p_i \colon (S^1)^n \to S^1\) is projection to the \(i\)-th factor for \(1 \leq i \leq n\). We get \(a_i^2 = 0\) and \(a_i \cup a_j = - a_j \cup a_i\) for \(i \neq j\). This gives us, that an additive basis of \(H^*(S^1)^n; \IZ\) is given by
 \[a_{i_1} \cup \dots \cup a_{i_k} \text{ for all tuples } 1 \leq a_i < a_2 < \dots < a_k \leq n\]
 This gives us \(\mathrm{rank}(H^*((S^^1)^n \IZ)) = 2^n\).
 The multiplicative structure is given by \(H^*((S^1)^n, \IZ) = \Lambda(a_1, \dots, a_n)\).

 Later we will compute \(H^*(\IC P^n; \IZ)\) via Poincaré-duality to get \(\cong \IZ[X]/(X^{n+1})\) for \(x \in H^2(\IC P^n, 2)\).

 We will now use a trick to at least calculate \(H^*(\IC P^2; \IZ)\). We know, that
 \[H^n(\IC P^2, \IZ) = \begin{cases}
    \IZ & n = 0,2,4 \\
    0 & \text{else}
 \end{cases}\]
 we take \(x \in H^2(\IC P^2; \IZ)\) a generator. The multiplicative structure is completely defined by which multiple of the generator of \(H^4(\IC P^2, \IZ)\) \(x^2\) is.

We use homogenous coordinate noatation for \(\IC P ^2\). For \(0 \neq (x,y,z) \in \IC^3\) we write \([x,y,z] \coloneq \IC \cdot (x,y,z) \in \IC P ^2\). We define a continuous map
\[\mu\colon \IC P ^1 \times \IC P^1 \to \IC P ^2\]
given by \(([v,w] , [x,y]) \mapsto [vx, vy+wx, wy]\).
We let \(e = [1,0]\) a basepoint in \(\IC P^1\). Then \(\mu(e, \_), \mu(\_, e) \colon \IC P ^1 \to \IC P ^2\).
are both the \enquote{standard inclusions} \([x,y] \mapsto [x,y,0]\).

\begin{proposition}
    The map \(\mu^* \colon H^4(\IC P^2, \IZ) \to H^4(\IC P^1 \times \IC P^1, \IZ)\) is injective and its image has index 2.
\end{proposition}

proof next time.

\newLecture{5.05.2025} Rather sleepy today, quality may be accordingly.

\textbf{Note.} Remember \(\IC P^2 \cong S^2\).

\begin{proof}
    We will drop coefficients from the notation. \(H^*(X) \coloneq H^*(X; \IZ)\). The continuous map \(\IC^2 \to \IC P^2\), \(\pi(a,b) = (a^2-b, 2a, 1)\) is an open embedding and a homoeomorphism onto the open \(4\)-cell \(\IC P^2 \setminus \IC P^1\). That is just the set \([x,y,1]\) for \((x,y) \in \IC^2\). Then
    \[(x,y) = (a^2 -b, 2a) \implies a = y/2, b = (a-x = y^2/4 -x)\]

    This gives an isomorphism of relative cohomology groups
    \[\pi^*\colon H^4(\IC P^2 \setminus \IC P ^1, \IC P^2 \setminus (\IC P ^1 \cup [0,0,1])) \to H^4(\IC^2, \IC^2\setminus (0,0))\]
    Then we have an excision isomorphism:
    \[H^4(\IC P^2, \IC P ^2\setminus [0,0,1]) \cong H^4(\IC P^2 \setminus \IC P ^1, \IC P ^2 \setminus \IC P ^1 \cup [0,0,1])\]
    The long exact sequence of the pair gives an isomorphism
    \[H^4(\IC P ^2, \IC P ^2 \setminus[0,0,1]) \to H^4(\IC P ^2)\]

    We also Define
    \[\pi'\colon \IC^2 \to \IC P ^1 \times \IC P ^1, (a,b) \mapsto ([a+b, 1], [a-b,1])\]
    A similar calulation gives
    \[H^4(\IC P ^1 \times \IC P ^1, \IC P^1 \times \IC P ^1\setminus ([0,1], [0,1]))\]
    as isomorphic .to \(H^4(\IC P ^1 \times \IC P ^1)\).

    We now also define \(\nu\).
    \[\nu \colon \IC^2 \to \IC ^2 ; \quad (a,b) \mapsto (a, b^2)\]

    Now a diagram I didn't copy commutes.


    The problem nor reduces to show that
    \[\nu^*\colon H^4(\IC^2, \IC^2\setminus (0,0)) \to H^4(\IC^2, \IC^2\times 0) , 0\]
    is multiplication by 2.

    A diagramm I didn't copy. He applied Künneth and found out some map is multiplication by 2.
\end{proof}

\begin{proposition}
    Let \(x \in H^2(\IC P^2, Z)\) be an additive generator. Then \(x^2\) is an additive generator of \(H^4(\IC P^2, \IZ)\). So \(H^*(\IC P^2, \IZ)\) is a truncated polynomial algebra i.e.
    \[H^*(\IC P ^2, \IZ) = \IZ[X]/ (x^3)\]

\end{proposition}

\textbf{Outlook.} \(H^*(\IC P^m; \IZ) = \IZ[X]/(x^{m+1})\)
This will be proven later using Poincaré-Duality.

\begin{proof}
    We write \(i \colon \IC P ^1 \to \IC P ^2\) for \enquote{the inclusion}, \(i[x,y] = [x,y,0]\). Then
    \[H^*(\IC P^1, \IZ) = \IZ\set{1, i^2(x)}\]

    \[\times\colon H^*(\IC P^2) \otimes H^*(\IC P^1) \cong H^*(\IC P^1\times \IC P^1)\]
    we write \(a \coloneq p_1^*(i^*(x)), b \coloneq p_2^*(i^*(x))\). Then
    \[H^*(\IC P^1 \times \IC P^1) = \IZ\set{1,a,b,a\cdot b}\]
    with \(a^2 = b^2 = 0, ab = ba\).

    \textbf{Clain.} We have
    \[\begin{tikzcd}
        \mu^*(x) = a+b \ar[r, phantom, "\in"] &  H^2(\IC p^1 \times \IC P^1) \ar[d, "\cong"] \\
        & H^2(\IC P^1 \vee \IC P^1) \ar[d, "\cong"] \\
        & H^2(\IC P^1) \times H^2(\IC P^1)
    \end{tikzcd}\]
    where we use that the wedge is an isomorphis on the \(2\)-skeleton of \(\IC P^1 \times \IC P^1\). The composite map is given by
    \[z \mapsto ((e, \_^*(z)), (\_, e)^*(z))\]

    We note
    \[\begin{split} 
        (e,\_)^*(a+b) &= (e, \_)^*(p_1^*(i^*(x))) + (e, \_)^*(p_2^*(i^*(x))) \\
        &= (i \circ \underbrace{p_1 \circ(e, \_)}_{\text{constant}})^*(x) + (i \circ \underbrace{p_2 \circ (e, \_)}_{\text{identity}})^*(x) \\
        &=  i^*(x)
    \end{split}\]
    and also
    \[\begin{split}
        (e, \_)^*(\mu^*(x)) &= (\underbrace{\mu \circ (e, \_)}_{= 1})^* = i^*(x)
    \end{split}\]
    This gives \(\mu^*(x) = a+b\)
    Now let \(y\in H^4(\IC P^2)\) be a generator and let \(n \in \IZ\) be such, that \(x^2 = n\cdot y\). Now
    \[2ab = (a+b)^2 = (\mu^*(x))^2 = \mu^*(x^2) = \mu^*(ny) = n \cdot \mu^*(y) = n\cdot 2 \cdot ab\]
    where the last equality uses degree \(2\) of \(\mu\). This holds in the free abelian group \(H^4(\IC P^1 \times \IC P^1) = \IZ\set{a,b}\). This means \(2 = 2n\) and hence \(n = 1\) and so j\(x^2 = y\).
\end{proof}

\underline{Application to the Hopf map.}

The Hopf map \(\eta \colon S^3 \to S^2\) is defined as
\[S^3 = S(\IC^2) \to \IC P^1 \cong S^2\]
given by \((x,y) \mapsto [x,y]\).

Then \(0 \neq [y] \in \pi_3(S^2, *) \cong \IZ\set{y}\).

\begin{proposition}
    Attaching a \(4\)-cell to \(\IC P^1\) yields a space homeomorphic to \(\IC P ^2\). Informally: \enquote{\(\eta\) is the attahing map of the \(4\)-cell in \(\IC P^2\).}
\end{proposition}

\begin{proof}
    Consider the map \(\alpha\colon D(\IC^2) \to \IC P^2\), \((x,y) \mapsto [x,y, 1- \abs x ^2 - \abs{y}^2]\).

    This restricts to a homeomorphism from \(D(\IC^4) \setminus S(\IC^2)\) onto \(\IC P^2 \setminus \IC P^1\) and the following commutes:
    \[\begin{tikzcd}
        S(\IC^2) \ar[d, hook] \ar[r, "\eta"] & \IC P^1 \ar[d, "i"] & {[x,y]} \ar[d]\\
        D(\IC^2) \ar[r, "\alpha"] & \IC P^2 & {[x,y,0] }\\
    \end{tikzcd}\]
    this gives a well-defined continuous map \(D(\IC^2) \cup_{S(\IC^2), \eta} \IC P^1 \to \IC P^2\), This is a continuous bijection between compact Hausdorff spaces, hence a homeomorphism.
\end{proof}

\begin{thm}{Hopf map is not constant}{}
    The Hopf map \(\eta\) is not homotopic to a constant map.
\end{thm}

\begin{proof}
    By contradiction. If \(\eta\) was homotopic to the constant map \(c\colon S^3 \to S^2\), then \(D^4 \cup_{S^3, \eta} \IC P^1 \) would be homotopy-equivalent to \(D^4 \cup_{S^3, \text{const}} \IC P^1 = \IC P^1 \vee (D^4/S^3)  \cong S^2 \vee S^4\).

    These spaces have the same additive cohomology. However, their cup-product differs. Namely in \(H^*(\IC P^1 \vee S^4, \IZ)\) the square of every 2-dimensional class is \(0\).

    As such,
    \(\IC P^1 \vee S^4 \not\sim \IC P^2\).
\end{proof}

\textbf{Outlook.} The Hopf map is sometimes presented as the map
\[S(\IC^2) \to \IC \cup \set \infty = \text{one point compactification of} \IC \cong S^2\]
given by \((x,y) \mapsto x/y\). For \(\IH =\) the quaternions \(= \IR^4\) with the skew-field multiplication \(= \IR\set{1,i,j,k}\) and \(i^2 = j^2 = k^2 = ijk = -1\). And then we get
\[\nu \colon S^7 = S(\IH^2) \mapsto \IH \cup \set \infty = S^4\]
given by \((x,y) \mapsto x/y = xy^{-1} \vee y^{-1}x\). This map is also called the second Hopf-map. Using that most of linear algebra still applies to skew-fields, we can define \(\IH P^n\) and see by a similar argument, that \(\nu\) is not nullhomotopic. Then \([\nu] \in \pi_7(S^4, *) \cong \IZ\set{\nu} \oplus \IZ/?\)
Schwede doesn't remember what exactly \(\pi_7\) is.

Then we also have \(\IO = \text{Cayley octonians} = \IR^8\) with a nonassociative, noncommutative division algebra structure \(\cdot \IO \times \IO \to \IO\). Then there is still an \(\IO P^2\) but no general \(\IO P^n\).

However this is enough to still calculate that \(H^*(\IO P^2, \IZ) = \IZ[w]/w^3\) where \(w \in H^8(\IO P^2, \IZ)\). And you can show
\[\sigma \colon S(\IO^2) \to \IO P^1 = \IO \cup \set \infty\]
given by \((x,y) \mapsto x/y\) is non zero-homotopic. And \([\sigma] \in \pi_{15}(S^8) = \IZ \oplus \IZ/120\).

He also talks about a theorem, that these are all the Hopf-Maps that exist. No more in higher dimensions.

\newLecture{07.05.2025}

\part{Manifolds and Poincaré Duality}

\chapter{Topological manifolds}

The long-time goal is to proove Poincaré duality. For that we first need to study manifolds.

\begin{defi}{Manifold}{}
    An \(m\)-manifold \index{manifold} is a Hausdorff space \(M\) such that every point of \(M\) has an open neighborhood homeomorphic to \(\IR^m\).\footnote{This is sometimes called a topological manifold to differentiate from smooth ones.}
\end{defi}


\begin{remark}
    \begin{itemize}
        \item The empty space is an \(m\)-manifold for all \(m \geq 0\).
        \item Let \(M\) be a non empty manifold. Then the dimension \(m\) is an intrinsic invariant. Let \(x \in M\) be a point, let \(U\) be an open neighborhood of \(x\) homeomorphic to \(\IR^m\). Let \(\varphi\colon \IR^m \to U\) be a homeomorphism such that \(\varphi(0) = x\). Then
        \[H_i(M, M \setminus \set x, \IZ) \xleftarrow{\cong} H_i(U, U\setminus \set x, \IZ) \xleftarrow{\varphi_*} H_i(\IR^m, \IR^m \setminus \set 0, \IZ)\]
        where we use excision for the first homeomorphism. Furthermore we see
        \[H_i(\IR^m, \IR^m\setminus \set 0, \IZ) \sim H_i(D^m, S^{m-1}, \IZ) = \begin{cases}
            \IZ & i = m \\
            0 & i \neq m \\
        \end{cases}\]
        We call this the local homology of \(x\). From this we can reproduce the dimension of \(M\).
        \item The Hausdorff condition is important to rule out pathological examples such as the \enquote{line with double origin}:
        \[\IR \amalg \IR/(x,0)\sim (x,1) \text{ for all } x \in \IR \setminus \set 0\]
        Can't draw the picture of the space.

        This is not Hausdorff, but locally \(\IR^1\). we don't want this to be a manifold.
    \end{itemize}
\end{remark}

\begin{example}
    \begin{itemize}
        \item open subsets of \(\IR^m\) are \(m\)-manifolds.
        \item Let \(M\) be a Hausdorff space, such that every point has an open neighborhood that is an \(m\)-manifold.Then \(M\) is an \(m\)-manifold.
        \item Let \(M\) be an \(m\)-manifold and \(N\) an \(n\)-manifold. Then \(M\times N\) is an \(m+n\)-manifold.
        \item The \(m\)-sphere \(S^m = \set{(x_1, \dots, x_{m+1}) \in \IR^m \mid x_1^2 + \dots + x_{m+1}^2 = 1}\) is an \(m\)-manifold.
        
        Let \(x = (x_1, \dots, x_{m+1}) \in S^m\) be a point. Let \(V = \set{y \in \IR^{m+1} \mid \skal{y,x} = 0}\) be the orthogonal complement of \(x\). The stereographic projection is a homeomorphism
        \[x \in S^m \setminus \set {-x} \to V\]
        given by some formula I couldn't copy before it was erased and he also had a nice picture.
        \item The real projective space \(\IR P^m \cong S^m/x \sim -x\) is an \(m\)-manifold. Let \(\set{x, -x}\) be a point in \(\IR P^m\) for \(x \in S^m\). Let \(x\) be one of the representatives. Let \(\IR^m \cong U = \set{z \in S^m \mid \skal{z,x} \geq 0}\) \enquote{The northern hemisphere with north-pole \(x\)}. As \(U \subseteq S^m\) we get via projection a map to \(\IR P^m\). This is an open embedding onto a neighborhood.
        
        \item Let \(\IC P^m = \set{l \in \IC^{n+1} : L \text{ complex line through } 0}\). is a \(2m\) manifold. Consider first the point \([0,0, \dots, 0, 1]\).
        
        Then \(\IR^{2n} \cong \IC^n \to \IC P ^n\) given by \((z_1, \dots, z_m) \mapsto [z_1, \dots, z_m, 1]\) is an homeomorphism onto a open neighborhood \(U\) of \([0,0,\dots, 0, 1]\).

        Let \(l \in \IC P^n\) be any point, let \(v \in l\) be a nonzero vector in \(l\). Let \(A \in Gl_n(\IC)\) such that \(A\cdot (0, \dots, 0,1) = v\).
        Then \(A \colon \IC P^n \to \IC P^n\), \(L_0 = [0, \dots, 0,1]\) given by \(L \mapsto A\cdot L\) sends \(A \cdot L_0 = L\). So we can take \(A(U)\) as an open neighborhood of \(L\) homeomorphic to \(\IR^{2n}\).
    \end{itemize}
\end{example}

Now we do some examples that are a little more involved.

\begin{construction}[Stiefel manifold]
    Let \(0 \leq k \leq n\). The Stiefel manifold \(V_{k,n} = \set{(v_1, \dots, v_k) \in \IR^n} \mid \text{orthonomal set.}\). We call this the \enquote{k-frame} this means
    \[\skal{v_i, v_j} = \begin{cases}
        1 & i = j \\
        0 & i \neq j
    \end{cases}\]
    note, that each \(v_i\) is a vector in \(\IR^n\). We give \(V_{k,n}\) the subspace topology of \((\IR^{n})^k\). This is even a closed subspace of \((S^{n-1})^k\), so \(V_{k,n}\) is compact.
\end{construction}

Examples are,
\begin{itemize}
    \item \(V_{0,n} = \set \emptyset\) is a point hence a \(0\)-manifold.
    \item \(V_{1,n} = S^{n-1}\).
    \item \(V_{n,n} = O(n)\) the \(n\)-th orthogonal group.
    \item \(V_{n-1, n} \xleftarrow{\cong} SO(n)\) given by \((Ae_1, \dots, A\cdot e_{n-1}) \mapsfrom A\) where \(e_i = \prescript{t}{}{(0,\dots, 1, \dots, 0)}\). This is bijective because it sends orthogonal matrices to the orthogonal vectors that span it. That is not what was written on the board. That was erased before I could copy.
\end{itemize}

\begin{proposition}
    \(V_{k,n}\) is a manifold of dimension \((n-1) + (n-2) + \dots + (n-k) = nk - \frac{k(k+1)}{2}\)
\end{proposition}
\begin{proof}
    By induction on \(k\). We have already seen \(V_{0,n} = \set \emptyset \) as a \(0\)-manifold and \(V_{1,n} = S^{n-1}\) a \((n-1)\)-manifold.

    Now let \(k \geq 2\). Let \(S^{n-1}_+ = \set{(x_1, \dots, x_n) \in S^{n-1} : x_1 \geq 0}\) be the \enquote{northern hemisphere}. We define a continuous map \(\psi \colon S^{n-1}_+ \to O(n)\) as the following composite
    \[S^{n-1}_* \to GL_n(\IR) \xrightarrow{\text{Gram-Schmidt}} O(n) \quad w \mapsto (\trans w, e_2, \dots, e_n) \mapsto \dots\]
    where Gram-Schmidt is a continuous way to orthonormalize a matrix.

    We remember the properties:
    \begin{itemize}
        \item \(\psi\) is contnuous
        \item \(\psi(e_1) = \psi(1,0,0,\dots, 0) = E_n\)
        \item \(\psi(w) \cdot e_1 = w\).
    \end{itemize}

    \textbf{Warning.} Ther is no continuous map \(\tilde \psi \colon S^{n-1} \to O(n)\) such that \(\tilde \psi (w) \cdot e_1 = w\).

    We show the manifold condition around \((e_1, \dots, e_k) \in V_{k,n}\). We set \(U = \set{(v_1, \dots, v_k) \in V_{k,n} : v_1 \in S^{n-1}_+}\) is open in \(V_{k,n}\) around \((e_1, \dots, e_k)\). The map
    \[U \to S^{n-1} \times V_{k-1, n-1}, \quad (v_1, \dots, v_k) \mapsto (v_1, (\psi(v_1))^{-1}(v_2), \dots, (\psi(v_1))^{-1}(v_k))\]
    where \((\psi(v_1))^{-1}(v_i)\) are in \(0 \times \IR^{n-1}\). The well-definedness follows from \(\psi(v_1)^{-1}\) is an orthogonal matrix such that \(\psi(v_1)^{-1}(v_1) = e_1\). This means, that \(\psi(v_1)^{-1}(v_2, \dots, v_k)\) will be an orthonormal \(k-1\)-set that is also orthogonal to \(e_1\), i.e. they sit in \(0 \times \IR^{n-1}\). He also rambles, as to why this is continuous.

    It is a homeomorphism. This shows that around \(e_1, e_2, \dots, e_k\) \(V_{k,n}\) is locally a manifold of dimension \((n-1) + \dim (V_{k-1,n-1}) = (n-1) + (n-2 + \dots + n-k)\).

    We have a continuous inverse:
    \[S^{n-1}_+ \times V_{k-1, n-1} \to U \quad (v, w_1, \dots, w_{k-1}) \mapsto (v, \psi(v)(0,w_1), \dots, \psi(v)(0,w_{k-1}))\]

    Now let \((v_1, \dots, v_k) \in V_{k,n}\) be any point. We choose an extension to an orthonormal basis \((v_1, \dots, v_k, v_{k+1} , \dots, v_n)\). Set \(A = (v_1, \dots, v_n) \in O(n)\). then
    \[A \cdot \_\colon V_{k,n} \to V_{k,n}\]
    is a self homeomorphism that sends \((w_1, \dots, w_k) \mapsto (A \cdot w_1, \dots, A \cdot w_k)\) and specifically \(e_1, \dots, e_k\) to \(v_1, \dots, v_k\). So the homeomorphism takes the previous neighborhood \(U\) homeomorphically onto the neighborhood \(A \cdot U\) of \((v_1, \dots, v_k)\)
\end{proof}

\begin{remark}
    What we really showed is, that \(V_{k,n} \to S^{n-1}\), \((v_1, \dots, v_k) \mapsto v_1\) is a smooth locally trivial fiberbundle with fiber \(V_{k-1, n-1}\).
\end{remark}

\textbf{Note.} Complex Stiefel Manifold. We can also define
\[V_{k,n}^\IC = \set{(v_1, \dots, v_k) \in \IC^n : \skal{v_i, v_j} = \delta_{i,j}}\]
where \(\delta\) denotes the Kronecker-symbol and we use the hermitian complex bilinear product. 

This is a manifold of dimension \((2n-1) + (2n-3) + (2n-5) + \dots + (2n-2k +1) = 2nk - k^2\). We will see.
\[V_{0,n}^\IC = \set *, \quad V_{1,n}^\IC = S^{2n-1}, \quad V_{n-1,n} \cong SU(n), \quad V_{n,n} \cong U(n)\]

For the quaternions \(\IH = \IR\set{1, i,j,k}\) with \(i^2 = j^2 = k^2 = ijk = -1\), we have quaternionic conjugation \(\lambda = a + bi + cj + dk \to \bar \lambda a- bi - cj - dk\) that is an anti-isomorphism: \(\bar{\lambda \cdot \mu} = \bar \mu \cdot \bar \lambda\). This gives a \enquote{Quaternionic skalar product} on \(\IH^n\) is defined by \([x,y] \coloneq \bar{x_1} y_1 + \dots + \bar{x_n} \cdot y_n\) for \(x,y \in \IH^n\). This is an \(\IH\)-sesquilinear, non degenerate positive definete \(\IR\)-bilinear form.

With the right definitions and being careful, all of this works.

This gives Quaternionic Stiefel manifolds:
\[V_{k,n}^\IH = \set{(v_1, \dots, v_k) \in (\IH^n)^k : [v_i, v_j] = \delta_{i,j}}\]
is a manifold of dimension \((4n-1) + (4n-5) + \dots + (4n - 4k +3) = 4nk - k(2k-1)\). And we see again
\[V_{1,n}^\IH = S^{4n-1}, \quad V_{n,n}^\IH = Sp(n) = \set{A \in M(n\times n, \IH) : A\cdot \bar A ^t = \bar A ^t \cdot A = E_n}\]
Where \(Sp\) is the simpletic group. There is no such thing as a special simpletic group, because you would need determinant for that, which then really needs commutativity.

\begin{construction}[Graßmann manifolds]
    Let \(0 \leq k \leq n\) The Graßmann manifold of \(k\)-pairs in \(\IR^n\) is
    \[Gr(k,n) = Gr_k(n) = Gr_k(\IR^n) = \set{L \subseteq \IR^n : L \text{ is } k\text{-dimensional } \IR\text{-subspace.}}\]
    There is a surjective map
    \[\mathrm{span}\colon V_{k,n} \to Gr(k,n) \quad (v_1, \dots, v_k) \mapsto \mathrm{span}(v_1, \dots, v_k).\]
    we give \(Gr(k,n)\) the quotient topology. Next time we will see \(Gr(k,n)\) is a compact manifold of dimension \(k \cdot (n-k)\).

    The map \(Gr(k,n) \mapsto Gr(n-k,n)\) given by \(L \mapsto L^\perp\) is a homeomorphism.
\end{construction}

\newLecture{12.05.2025}

\begin{Example}
    We have \(Gr(1,n) = \IR P^{n-1}\).
\end{Example}

\begin{thm}{Graßmann Manifolds}{}
    \(Gr(k,n)\) is a compact manifold of dimension \(k \cdot (n-k)\).
\end{thm}

\begin{proof}
    We first show compactness. Quasicompactness is clear, as it is a quotient space of a compact space.

    We will show Hausdorff by constructing an injection into a Hausdorff space. For \(V \in Gr(k,n)\) we consider the orthogonal projection \(p_V\colon \IR^n \to \IR^n\). Let \((v_1, \dots, v_k)\) be an orthonomal basis. then
    \[p_V(x) = \skal{x,v_1} \cdot v_1 + \dots + \skal{x, v_k} \cdot v_k\]
    We will sometimes also write \(p_V \colon \IR^n \to \IR^k\).

    The map \(Gr(k,n) \to \Hom_\IR(\IR^n \to \IR^n)\) given by \(V \mapsto p_V\) is injective. Claim: this map is continuous.

    By the quotient topology, we need to show, that the composite \(V_{k,n} \to Gr(k,n) \to \Hom_\IR(\IR^n, \IR^n) \) is continuous. This map is
    \[(v_1, \dots, v_k) \mapsto \sum_{i = 1, \dots, k} \skal{\_, v_i} \cdot v_i\]
    and as a sum of continuous maps it is continuous.
    Because \(Gr(kn)\) admits an injective continuous map to a Hausdorff space, it is Hausdorff.

    \textbf{Manifold property.} Let \(V \in Gr(k,n)\) be any \(k\)-plane. Set \(U \coloneq \set{L \in Gr(k,n) : L \cap V^\perp = \set 0}\). Claim: \(U\) is an open subset of \(Gr(k,n)\). We choose an orthonormal basis \((v_1, \dots, v_k)\) of \(V\).

    \textbf{Claim.} \(\spn{-1}(U) = \set{(l_1, \dots, l_k) : \det(\skal{l_i, v_j}_{1 \leq i,j \leq k}) \neq 0} \subseteq V_{k,n}\).

    If we show this, we are done, as \(\det \neq 0\) is an open condition.

    \textbf{Note.} \(V^\perp\) is the kernel of \(p_V \colon \IR^n \to \IR^n\). So \(L \cap V^\perp = \set 0 \Lra pr\rvert_L \colon L \to V\) is injective.
    
    As \(\dim(L) = \dim(V) = k\), this is equal to \(pr\rvert_L\colon L \to V\) is bijective. Since \((\skal{l_i, v_j})_{1 \leq i,j \leq k}\) is the matrix that expresses \((pr)\rvert_L\) in terms of the basis \((l_i)_{1 \leq i \leq k}\) and \((v_j)_{1 \leq j \leq k}\), this is equivalent to \(\det(\skal{l_i, v_j}) \neq 0\).

    The map \(V_{k,n} \to \IR, (l_1, \dots, l_k) \mapsto \det(\skal{l_i, v_j})\) is continuous, so \(\spn^{-1}(U)\) is open in \(V_{k,n}\), hence \(U\) is open in \(Gr(k,n)\).

    Next, we exhibit a homeomorphism
    \[\begin{tikzcd}
        U \ar[r, bend left, "\Psi"] & {\Hom_\IR(V, V^\perp)} \ar[l, bend left, "\Gamma"]
    \end{tikzcd}\]
    We then use \(\dim(V) = k, \dim(V^\perp) = n-k\), so \(\Hom_\IR(V, V^\perp) \cong \IR^{k(n-k)}\).

    Note that \(\Gamma(f) \cap V^\perp = \set{v \oplus f(V) : v = 0} = \set{0,0}\).

    We define \(\Gamma\colon \Hom(V, V^\perp) \to U\) using that \(\IR^n = V \oplus V^\perp\). Then
    \[\Gamma(f\colon V \to V^\perp) = \text{Graph of } f = \set{v \oplus f(v) : v \in V}\]
    The graph map factors as the composite after choice of orthonormal basis \(v_1, \dots, v_k\) of \(V\) as
    \[\Hom_\IR(V, V^\perp) \xrightarrow{\text{Gram-Schmidt}} V_{k,n} \xrightarrow{\spn} Gr(k,n)\]
    so \(\Gamma\) is a continuous map.

    We define \(\Psi \colon U \to \Hom_\IR(V, V^\perp)\) as follows: If \(L \in U\), then \(p_V\rvert_L \colon L \to V\) is a linear isomorphism.

    We define \(\Psi(L)\) as the composite \(V \xrightarrow{(p_V)\rvert_L^{-1}} L \xrightarrow{(p_{V^\perp})\rvert_L} V^\perp\).

    This is inverse to \(\Gamma\) by go check yourself.

    For Continuity of \(\Psi\colon U \to \Hom_\IR(V, V^\perp)\). Since \(\spn\colon V_{k,n} \to Gr(k,n)\) is a quotient map, so is its restriction
    \[\spn\colon \spn^{-1}(U) \to U\]
    So it suffices to show, that the composite
    \[\spn^{-1}(U) \to U \xrightarrow{\Psi} \Hom_\IR(V, V^\perp)\]
    is continuous.

    To proove that, we choose orthonormal bases \((v_1, \dots, v_k)\) of \(V\) and \(w_1, \dots, w_{n-k}\) of \(V^\perp\). Expressing a linear map in the basis is a linear isomorphism
    \[\Hom_\IR(V, V^\perp) \cong M(k \times (n-k), \IR).\]
    So we only need to show that
    \[\spn^{-1}(U) \xrightarrow{\spn} U \xrightarrow{\Gamma} \Hom_\IR(V, V^\perp) \cong M(k \times (n-k), \IR)\]
    is continuous. Did not copy the argument. Something about how we just compose matrices.
\end{proof}


\begin{corollary}
    The map \(Gr(k,n) \to P_{k,n} \coloneq {q \in \Hom_\IR(\IR^n, \IR^n) : q^2 = q = q^*, \; \mathrm{trace}(q) = k}\) is a homeomorphism.
\end{corollary}

\begin{corollary}
    For all \(0 \leq k \leq n\), the map \(Gr(k,n) \to Gr(n-k, n)\) given by \(V \mapsto V^\perp\) is an homeomorphism.
\end{corollary}

\begin{proof}
    We need only show continuity.
    \[\begin{tikzcd}
        Gr(k,n) \ar[d, phantom, sloped, "\cong"] \ar[r, "V \mapsto V^\perp"] & Gr(n-k, n) \ar[d, phantom, "\cong", sloped] \\
        P_{k,n} \ar[r, "f \mapsto \Id - f"] & P_{n-k, n} \\
    \end{tikzcd}\]
\end{proof}

We can define the compelx analogue: \(Gr^\IC(k,n) = \set{ L \subseteq \IC^n : \text/ Complex linear subspaces}\) with quotient topology by \(V_{k,n}^\IC \xrightarrow{\spn}Gr^\IC(k,n)\) is a compact manifold of dimension \(2k \cdot (n-k)\).

We can even define this for Quarternions:
\[Gr^\IH(k,n) = \set{L \subseteq \IH^n : \IH\text{- right submoule of dimension } k}\]
is a compactr manifold of dimension \(4\cdot k \cdot (n-k)\).

\textbf{Bigger picture.}  The orthogonal group \(O(n)\) acts transitively on \(V_{k,n}\). This gives an isomoprhism \(O(n)/(1 \times O(n-k)) \to V_{k,n}\), that is even a homeomorphism.

Similarly, we have a transitive action \(O(n) \to Gr(k,n)\). Looking at the stabilizor of \(\IR^k\). We get \(O(n)/O(k) \times O(n-k) \xrightarrow{\cong} Gr(k,n)\) an homeomorphism.

This works similarly for complex and quaternionic Stiefel/Graßmann manifolds. This can be summarized as: \enquote{Stiefel manifolds and Grassmannians are homogenous spaces}.

\textbf{Fact.} Let \(G\) be a liegroup. \(H\) a closed subgroup. Then \(G/H\) is a (smooth) manifold of dimension \(\dim G - \dim H\).


\section{Orientations}

We will (again) try \(H_n(X)\) for \(H_n(X, \IZ)\) and see how long it takes for him to forget he wanted to drop the \(\IZ\) from the notation.

\begin{notation}
For \(Y \subseteq X\), write \(H_n(X \mid Y)\coloneq H_n(X, X\setminus Y; \IZ)\), we call the \enquote{local homology of \(X\) at \(Y\)}\index{Local homology}.

This is because for \(Y \subseteq U \subseteq X\), \(U\) a neighborhood of \(Y\), then excision gives
\[H_n(U \mid Y) = H_n(U ; U\setminus Y; \IZ) \xrightarrow{\cong} H_n(X, X\setminus Y; \IZ) = H(X \mid Y)\]
\end{notation}

If \(M\) is an \(m\)-manifold, and \(x \in M\), then \(H_n(X \mid x) = H_n(X \mid \set x)\). This is \(\IZ\) iff \(m = n\) and else \(0\).

\begin{defi}{Local orientation}{}
    Let \(M\) be an \(m\)-manifold. A local orientation of \(M\) at  \(x \in M\) is a generator of \(H_m(X \mid x)\).
\end{defi}

There are exactly two local orientations at every point.

\begin{construction}[Orientation covering]
    Let \(M\) be an \(m\)-manifold. We define the set \(\tilde M = \set{(x, \mu) : x \in M, \mu \text{ is a local orientation at } x}\). This comes with a map \(p \colon \tilde M \to M\), \(p(x, \mu) = x\). This map is surjective and every point in \(M\) has exactly tow preimages.

    A subset \(B\) of \(M\) is a \emph{Local ball} if \(B\) is a local subset of \(M\), such that there exists a homeomorphism \(\phi \colon \IR^n \to M\) onto some open subset, sucht that \(\phi(\inner D^n) = B\).

    \textbf{Note.} IF \(B\) is a local ball in \(M\), then \(M\setminus B \to M \setminus \set x\) is a homotopy-equivalence (here we need the special definition of open ball). This induces a isomorphism \(r_x^B\colon H_m(M \mid B) \to H_m(X \mid x) \cong \IZ\) for all \(x \in B\).
    If \(\mu\) is a local orientation at \(x\), i.e. a generator of \(H_m(X \mid B)\), we set \(U(B, \mu) = \set{(x, r_x^B(\mu)) : x \in B} \subseteq \tilde M\).
\end{construction}

\begin{thm}{Orientation covering}{}
    Let \(M\) be an \(m\)-manifold.
    \begin{enumerate}
        \item As \((B, \mu)\) varies over all pairs of local balls \(B\) and generators \(\mu\) of \(H_m(M \mid B)\), the subset \(U(B, \mu)\) of \(\tilde M\) are the basis of a topology on \(\tilde M\).
        \item In this topology on \(\tilde M\), the map \(p \colon \tilde M \to M\), \(p(x, \mu) = x\) is a twofold covering, the orientation covering of \(M\). \index{Orientation covering}
        \item \(\tilde M\) is an \(m\)-manifold.
    \end{enumerate}
\end{thm}

\begin{proof}
    \begin{enumerate}
        \item We need to show, that for all local balls \(B, B'\) and all generators \(\mu \in H_m(X \mid B), \mu' \in H_m(X,B')\), the set \(U(B, \mu) \cap U(B', \mu')\) is a union of basiss sets. Let \((x, \nu) \in U(B, \mu) \cap U(B', \mu')\). so \(x \in B \cap B'\). and \(r_x^B(\mu) = r_x^{B'}(\mu') \coloneq \nu\).

        Choose a smaller local ball, s.t. \(x \in B'' \subseteq B \cap B'\). We consider the following diagram of local homology groups:
        \[\begin{tikzcd}
            H_m(X \mid B) \ar[rrd, "r_x^B", "\cong"', bend left=15]\ar[d] \ar[dr, "\cong"] & & \\
            H_m(X\mid B \cap B') \ar[r] & H_m(X \mid B'') \ar[r, "r_x^{B''}", "\cong"'] & H_m(X \mid x) \\
            H_m(X \mid B') \ar[rru, "r_x^{B'}", "\cong"', bend right=15] \ar[u] \ar[ur, "\cong"]\\
        \end{tikzcd}\]
        so \(\mu\) and \(\mu'\) map to the same generator of \(H_m(X \mid B'')\). Set \(\mu'' = \incl_*(\mu) = \incl'_*(\mu')\). Then \((x, \nu) \in U(B'', \mu'') \subseteq U(B, \mu) \cap U(B', \mu)\)

        \item Because \(M\) is a manifold, the local balls form athe basis of a topology of \(M\). So it suffices to establish for all local balls \(B\) in \(M\) a homeomorphism
        \[\begin{tikzcd}
            p^{-1}(B)\ar[d, "p"] \ar[r, phantom, "\cong"] & B\amalg B  \ar[dl, "fold"]\\
            B
        \end{tikzcd}\]
        I did not manage to copy the rest of this argument.

        \item is a special case of 
        \begin{proposition}
            Let \(p \colon N \to M\) be a covering map and \(M\) and \(m\)-manifold. Then \(N\) is an \(m\)-manifold.
        \end{proposition}
        \begin{proof}
            Hausdorff is clear. 

            For \(y \in N\) choose an open neighborhood \(U\) of \(p(y) = x\) in \(M\), such that \(U \cong \IR^m\) and \(p\) is locally trivial over \(U\). Choose a homeomorphism of \(p^{-1}(U) \cong U\times F\) for \(F\) some discrete space. Then \(U\times {f}\) is again homeomorphic to \(\IR^n\) and its preimage is an open neighborhood of \(y \in N\).
        \end{proof}
\end{enumerate}
\end{proof}


\newLecture{19.05.2025}
\footnote{I recently got a new laptop (including a new keyboard), which may negatively affect my writing speed and subsquently quality of the script for the next few lectures}


\begin{defi}{Orientation}{}
    An orientation\index{Orientation} of an \(m\)-manifold is a continuous section \(s\colon M \to \tilde M\) of \(p\) the orientation covering.
\end{defi}

\begin{defi}{Orientablity}{}
    An manifold is orientable, if it has an orientation.
\end{defi}

\begin{remark}
    If \(M\) is connected and orientable, \(s\colon M \to \tilde M\) an orientation. Then \(\tau \circ s \colon M \to \tilde M\) is another orientation where
    \(\tau \colon \tilde M \to \tilde M, (x, \mu) \mapsto (x, -\mu)\).

    Then \(M \amalg M \cong \tilde M\) is a homeomorphism and \(p\colon \tilde M \to M\) is the trivial covering.

    So for connected \(M\), the following are equivalent: \(M\) is orientable, \(p\colon \tilde M \to M\) has a continuous section, \(p\) is trivial, i.e. \(\tilde M \cong M\amalg M\).

    An orientable connected manifold has exactly two orientations

    If \(M\) is orientable and has \(n\) path-components, then \(M\) has exactly \(2^n\) orientations.
\end{remark}

\begin{corollary}
    Let \(M\) be a connected  \(m\)-manifold such that for some (hence any) \(x \in M\), the group \(\pi_1(M, x)\) does not have a subgroup of index \(2\). Then \(M\) is orientable.
\end{corollary}

\begin{proof}
    We argue by contradiction. If \(M\) was not orientable, then \(p\colon \tilde M \to M\) is not a product covering. So \(\tilde M\) is path connected, so for every \(\tilde x \in \tilde M\) the homomorphism on fundamental groups \(p_*\colon \pi_1(\tilde M, \tilde x) \to \pi_1(M, p(\tilde x))\) is a monomorphism with image of index \(2\). So \(\img(p_*)\) is an index \(2\) subgroup of \(\pi_1(M,x)\).
\end{proof}

This gives that in particular every simply connected manifold is orientable.

\begin{example}
    The spaces \(S^n\) (for \(n \geq 2\)), \(\IC P^n, \IH P^n\) are orientable manifolds.

    He continues to draw, that \(S^1\) is also orientable.

    Let \(M\) be an \(m\)-manifold, that is also a topological group, i.e. there is a continuous map \(m\colon M \times M \to M\) that is also a group structure on \(M\) and such that \(m \mapsto m^{-1}\) is continuous. Then \(M\) is orientable.

    \begin{proof}
        choose a local orientation \(\mu \in H_m(M \mid 1)\), where \(1 \in M\) is the multiplicative unit. For every \(m \in M\),
        \[m\cdot \_\colon M \to M\]
        is a homoeomorphism that takes \(1\) to \(m\), so \((m\cdot \_)_* \colon H_m(M \mid1) \to H_m(M \mid m)\) is an isomorphism. Set \(\mu_m\coloneq (m\cdot \_)_*(\mu)\). Then \(\set{\mu_m}_{m \in M}\) is an orientation of \(M\).
    \end{proof}

    Examples for this are \(S^1, O(n), U(n), \mathrm{Sp}(n), SO(n), SU(n), \dots\)
\end{example}

\begin{proposition}
    Let \(M\) be any \(n\)-manifold.
    \begin{enumerate}
        \item The manifold \(\tilde M\) is orientable, and the map \(\tau \colon \tilde M \to \tilde M\) given by \(\tau(x, \mu) = (x, -\mu)\) is orientation reversing.
        \item Let \(q \colon N \to M\) be a twofold covering and \(N\) be orientable manifold, \(\tau \colon N \to N\) the free deck-transformation. If \(\tau \colon N \to N\) is orientation reversing, then \(q\colon N \to M\) is isomorphic as a covering to \(p\colon \tilde M \to M\).
    \end{enumerate}
\end{proposition}

\begin{proof}\leavevmode
    \begin{enumerate}
        \item Let \(\tilde x = (x, \mu) \in \tilde M\) be any point in \(\tilde M\). Since \(p\colon \tilde M \to M\) is a local homeomorphism. so \(p\colon H_n(\tilde M, \tilde x) \xrightarrow{\cong} H_n(M, x) \ni \mu\). Set \(\mu_{\tilde x} \coloneq p_*^{-1}(\mu)\). then \(\set{\mu_{\tilde x \in \tilde M}}\) is an orientation of \(\tilde M\).
        
        The map \(\tau \colon \tilde M \to \tilde M\) reverses this orientation.
        \[\tau_*\colon H_n(\tilde M, \tilde x) \to H_n(\tilde M, \tau_*(x))\quad p_*^{-1}(\mu)\mapsto \tau_*(p_*^{-1}(\mu)) = p_*^{-1}(\mu) = - p_*^{-1}(-\mu) = \mu_{\tau(\tilde x)}\]

        \item We have
        \[\begin{tikzcd}
            N \ar[rd, "q"] & & \tilde M \ar[ld, "p"]\\
            & M & \\
        \end{tikzcd}\]
        and we look for a map \(N \to \tilde M\). Define \(f\colon N \to \tilde M\) by \(f(y) = (q(y), q_*(\mu_y))\). We use \(q_*\colon H_n(N\mid y) \xrightarrow{\cong} H_n(M\mid q(y))\). this \(f\) is continuous. We will not check this, \(f\) commutes with the free involution:
        \[f(\tau y) = (q(y), q_*(\mu_{\tau_y})) = (q(y), q_*(-\tau_*(\mu_y))) = (q(y), -q_*(\mu_y)) = \tau(q(y), q_*(\mu_y)) = \tau(f(y))\]
        So \(f\) is a continuous bijection over \(M\), hence a homeomorphism.
    \end{enumerate}
\end{proof}

\subsection{Orientability of \(\IR P^n\)}

We already know \(\IR P ^1\) and \(\IR P^3\) are orientable, as \(\IR P^1 \cong S^1\) and \(\IR P^3 \cong SO(3)\).

\textbf{Recall.} The antipodal map \(A\colon S^n \to S^n\), given by \(x \mapsto -x\) has degree \((-1)^{n+1}\).

Let \(\mu \in H_n(S^n, \IZ)\) be any generator, define an orientation on \(S^n\) by \(x \in S^n : \mu_x \coloneq r_x^{S^n}(\mu) \in H_n(S^n(x))\).

We look at
\[\begin{tikzcd}
    H_n(S^n, \IZ) \ar[d, "A_*"] \ar[r, "\cong", "r_x^{S^n}"'] & H_n(S^n \mid x) \ar[d, "A_*"] \\
    H_n(S^n, \IZ) \ar[r, "\cong"] & H_n(S^n \mid -x) \\
\end{tikzcd}\]

Some of this diagram is missing.

this gives if \(n\) is even, then \(A \colon S^n \to S^n\) is orientation reversing. if \(n\) is odd, then \(A\colon S^n \to S^n\) is orientation preserving:

 So for even \(n\), then \(q\colon S^n \to \IR P^n\) is twofold covering and flip reverses orientation, so this \enquote{is} the orientation covering. As \(S^n \not \cong \IR P^n \amalg \IR P^n\) we have no continuous section to \(S^n \xrightarrow{q} \IR P^n\) and \(\IR P^n\) is not orientable.

 For \(n\) odd we have

 \begin{proposition}
    Let \(f\colon N \to N\) be continuous free involution of a connected oriented \(m\)-manifold. Then
    \begin{enumerate}
        \item \(M \coloneq N/x\sim f(x)\) is an \(m\)-manifold.
        \item If \(N\) is orientable and \(f\) is orientation preserving, then \(M\) is orientable.
    \end{enumerate}
 \end{proposition}

 \begin{proof}
    \begin{enumerate}
        \item We have implicitly already done he's trying to convince me.
        \item Choose an orientation \(\set{\mu_y}_{y \in N}\) of \(N\). We define an orientation of \(M\) as follows: For \(x \in M\) choose \(y \in N\), such that \(p(y) = x\). Then we have
        \[p_*\colon H_n(N\mid y) \xrightarrow{\cong} H_n(M \mid x)\]
        and set \(\mu_x \coloneq q_*(\mu_y)\). This is independent of the choice of \(y\): the other choice is \(f(y)\).

        Some diagram I couldn't copy. As \(f\) is orientation preserving, the chocie does not matter.
    \end{enumerate}
 \end{proof}

 Then \(\IR P^n\) is orientable for \(n\) odd.

\textbf{Next.} We want to show, that for a \(m\)-manifold and all \(n > m\): \(H_n(M,A) = 0\).
As a slogan \enquote{Homology vanishes above the geometric dimension.}

In many examples, \(m\)-manifolds \(M\) have \(m\)-dimensional CW-structure, e.g.
\[S^n, \IR P^n, \IH P^n, \IC P^n\]
We could also produce CW-strucutre on the Grassmannians and Stiefel-Manifolds.

\textbf{Warning.} an \(m\)-manifold need not admit a CW-structure! Smooth manifolds admit triangulations, hence CW-structures. But there are non-smoothable manifolds, that do not admit CW-structures.

\begin{thm}{Vanishing homology in high dimensions}{}
    Let \(M\) be an \(m\)-manifold; let \(K\) be a compact subset of \(M\). Then
    \begin{enumerate}
        \item \(H_i(M, M\setminus K; A) = 0\) for all \(i > n\) and \(A\) any abelian group.
        
        In particular, if \(M\) is compact, then \(H_i(M; A) = 0\) for all \(i > n\) and all \(A\).
        \item A class in \(H_n(M, M\setminus K, A)\) is zero if and only if for all \(x \in K\), its image under \(r_x^K\colon H_n(M, M\setminus K, A) \to H_n(M, M\setminus \set x; A)\) is zero.
        
        In particular, if \(M\) is compact, the maps \(r_x\colon H_n(M,A) \to H_n(M, M \setminus \set x; A)\) are jointly injective.
    \end{enumerate}
\end{thm}

\begin{proof}
    In \(6\) bootstrapping-steps.
    \begin{description}
        \item[Step 1] If \(M = \IR^n\), and \(K \) is a convex compact subset. Choose \(R > 0\) such that \(K\) is contained in the open ball of radius \(R\) around \(y \in K\).
        \[\set{z \in IR^n : \abs{z-y} = R} \eqcolon S^{n-1}_R \subseteq  M\setminus K \subseteq M \setminus \set y\]
        and these are homotopy equivalences.

        So \(H_i(M \mid K) \cong H_i(M \mid x) = 0\) for \(i > n\).

        \item[Step 2] \(K_1, K_2\) two compact subsets of \(M\). Suppose the claim holds for \(K_1, K_2\) and \(K_1 \cap K_2\). Then it also holds for \(K_1 \cup K_2\).
        
        We do this by a Mayer-Vietoris argument for local homology.
        \[M\setminus(K_1 \cap K_2) = (M\setminus K_1) \cup (M\setminus K_2)\]
        And \(M\setminus K_1 \cap M\setminus K_2 = M \setminus (K_1 \cup K_2)\)

        Remembering the theorem of small simplices we get long fractions of chain complexes, which I did not copy. We get a long exact sequence of homology groups
        \[H_{n+1}(M \mid K_1 \cap K_2) \xrightarrow{\partial} H_{n}(M\mid K_1 \cup K_2) \to H_n(M \mid K_1) \oplus H_n(M \mid K_2) \to H_n(M \mid K_1 \cap K_2)\]

        For \(i > n\) we have \(H_i(M \mid K_1 \cup K_2)\) lies between \(H_{i+1}(M \mid K_1 \cap K_2) = 0\) and \(H_i(M \mid K_1) \oplus H_i(M \mid K_2) = 0\).

        For \(i = n\)
        \[0 = H_{n+1}(M\mid K_1 \cap K_2) \to H_n(M \mid K_1 \cup K_2) \hookrightarrow H_n(M \mid K_1) \oplus H_n(M \mid K_2)\]
        Let \(Z \in H_n(M \mid K_1 \cup K_2)\) such that \(r_x^{K_1 \cup K_2}(z) = 0\) for all \(x \in K_1 \cup K_2\).

        \textbf{Claim.} \(r_{K_1}^{K_1 \cup K_2}(z) = 0\). To see this pick \(x \in K_1\). Then
        \[\begin{tikzcd}
            H_n(M \mid K_1 \cup K_2) \ar[r, "r_{K_1}^{K_1 \cup K_2}"] \ar[dr] & H_n(M \mid K_1) \ar[d, "r_X^{K_1}"] \\
            & H_n(M \mid x) \\
        \end{tikzcd}\]

        For all \(x \in K_1\), \(r_X^{K_1}(r_{K_1}^{K_1 \cup K_2}) = 0\) and so \(r_{K_1}^{K_2 \cap K_1} (z) = 0\) because the claim ii) holds for \(K_1\). Similarly \(r_{K_2}^{K_1 \cap K_2}(z) = 0\) and then also \(z = 0\) by the injectivity of
        \[(r_{K_1}^{K_1 \cap K_2}, r_{K_2}^{K_1 \cup K_2})\]

\newLecture{21.05.2025}

\textbf{Interlude.}

\begin{construction}
    Let \(\mu \in H_m(M \mid x)\) be a local orientation of a manifold \(M\) at \(x\). Let \(\nu \in H_n(N\mid y)\) similarly. We construct the relative Künneth isomorphism 
    \[\begin{tikzcd}
        H_m(M \mid x) \otimes H_n(N \mid y) \ar[d, "\cong", phantom, sloped] & H_{m+n}(M \times N \mid (x,y)) \ar[d, "\cong", phantom, sloped] \\
        H_m(M, M \setminus \set x; \IZ) \otimes H_n(N, N \setminus \set x; \IZ) \ar[r, "\times"] & H_{m+n}(M\times N, M \times (N\setminus \set y) \cup (M\setminus \set x)\times N; \IZ) \\
    \end{tikzcd}\]
    \enquote{We can combine local orientations to get some in the product.}
    \(\mu \times \nu\) is a local orientation of \(M \times N\) at \((x,y)\). Because we didn't make any choices, this is probably continuous. Let \(\set{\mu_x}_{x \in X}\) and \(\set{\nu_y}_{y \in Y}\) be orientation of \(M\) and \(N\). Then \(\set{\mu_x \times \nu_y}_{(x,y) \in X \times Y}\) is an orientation of \(M \times N\).
\end{construction}

\textbf{End of interlude.}

    \item[Step 3] Let \(M = \IR^n\), \(K = K_1 \cup \dots, \cup K_m\) for \(K_1, \dots, K_m\) being compact convex subsets of \(M = \IR^n\).
    
    \begin{proof}
        By induction on \(m\). \(m = 1\) is clear by step 1.

        Now let \(K = (K_1 \cup \dots \cup K_{m-1}) \cup K_m\) where we have for both terms the conditions by induction. We need to look at
        \[(K_1 \cup \dots \cup K_{m-1}) \cap K_m = (K_1 \cap K_m) \cup \dots \cup (K_{m-1} \cap K_m)\]
        where we use that the intersection of convex subsets is again convex. Then the union is again fullfilling the conditions by induction.
    \end{proof}

    \item[Step 4] \(M = \IR^n\) and \(K\) any compact subset of \(\IR^n\). We need the following claim.
    
    \textbf{Claim.} Let \(\alpha\in H_i(\IR^n \mid K)\). Then there is a compact neighborhood \(N\) of \(K\)and a class \(\alpha'\in H_i(\IR^n \mid N)\) such that \(\alpha \in r_K^N(\alpha')\).

    \begin{proof}
        Let \(\alpha = [x + C_i(\IR^n \setminus K)]\) for some \(x \in C_i(\IR^n)\) such that \(d_i(x) \in C_{i-1}(\IR^n\setminus K)\). Then
        \[d_i(x) = \sum_{\text{finite}} \alpha_j \cdot (f_j\colon \nabla^{i-1} \to \IR\setminus K)\]
        with \(a_j \in A\) and \(f_j\) continuous. Set \(L = \mathrm{supp}(d_i(x)) \coloneq \bigcup f_j(\nabla^{i-1})\) a compact subset of \(\IR^n\setminus K\). Now \(K,L\) are disjoint subsets of \(\IR^n\). so there is an \(N\), compact neighborhood of \(K\) with \(N \cap L = \emptyset\). You can for example look at \(\dist(L,K) > 0\) and take \(N\) accordingly.

        Then \(d_i(x) \in C_{i-1}(L) \subseteq C_{i-1}(\IR^n \setminus N)\) so \(\alpha' \coloneq [x + C_i(\IR^n \setminus N)] \in H_i(\IR^n \mid N)\) satisfies \(r_K^N(\alpha') = \alpha\).
    \end{proof}

    Now for the proof of step 4: either \(i > n\) or \(i = n\) and \(res_x^K(\alpha) = 0\) for all \(x \in K\).

    Let \(N\) be a compact neighborhood of \(K\) and \(\alpha' \in H_i(\IR^n \mid N)\) with \(r_K^N(\alpha') = \alpha\). Since \(K\) is  compact, it can be covered with finitely many metric open balls in \(N\). Let \(b_1, \dots, B_m\) be the closed metric balls, whlich still lie in \(N\). Now
    \(K \subseteq B_1 \cup \dots \cup B_m \subseteq N\) and \(B_i\) is a convex compact subset, so laim \(i)\) and \(ii)\) holf for \(B_1 \cup \dots \cup B_m\). So
    \[r_{B_1 \cup \dots \cup B_m}^n(\alpha') = 0\]
    if \(i > n\). and
    \[res_X^{B_1\cup \dots \cup B_m}(res_{B_1 \cup \dots \cup B_m}^N(\alpha')) = 0\]
    for \(i = n, x \in B_1 \cup \dots \cup B_m\). And then
    \[r_{B_1 \cup \dots \cup B_m}^N(\alpha') = 0\] by step 3. and then \(\alpha = r_K^N(\alpha') = r_K^{B_1 \cup \dots \cup B_m}(\underbrace{r_{B_1\cup \dots \cup B_m}^N}_{= 0}(\alpha')) = 0\)

    \item[Step 5] \(M\) arbitrary, there is an open neighborhood \(U\) of \(K\) homeomorphic to \(\IR^n\).
    
    Choose a homeomorphism \(\phi\colon U \xrightarrow{\cong} \IR^n\). contemplate the commutative diagram

    \[\begin{tikzcd}
        H_i(M \mid K) & H_i(U \mid K)\ar[d, "r_*^K"]  \ar[l, "\cong", "\text{excision}"'] \ar[r, "\phi_*"', "\cong"] & H_i(\IR^n \mid \phi(K)) \ar[d, "r_{\phi(x)}^{\phi(K)}"] \\
        & H_i(U\mid x) \ar[r, "\cong", "\phi_*"'] & H_i(\IR^n \mid \phi(x)) \\
    \end{tikzcd}\]

    \item[Step 6] General case.
    
    \textbf{Claim.} There are compact subsets \(K_1, \dots, K_m\) of \(M\) such that
    \begin{itemize}
        \item \(K_i\) is contained in an open subset homeomorphic to \(\IR^m\).
        \item \(K = K_1 \cup \dots \cup K_m\).
    \end{itemize}

    When we proove this claim, we are done by step 2.

    \begin{proof}[Proof of claim.]
        Each \(x \in K\) has an open neighborhood homeomorphic to \(\IR^n\). So there is a compact neighborhood \(N_x\) of \(x\) such that \(x \in N_x \subseteq U \cong \IR^n\). \(K \subseteq \bigcup_{x \in K} N_x\) so by compactness \(K = N_{x_1} \cup \dots \cup N_{x_m}\).
    \end{proof}
    \end{description}
\end{proof}

\section{The fundamental class}

We know \(H_n(S^n; \IZ) \cong \IZ\), \(H_{k+l}(S^k \times S^l; \IZ) \cong \IZ, \; H_n(S^n \amalg S^n; \IZ) \cong \IZ \oplus \IZ\) annd \(H_{2n}(\IC P^n; \IZ) \cong \IZ\)
, \(H_{4n}(\IH P^n; \IZ) \cong \IZ\)
\[H_n(\IR P^n, \IZ) \cong \begin{cases}
    \IZ & n \text{ odd} \\
    0 & n \text{ even.}
\end{cases}\]
\(H_m(\IR^n; \IZ) = 0\) for \(n \geq 1\).

The pattern should be for connected manifolds, their top homology is \(\IZ\), if \(M\) is compact and orientable.

We want to show:
\begin{proposition}
    Let \(M\) be a compact connected \(n\)-manifold. Then
    \[H_n(M; \IZ) \cong \begin{cases}
        \IZ & \text{ if } M \text{ is orientable} \\
        0 & M \text{ is not orientable.}
    \end{cases}\]
\end{proposition}

For that we first study fundamental classes.
\begin{thm}{Fundamental Classes}{}
    Let \(M\) be an oriented \(n\)-manifold and \(K\) a compact subset of \(M\). then there is a unique class \(\mu_K\in H_n(M, M \setminus K; \IZ) \cong H_n(M \mid K)\) such that
    \[r_x^K(\mu_K) = \mu_x \in H_n(M \mid x)\]
    for all \(x \in K\).

    The class \(\mu_K\) is the relative fundamental class \index{fundamental class}.
\end{thm}

\textbf{Special Case.} Let \(M\) be a compact oriented \(m\)-manifold. Then there is a unique class \([M] \in H_n(M; \IZ)\) such that \(r_x^K([M]) = \mu_x\) for all \(x \in M\). Then \([M]\) is the fundamental class.

If in addition \(M\) is connected, then \([M]\) generates \(H_n(M, \IZ) \cong \IZ\).

\begin{proof}
    Uniqueness is already done. We make a construction in three steps.
    \begin{description}
        \item[Step 1] Suppose \(K\) is contained in an open subset \(U\) homeomorphic to \(\IR^n\). Then \(K\) is contained in a local ball \(B\) in the sense of the definition of the orientation covering. We consider the commutative diagram:
        For \(x, y \in K\):
        \[\begin{tikzcd}
            &H_n(M \mid B) \ar[d, "r_K^B"]\ar[ldd, bend right, "r_x^B \; \cong"] \ar[ddr, bend left, "r_y^B \; \cong"] & \\
            &H_n(M \mid K) \ar[ld, "r_x^K"] \ar[dr, "r_y^K"] & \\
            H_N(M \mid x)&  & H_n(M \mid y) \\
        \end{tikzcd}\]
        Then \(r_K^B(\mu_B)\) has the desired property.

        \item[Step 2] Suppose that \(K = K_1 \cup K_2\), \(K_1, K_2\) compact , claim true for \(K_1\) and \(K_2\). Let \(\mu_{K_1} \in H_n(M \mid K_1)\) and \(\mu_{K_2} \in H_n(M \mid K_2)\) be the relative fundamental classes. We showed in the proof of Step 2 of the previous result that the following is exact:
        \[0 \to H_n(M \mid K) \xrightarrow{(r_{K_1}^K, r_{K_2}^K)} H_n(M \mid K_1) \oplus H_n(M \mid K_2) \to H_n(M \mid K_1 \cap K_2) \to \dots\]
        so we see \((\mu_{K_1}, \mu_{K_2}) \mapsto r_{K_1 \cap K_2}^{K_1}(\mu_{K_1}) - r_{K_1 \cap K_2}^{K_2}(\mu_{K_2})\), where both terms are equal to \(\mu_{K_1 \cap K_2}\), so their difference is \(0\). Any \(x \in K\) is contained in \(K_1 \cap K_2\). If \(x \in K_1\), then
        \[r_x^K(\mu_K) = r_x^{K_1}(r_{K_1}^K(\mu_K)) = r_x^{K_1}(\mu_{K_1}) = \mu_x\]

        \item[Step 3] General case. As in step 6 of the previous proof, we can write \(K = K_1 \cup \dots \cup K_m\) where all \(K_i\) are compact and all \(K_i\) are contained in an open subset homeomorphic to \(\IR^n\). Then \(\mu_{K_i}\) exists for all \(i = 1, \dots, m\) by step 1. By induction on \(m\) and Step 2 it holds for \(K = K_1 \cup \dots \cup K_m\).
    \end{description}
\end{proof}

We draw some corollaries:

\begin{thm}{Top Homology of connected Manifolds}{}
    Let \(M\) be a connected, compact, oriented \(n\)-manifold. Then \(H_n(M; \IZ) \cong \IZ\), generated by \([M]\).

    Moreover, for all \(x \in M\), the restriction \(r_x^M\colon H_n(M, \IZ) \to H_n(M \mid x)\) is an isomorphism.
\end{thm}

\begin{proof}
    \textbf{Claim.} Let \(\alpha\in H_n(M, \IZ)\). Then the set \(x \in \set{M \mid r_x^M(\alpha) = 0 \text{ in } H_n(M \mid x)}\) is open and closed in \(M\).

    Let \(x \in M\) be such that \(r_x^M(\alpha) = 0\) (or \(r_x^M(\alpha) \neq 0\)). Let \(B\) be a local ball in \(M\) containing \(x\). We get a commutative diagram:
    For \(x,y \in B\)
    \[\begin{tikzcd}
        & H_n(M) \ar[d, "r_B^M"] & \\
        & H_n(M \mid B) \ar[dl, "r_x^B", "\cong"'] \ar[rd, "r_y^B", "\cong"'] & \\
        H_n(M \mid x) && H_n(M \mid y) \\
    \end{tikzcd}\]
    For some reason, this shows the claim.

    Then \(r_x^M \colon H_m(M; \IZ) \to H_n(M \mid x) \cong \IZ\) is surjective, because \(r_x^M([M]) = \mu_x\) generates \(H_n(M \mid x)\). If \(\alpha \in H_n(M, \IZ)\) is such that \(r_x^M(\alpha) = 0\), then \(\set{y \in M: r_y^M(\alpha) = 0}\) is open, closed and nonempty. Since \(M\) is connected, this set is all of \(M\), so \(\alpha = 0\) by the detection property (ii) in the previous theorem.
\end{proof}

\begin{corollary}
    Let \(M\) be a compact connected and non-orientable \(n\)-manifold. Then
    \(H_n(M; \IZ) = 0\).
\end{corollary}

\begin{proof}
    Let \(\alpha \in H_n(M, \IZ)\). As in the previous proof,
    \(\set{x \in M : r_x^M(\alpha) = 0}\) is an open and closed subset. Also \(r_x^M\colon H_n(M, \IZ) \to H_n(M \mid x)\) is injective for all \(x \in M\). Hence \(H_n(M; \IZ) \hookrightarrow H_n(M \mid x) \cong \IZ\), so \(H_n(M; \IZ)\) is torsion free.

    Let \(p\colon \tilde M \to M\) be the orientation covering. Since \(M\) is connected and not orientable, \(\tilde M\) is connected, compact and orientable. Let \(\set{\mu_x}_{x \in \tilde M}\) be an orientation of \(\tilde M\). Then
    \[H_n(\tilde M; \IZ) \cong \IZ\]
    generated by \([\tilde M]\). Let \(\tau\colon \tilde M \to \tilde M\) be the nonidentity deck transformation. Then \(\tau\) is orientation reversing, \(\tau_*[\tilde M] = - [\tilde M]\) (A diagram I didn't copy as proof).

    So \(p_*[\tilde M] = p_*(\tau_*([\tilde M])) = p_*(- [\tilde M]) = - p_* [\tilde M]\) so \(2 \cdot p_*[\tilde M] = 0\). so by torsion-freeness \(p_*([\tilde M]) = 0\).

    As \([\tilde M]\) generates \(H_n(\tilde M, \IZ)\), \(p_* = 0\colon H_n(\tilde M; \IZ) \to H_n(M; \IZ)\).

    Let \(\mathrm{tr}\colon H_n(M, \IZ) \to H_n(\tilde M; \IZ)\) be the transfer map. Then \(0 = p_* \circ \mathrm{tr} = \text{ multiplication by 2 on } H_n(M; \IZ) \) so \(H_n(M, \IZ) = 0\).
\end{proof}

\newLecture{26.05.2025}

\breakline{Was unfortunately unable to attend the lecture.}

\newLecture{28.05.2025}

\chapter{Cap Product}

Due to missing the last lecture I intented to copy the repetition Schwede does at the beginning of his lecture. However, public transport prevented me from copying the start of the Repetition. The most important definition of last lecture was the Cap-Product:

It will later be used for Poíncare-Duality.

\begin{defi}{Cap-Product}{}
    Let \(X\) be a simplicial set, \(Y \subseteq X\) simplicial subset, \(R\) any commutative ring. For \(0 \leq i \leq n\) we define
    \[\cap \colon C_n(X, Y ; R) \times C^{i}(X,Y, R) \to C_{n-i}(X;R)\]
    given by
    \[x \in C_n(X,Y;R) = \frac{R[X_n]}{R[Y_n]}, f \in C^{i}(X,Y;R), \text{i.e. } f \colon X_i \to R, f(Y_i) = \set{0}\]
    and
    \[x \cap f = \underbrace{f(x[0,i])}_{\in R} \cdot x[i, n]\]
    and then i missed something due to broken board
\end{defi}

\begin{proposition}
    Let \(Y \subseteq X\) be simplicial sets, \(R\) some commutative ring. Then the following hold:
    \begin{enumerate}
        \item \(d(x \cap f) = (-1^{i}) \cdot (dx \cap f - x \cap df)\) for \(x \in C_n(X,Y;R), f \in C^{i}(X,Y;R)\)
        \item The cap product descends to a well defined and \(R\)-bilinear map
        \[\cap \colon H_n(X,Y;R) \times H^{i}(X,Y;R) \to H_{n-1}(X,R), \quad [x] \cap [f] \coloneq [x \cap f]\]
        \item If \(Y = \emptyset\) \(\xi \in H_n(X,R)\), \(\alpha \in H^{i}(X,R), \beta \in H^{j}(X,R)\) we have
        \[(\xi \cap \alpha) \cap \beta = \xi \cap (\alpha\cup \beta)\]
        \item if \(Y = \emptyset\), then \(\xi \cap 1 = \xi\).
        \item Let \(\Psi \colon X \to X'\) bee a morphism of simplicial sets, s.t. \(\Psi(Y) \subseteq Y'\), \(\xi \in H_n(X,Y;R), \alpha \in H^{i}(X',Y', R)\). Then
        \[\Psi_*(\xi) \cap \alpha = \Psi_*(\xi \cap \Psi^*(\alpha))\]
    \end{enumerate}
\end{proposition}

\begin{proof}
    \begin{enumerate}
        \item Was done last time.
        \item Suppose \(dx = 0\), \(df = 0\), then
        \[d(x \cap f) = (-1)^{i}(\underset{= 0}{(dx)} \cap f - x \cap \underset{= 0}{df}) = 0\]
        so \(x \cap f\) is a cycle. For \(y \in C_{n+1}(X,Y;R)\)
        \[(x+dy) \cap f = x \cap f \pm d(y\cap f)\]
        which gives \([(x+dy) \cap f] = [x \cap f]\) and similarly \([x \cap (f+ dg)] = [x\cap f]\) for \(g \in C^{i-1}(X,Y; R)\).
        \item Let \(x \in C_n(X;R)\) represent \(\xi \in H_n(X,R)\). Let \(a\colon X_i \to R, b\colon X_j \to R\) represent \(\alpha\in H^{i}(X,R)\) and \(\beta \in H^j(X,R)\). Then
        \[\begin{split}
            (x\cap a) \cap b &= (a(x[0,i]) \circ x[i, n]) \cap b\\
            &= a(x[0,i])\cdot (x[i,n] \cap b) \\
            &= a(x[0,1]) \cdot b(x[i,n][0,j]) \cdot x[i,n][j,n-1] \\
            &= a(x[0,i]) \cdot b(x[i, i+j]) \cdot x[i+j, n] \\
            &= a(x[0, i+j][0,i]) \cdot b(x[0, i+j][i, i+j]) \cdot x[i+j,n] \\
            &= (a\cup b)(x[0, i+j]) \cdot x[i+j,n] \\
            &= x \cap (a\cup b) \\
        \end{split}\]
        \item For some reason clear.

        \item We let \(x \in X_n\) be an \(n\)-simplex, \(f \in C^{i}(X', Y'; R)\) representing a chain in \(H^{i}(X', Y'; R)\).
        \[\begin{split}
            \Psi_*(x) \cap f &= \Psi_n(x) \cap f \\
            &= f((\Psi_n(x))[0, i]) \cdot (\psi_n(x))[i,n] \\
            &= \underset{\in R}{f(\psi_i(x[0,i]))} \cdot \Psi_[n-i](x[i,n]) \\
            &= \Psi_*(f(\Psi_i(x[0,i])) \cdot x[i,n]) \\
            &= \Psi_*(x \cap \Psi^*(f))
        \end{split}\]
    \end{enumerate}
\end{proof}

\chapter{Poincaré Duality}

\section{Cohomology with compact support}

We are working towards:

\begin{thm}{Poíncare-duality}{}
    Let \(M\) be a compact \(n\)-manifold, \(i \geq 0\).
    \begin{itemize}
        \item If \(M\) is oriented, then \([M] \cap \_ \colon H^{i}(M; \IZ) \to H_{n-i}(M; \IZ)\) is an isomorphism.
        \item The map \(\nu_M \cap \_ \colon H^{i}(M; \IF_2) \to H_{n-i}(M; \IF_2)\) is an isomorphism.
    \end{itemize}
\end{thm}

Our idea is to proof this by
\begin{itemize}
    \item Proof for \(\IR^n\)
    \item Patching/Mayer-Vietoris argument \(M = U_1 \cap U_2\).
\end{itemize}

However, we have a problem: \(M = \IR\) is an oriented \(1\)-manifold. \(H^{1}(R; \IZ) \cong 0\), but \(H_0(\IR; \IZ) \cong \IZ\), which is not the same.

To solve this, we introduce Compactly supported cohomology as a variation of singular cohomology that has Poincaré duality for not necessarily compact manifolds: We will get
\[H_{comp}^{i}(M; \IZ) \xrightarrow{\cong} H_{n-i}(M; \IZ)\]
for compact \(M\).

\begin{construction}
    Let \(X\) be a topological space, \(A\) an abelian group. A singular cochain \(f \in C^n(\cS(X),A)\)
    \[f \colon \cS(X)_n = \mathrm{maps}^{\text{cont}}(\nabla^n; X) \to A\]
    is supported in a subset \(K\) of \(X\) if \(f(\phi) = 0\) for all \(\phi\colon \nabla^n \to X\) with \(\phi(\nabla^n) \subseteq X \setminus K\). Equivalently \(f\) belongs to the kernel of the homomorphism
    \[C^n(\cS(X);A) \to C^n (\cS(X\setminus K); A)\]

    \(f \in C^n(\cS(X);A) \) is \emph{compactly supported}\index{compactly supported} if there is a compact subset \(K\) of \(X\) such that \(f\) is supported on \(K\).
\end{construction}

\begin{proposition}
    Compactly supported cochains form a subcomplex of \(C^*(\cS(X); A)\).
\end{proposition}
\begin{proof}
    Suppose \(f\colon \cS(X)_n \to A\) is supported on \(K\) with \(K\) compact. Then
    \[(df)(\Psi) = \sum_{i = 0, \dots, n+1} (-1)^{i} f(d_i^*(\Psi)) = \sum_{i = 0, \dots, n+1} (-1)^{i} f(\Psi \circ (d_i)_*)\]
    and then he outspeeded me.
\end{proof}

\begin{example}
    There are no continuous maps \(\phi \colon \nabla^n \to X\) with image in \(X \setminus X\), so every cochain \(f\in C^n(\cS(X), A)\) is supported on \(X\).

    So if \(X\) is compact, then every cochain is compactly supported, hence \(C_{comp}^*(X,A) = C^*(X,A)\).
\end{example}

\begin{example}
    Note: \(\set 0\) is a compact subset of \(\IR^n\). So
    \[\underbrace{C^*(\cS(X), \cS(X\setminus \set 0); A)}_{\text{Cochains supported on } \set 0} \subseteq C^*_{comp}(\cS(X),A)\]

    \begin{proposition}
        For all \(n \geq 1\), all abelian groups \(A\), the inclusion
        \[C^*(\cS(\IR^n); \cS(\IR^n\setminus \set 0);A) \to C_{comp}^*(S(\IR^n;A))\]
        is a quasi-isomorphism. In paricular:
        \[H_{comp}^{i}(\IR^n; A) = H^{i}(\IR^n; \IR^n\setminus \set 0; A) \cong \begin{cases}
            A & \text{if } i = n \\
            0 & \text{ else} \\
        \end{cases}\]
    \end{proposition}

    \begin{proof}
        By the five lemma, it suffices to show that the quotient complex
        \[\frac{C_{comp}^*(\cS(\IR^n))}{C^*(\cS(\IR^n), \cS(\IR^n\setminus \set 0))}\]
        is acyclic. Let \(f \in C^{i}_{comp}(\cS(X))\) be a compactly supported cochain  class in the quotient complex in a cocycle, i.e. \(df \in C^{i+1}(\cS(X), \cS(X \setminus 0))\). Then ?? of \(f\) is supported on \(\set 0\).

        Since \(f\) is compactly supported and every compact subset of \(\IR^n\) is contained in a sufficiently large ball, there is a \(r > 0\) s.t. \(f\) is supported on \(D_r^n = \set{x \in \IR^n : \abs x \leq r}\) the disc of radius \(r\) around \(0\). Write \(\mathrm{res}(f)\) for the restriction of \(f\) to \(C^*(\cS(X), \cS(X \setminus D_r^n))\). The inclusion
        \[\IR^n \setminus D_r^n \to \IR^n \setminus \set 0\]
        is a homotopy equivalence, so the cohomology of the pair \(\IR^n\setminus \set 0; \IR^n \setminus D_r^n\) is tirivial. Equivalently, the residue cochain complex
        \[C^*(\cS(\IR^n \setminus 0), \cS(\IR^n \setminus D_r^n))\]
        is trivial. So the cycle \(\mathrm{res}(f)\) in the acyclic complex is a coboundary. So there exists \(g \in C^{i-1}(\cS(\IR^n \setminus 0), \cS(\IR^n \setminus D_r^n))\) s.t. \(dg = \mathrm{res}(f)\). We extend \(g\) to a cochain
        \[\tilde g \colon \cS(X)_{i-1} \to A\] by zero, i.e. 
        \[\tilde g(\phi) = \begin{cases}
            g(\phi) & \text{if } \phi(\nabla^{i-1}) \subseteq \IR^n \setminus 0 \\
            0 & \text{else} \\ 
        \end{cases}\]
        Then \(\mathrm{res}(\tilde g) = g\). In particular, \(\tilde g\) is supported on \(D_r^n\) because \(g\) is.

        Then
        \[\mathrm{res}(d\tilde g) = d(\mathrm{res}(\tilde g)) = dg \mathrm{res}(f)\]
        so \(\mathrm{res}(f - d\tilde g) = 0\) in \(C^i(\cS(\IR^n \setminus 0), \cS(\IR^n \setminus D_r^n))\). so \(f - d\tilde g\) is supported on \(0\), so ?? to \(C^*(\cS(\IR^n); \cS(\IR^n \setminus 0))\).
        Since \([f] = [f - d \tilde g] = 0\). 
    \end{proof}

    He talks about how this could also be done using some category theory.
\end{example}

\textbf{Note.} We can take \(\IR^n \to \IR^{n+1} \to \IR^n\) taking inclusion and projection. So we get \(\IZ \cong H_{comp}^n(\IR^n; \IZ)\) is a retract of \(H_{comp}^n(\IR^{n+1}; \IZ) \cong 0\). That doesn't make sense and gives the following

\textbf{Warning!} \(H_{comp}^*\) is \underline{not} functorial for arbitrary continuous maps!

Compactly supported cohomology is
\begin{itemize}
    \item contravariantly functorial in \emph{proper} continuous maps
    \item covariantly functorial in open embeddings.
\end{itemize}

\begin{defi}{Proper maps}{}
    A continuous map \(f\colon X \to Y\) is \emph{proper} if for every compact subset \(K\) of \(Y\), the set \(f^{-1}(K)\) is compact with the subspace topology of \(X\).
\end{defi}

\begin{example}
    Let \(X\) be a space. Then \(X \to \set x\) is proper iff \(X\) is compact.

    If \(K\) is compact and \(X\) is any space, then the projection \(X \times K \xrightarrow{p} X\) is proper: For \(L \subseteq X\) compact, \(p^{-1}(L) = L \times K\) is compact.
\end{example}

\begin{proposition}
    Let \(\psi\colon X \to Y\) be a proper continuous map. Then the cochain map
    \[\psi^*\colon C^*(\cS(Y),A) \to C^*(\cS(X); A)\]
    takes compactly supported cochains to compactly supported cochains, so it restricts to a chain map
    \[\psi^*\colon C_{comp}^*(\cS(Y),A) \to C_{comp}^*(\cS(X),A)\]

    This restriction induces group homomorphisms \(\psi^*\colon H_{comp}^i(Y;A) \to H_{comp}^i(X,A)\)
\end{proposition}

\begin{proof}
    We let \(f\colon \cS(Y)_n \to A\) be a simplicial cochain that is supported on the compact subset \(K\) of \(Y\). then \(\psi^{-1}(K)\) is compact because \(f\) is proper. 

    \textbf{Claim.} \(\psi^*(f)\) is  supported on \(\psi^{-1}(K)\).

    Let \(\phi\colon \nabla^n \to X\) be continuous with image in \(X \setminus \psi^{-1}(K)\). Then
    \[(\psi^*(f))(\phi) = f(\psi \circ \phi) = 0\]
    because \(\psi \circ \phi\) has image in \(Y \setminus K\).
\end{proof}


\newLecture{02.06.2025}

So \(f\) from the last proposition induces a homomorphism
\[f^*\colon H^i_{comp}(Y,A) \to H_{comp}^i(X,A)\]
and we get a contravariant functor in proper maps.

Today we will talk about

\subsection{Covariant functoriality for open embeddings}

This comes formally from understanding that compactly supported Cohomology is a colimit of some relative Cohomologies.

Let \(K\) be a compact subset of some space \(X\). Then
\[C^*(\cS(X), \cS(X\setminus K)) \subseteq C_{comp}^*(X,A)\]
We write
\[\lambda_K\colon H^i(X,X\setminus K; A) \to H_{comp}^i(X,A)\]
for the induced map in cohomology.

For \(K \subseteq L \subseteq X\), \(L\) also compact, then \(X\setminus K \supseteq X \setminus L\) so
\[C^*(\cS(X), \cS(X \setminus K), A) \hookrightarrow C^*(\cS(X), \cS(X\setminus K); A) \subseteq C^*_{comp}(X,A)\]
giving a commutative triangle of cohomology groups:
\[\begin{tikzcd}
    {H^i(X, X\setminus K; A)} \ar[rrd, "\lambda_K"] \ar[dd, "\incl^*"] & & \\
    & & {H_{comp}^i(X;A)} \\
    {H^i(X,X\setminus L; A)} \ar[rru, "\lambda_L"] & & \\
\end{tikzcd}\]

\begin{proposition}
    Let \(X\) be a Hausdorff space, \(A\) and \(B\) two abelian groups. Let \(\alpha_K\colon H^i(X, X\setminus K;A ) \to B\) be a homomorphism, for all compact subsets \(K\) of \(X\), such that for all \(K \subseteq L \subseteq X\), with \(L\) compact, the following commutes:
    \[\begin{tikzcd}
        {H^i(X, X\setminus K; A)} \ar[rrd, "\alpha_K"] \ar[dd, "j_L^K"] & & \\
    & & B \\
    {H^i(X, X \setminus L; A)} \ar[rru, "\alpha_L"] & & \\
    \end{tikzcd}\]

    Then there is a unique homomorphism \(\alpha\colon H^i_{comp}(X,A) \to B\) such that \(\alpha \circ \lambda_K = \alpha_K\) for all \(K \subseteq X\) compact.\footnote{The morphisms \(\set{\lambda_K}_K\) express \(H_{comp}^i(X,A)\) as a colimit of the groups \(H^i(X, X\setminus K; A)\)}
\end{proposition}

\begin{proof}
    We drop the coefficients \(A\) from the notation (but only when he remembers he did this). Furthermore we set for this proof \(n = i\).
    \begin{description}
        \item[Uniqueness] We let \(\alpha\colon H^i_{comp}(X) \to B\) be a homomorphism such that \(\alpha\circ \lambda_K = \alpha_K\) for all \(K\) compacct in \(X\).
        
        Let \(f \in C_{comp}^n(X,A)\) be a compactly supported cochain. Let \(f\) be supported on the compact subset \(K\). Then \(f \in C^n(X, X\setminus K)\), and \(\lambda_K[f] = [f]\). Hence
        \[\alpha[f] = \alpha(\lambda_K[f]) = \alpha_K[f]\]

        \item[Existence/Construction] Given \(f \in C^n_{comp}(X,A)\) let \(K\) be a compact subset on which \(f\) is supported.
        
        \textbf{Claim.} \(\alpha[f] \coloneq \alpha_K[f]\) is independent of the choice of \(K\).

        Let \(K'\) be another compact subset of \(X\) on which \(f\) is supported. Let \(L = K \cup K'\), another compact subset of \(X\). We consider
        \[\alpha_K[f] = \alpha_L(j_K^L[f]) = \alpha_L[f] = \alpha_L(j_{K'}^L[f]) = \alpha_{K'}[f]\]

        \textbf{Claim.} \(\alpha_K[f]\) only depends on the cohomology class of \(f\).

        Let \(g \in C_{comp}^{n-1}(X,A)\) be any cochain. Then \(g\) is suppored on some compact subset \(K'\) of \(X\). Then \(f\) and \(g\) are supported on \(L = K \cup K'\). So \(\alpha[f] = \alpha_K[f] = \alpha_L[f + dg] = \alpha[f+dg]\), so \(\alpha\colon H_{comp}^i(X,A) \to B\) is a well defined map and \(\alpha\circ \lambda_K = \alpha_K\) for all \(K\) compact.

        We still need to show, that \(\alpha\) is a group homomorphism: Let \(f, f' \in C_{comp}^n(X,A)\). Let \(f \) be supporetd on \(K\), \(f'\) supported on \(K'\), for \(K, K'\) compact in \(X\). Then \(f\) and \(f'\) are both supported on \(L = K \cup K'\), and
        \[\alpha[f + f'] = \alpha_L[f+f'] = \alpha_L[f] + \alpha_L[f'] = \alpha[f] + \alpha[f']\]
    \end{description}
\end{proof}

\begin{excursion}[filtered colimits.]
    \begin{Definition}
        A PoSet (partially ordered set) is a pair \((P, \leq)\) consisting of a set \(P\) and a relation \(\leq\), that is
        \begin{itemize}
            \item reflexve
            \item transitive
            \item antisymmetric, i.e. if \(x \leq y\) and \(y \leq x\), then \(x = y\).
        \end{itemize}

        A poset \((P, \leq)\) is \emph{filtered} if for all \(x,y \in P\), there is \(z \in P\) s.t. \(x \leq z, y \leq z\).
    \end{Definition}

    Some examples are
    \begin{itemize}
        \item \((\IN, \leq), (\IR, \leq)\)
        \item the set of finite subsets of a given set, under \(\subseteq\).
        \item The set of finite dimension vector subspaces of a vector space under \(\subseteq\)
        \item the set of compact subsets \(\cC(X)\) in a Hausdorff space under \(\subseteq\).
    \end{itemize}

    Every poset \(P, \subseteq\) gives rise to a category \(\underline{P}\) with
    \begin{itemize}
        \item \(\mathrm{Ob}(\underline P) = P\)
        \item \(\underline{P}(x,y) = \begin{cases}
            \set{(x,y)} & \text{if } x \leq y \\
            \emptyset & \text{else}
        \end{cases}\)
    \end{itemize}
    Composition is then uniqueily defined.

    A filtered colimit is a colimit over a filtered poset.
    \begin{example}
        Let \(X\) be a Hausdorff space, \(\cC(X)\) the poset of its compact subsets. A functor \(\cC(X) \to \mathbf{Ab}\) is given by \(K \mapsto H^i(X, X\setminus K; A)\) and for \(K \subseteq L\), we get \(j_L^K\).

        We can now say
        \[C_{comp}^i(X,A) = \colim_{\cC(X)} C^*(X, X\setminus K; A)\]
    \end{example}

    \begin{Proposition}[filtered colimits are exact]
        Let \((P, \leq)\) be a filtered poset, \(F,G,H \colon \underline{P} \to \mathbf{Ab}\) functors, \(\phi\colon F \to G\) and \(\psi\colon G \to H\) natural transformations of functors. Suppose that for all \(x \in P\),
        \[0 \to F(x) \xrightarrow{\phi(x)} G(x) \xrightarrow{\psi(x)} H(x) \to 0\]
        is exact. Then the sequence
        \[0 \to \colim_{\underline P}(F) \xrightarrow{\colim_{\underline P} \phi} \colim_{\underline P} G \xrightarrow{\colim_P \psi} \colim_{\underline P} H \to 0\]
        is exact.
    \end{Proposition}

    This might become an exercise.

    \begin{corollary}
        Let \((P, \leq)\) be a filtered coposet, \(F\colon \underline P \to \mathbf{Chains}\) a functor. Then the canonical morphism
        \[\colim_P(H_n \circ F) \to H_n(\colim_P(F))\]
        is an isomoprhism.
    \end{corollary}

    \begin{corollary}
        \[H^i_{comp}(X,A) = H^i(\colim_{\cC(X)} C^*(X, X\setminus K; A)) \xleftarrow{\cong} \colim_{\cC(X)} H^i(C^*(X,X\setminus K; A)) = \colim_{\cC(X)} H^i(X, X\setminus K;A)\]
    \end{corollary}

    \begin{Definition}
        Let \((P \leq)\) be a poset. Let \(Q \subseteq P\) a subset. \(Q\) is \emph{cofinal} in \(P\), if for all \(x \in P\), there is \(y \in Q\) such that \(x \leq y\).
    \end{Definition}

    If \(P\) is filtered, \(Q\) cofinal in \(P\), then \(Q\) is also filtered.

    \begin{Proposition}
        Let \(Q\) be a cofinal subset of a filtered Poset \(P\). Let \(F\colon \underline P \to \cC\) be a functor to any category that has a colimit. Then \(\colim_Q F = \colim_P F\).
    \end{Proposition}
    Proof might be another exercise.

    \begin{example}
        \(\cC(\IR^n) \subseteq \set{D_r^n : r\geq 0} \subseteq {D_r^n : r \in \IN}\) are cofinal, where \(D_r^n\) again denotes the ball of radius \(r\) around \(0\). This gives

        \[H_{comp}^i(\IR^n; A) = \colim_{K \in \cC(X)} H^i(X, X \setminus K; A) \underset{\text{cofinal}}\cong \colim_{r \in (\IN, \leq)} H^i(\IR^n, \IR^n \setminus D_r^n; A)\]
        \[= \colim(H^i(\IR^n, \IR^n\setminus\set 0) \xrightarrow{\cong} H^i(\IR^n; \IR^n\setminus D_1^n) \xrightarrow{\cong} \dots \xrightarrow{\cong} H^i(\IR^n, \IR^n\setminus D_k^n) \xrightarrow{\cong} \dots)\]
        which is by category theory already isomorphic to \(H^i(\IR, \IR^n\setminus 0)\). This roughly represents the proof we did last week in more detail.
    \end{example}
\end{excursion}

\begin{construction}
    Let \(Y\) be a open subset of a Hausdorff space \(X\). We'll define a homomorphism
    \[\iota_Y^X\colon H_{comp}^i(Y;A) \to H_{comp}^i(X;A)\]
    Let \(K\) be a compact subset of \(Y\). We define
    \[\alpha_K\colon H^i(Y, Y\setminus K;A) \to H_{comp}^i(X,A)\]
    by looking at \(H^i(Y, Y\setminus K; A) \xleftarrow[\text{excision}]{\cong} H^i(X, X\setminus K; A)\) is an isomorphism and we can invert it. We use \((X,Y,K)\) is excisive. We then compose with \(\lambda_K \colon H^i(X, X\setminus K; A) \xrightarrow{\lambda_K} H^i_{comp}(X,A)\).

    The maps \(\alpha_K\) are compatible for \(K \subseteq L \subseteq Y\) all compact.

    \[\begin{tikzcd}
    H^i(Y,Y \setminus K; A) \ar[dd, "j_L^K"] \ar[rrd, bend left, "\alpha_K"] & 
    H^i(X, X \setminus K; A) \ar[l, "\cong"'] \ar[dd, "j_L^K"] \ar[rd, "\lambda_K"] & 
    \\
    & & H_{comp}^i(X;A) \\
    H^i(Y, Y \setminus L; A) \ar[rru, bend right, "\alpha_L"'] & 
    H^i(X, X \setminus L; A) \ar[l, "\cong"] \ar[ru, "\lambda_L"'] & 
    \end{tikzcd}\]

    The universal property of \(H^i_{comp}(Y,A)\) as a colimit yields a unique homomorphism \(\iota_Y^X\) that admits a commutative square I couldn't copy for all \(K \subseteq Y\) and \(K\) compact.
\end{construction}

\begin{proposition}
    Let \(X\) be a Hausdorff space.
    \begin{enumerate}
        \item The homomorphism \(\iota_X^X\colon H_{comp}^i(X,A) \to H_{comp}^i(X,A)\) is the identity.
        \item If \(Z \subseteq Y \subseteq X\) with \(Z\) and \(Y\) open, then
        \[\iota_Y^X \circ \iota_Z^Y = \iota_Z^X \colon H_{comp}^i(Z,A) \to H_{comp}^i(X,A)\]
    \end{enumerate}
\end{proposition}

\begin{proof}
    We let \(K\) be any compact subset of \(Z\).

    \[\begin{tikzcd}
        {H^i(X, X\setminus K;A)} \ar[r, "\cong", "\text{excision}"'] \ar[d, "\lambda_K"] \ar[rr, bend left, "\cong", "\text{excision}"'] & {H^i(Y, Y \setminus K, A)} \ar[r, "\cong", "\text{excision}"'] \ar[d, "\lambda_K"] & {H^i(Z, Z\setminus K; A)} \ar[d, "\lambda_K"] \\
        {H_{comp}^i(X,A)} & {H_{comp}^i(Y;A)} \ar[l, "\iota_Y^X"] & {H_{comp}^i(Z;A)} \ar[l, "\iota_Z^Y"] \ar[ll, bend left, "\iota_Z^X"] \\
    \end{tikzcd}\]

    Something about precomposing with any \(H^i(Z, Z\setminus K; A)\) is enough.
\end{proof}

The maps \(\iota_Y^X\) form a covariant fun ctor from the poset op open subsets of \(X\).

\newLecture{04.06.2025}

\section{The duality map}

For compact oriented manifolds \(M\), Poincaré duality says that
\[[M] \cap \_ \colon H^i(M; \IZ) \to H_{n-i}(M;\IZ)\]
is an isomorphism.

If \(M\) is oriented, but not necessarily compact, this is generalized by the duality map\index{duality map}, for which we will write
\[D_M\colon H^i_{comp}(M;\IZ) \to H_{n-i}(M; \IZ)\]
This will be an isomorphism.

\begin{construction}
    Let \((M, \mu)\) be an oriented \(n\)-manifold. \(\mu = \set{\mu_x}_{x \in M}\) an orientation; earlier we saw: For every compact subset \(K\) of \(M\) there is a unique class (\enquote{relative fundamental class}) \(\mu_K \in H_n(M, M\setminus K; \IZ)\) such that
    \[r_x^K(\mu_K) = \mu_x\]
    for all \(x \in K\). So the cap product gives a map
    \[\mu_K \cap \_\colon H^i(M, M \setminus K; \IZ) \to H_{n-i}(M; \IZ)\]

    \textbf{Claim.} for \(K \subseteq L\) both compact, the map
    \[\incl_*\colon H_n(M, M\setminus L; \IZ) \to H_n(M, M\setminus K; \IZ)\]
    satisfies \(\incl_*(\mu_L) = \mu_K\).

    Indeed for all \(x \in K\),
    \[r_x^K(\incl_*(\mu_L)) = r_x^L(\mu_L) = \mu_x\]
    for all \(x \in K\), so \(\incl_*(\mu_L)\) has the properties that charecterize \(\mu_K\).

    \textbf{Claim.} For \(\alpha\in H^i(M; M\setminus K; \IZ)\) we have
    \[\mu_K \cap \alpha = \incl_*(\mu_L) \cap \alpha = \mu_L\cap \incl^*(\alpha) \]
    in \(H_{n-i}(M; \IZ)\) using the mixed functoriality of cap.

    So the following commutes
    \[\begin{tikzcd}
        H^i(M, M\setminus K; \IZ) \ar[rd, "\mu_K \cap \_"] \ar[d, "j_L^K = \incl^*"] & \\
        H^i(M, M\setminus L; \IZ) \ar[r, "\mu_L \cap \_"] & H_{n-i}(M, \IZ)\\
    \end{tikzcd}\]
    so there is a unique homomorphism \(D_M\colon H^i_{comp}(M, \IZ) \to H_{n-i}(M; \IZ)\) such that for all \(K \subseteq M\) compact, the following commutes:
    \[\begin{tikzcd}
        H_{comp}^i(M; \IZ) \ar[r] & H_{n-i}(M; \IZ) \\
        H^i(M, M\setminus K; \IZ) \ar[u, "\lambda_K"] \ar[ru, "\mu_K \cap \_"] & \\
    \end{tikzcd}\]
    This can be thought of to some extend as \enquote{capping with a non-existent fundamental class.}
\end{construction}

\begin{proposition}
    For any orientation of \(\IR^n\), all \(i \geq 0\), the duality map
    \[D_{\IR^n}\colon H_{comp}^i(\IR^n, \IZ) \to H_{n-i}(\IR^n; \IZ)\]
    is an isomorphism.
\end{proposition}

\begin{proof}
    Earlier: \(H^i_{comp}(\IR^n; \IZ) = \begin{cases}
        \IZ & \text{if } i = n \\
        0 & \text{else} \\
    \end{cases}\)
    and \(H_{n-i}(\IR^n; \IZ) = 0\), if \(i \neq n\), so \(D_M\) is an isomorphism between trivial groups for \(i \neq n\).

    For \(i = n\), let
    \[\Psi\colon H^n(\IR^n; \IR^n\setminus \set 0; \IZ) \to \Hom(H_n(\IR \mid 0), \IZ)\]
    be the evaluation homomorphism from the universal coefficient theorem. This map is surjective by the universal coefficient theorem. Let \(\mu_0 \in H_n(\IR^n \mid 0)\) be the chosen local orientation. So there is a class \(\alpha \in H^i(\IR^n, \IR^n\setminus \set 0; \IZ)\) such that
    \[\Phi(\alpha) (\mu_0) = 1.\]

    Since \(\IR^n\) is path connected, every point \(x \in \IR^n\) represents the same element \(e = [x] \in H_0(\IR^n; \IZ)\) a generator. Let \(f \in C^n(\cS(\IR^n), \cS(\IR^n\setminus 0); \IZ)\) be a cochain that represents \(\alpha\),  and let
    \[\sum a_i \cdot \psi_i \in C_n(\cS(\IR^n), \cS(\IR^n \setminus 0); \IZ)\]
    represent \(\mu_0\). with some \(\psi_i\colon \nabla^n \to \IR^n, a_i \in \IZ\). Then
    \[\begin{split}
        \mu_0 \cap \alpha &= [\sum a_i \cdot \psi_i] \cap [f] \\
        &= \sum a_i \cdot [\psi_i \cap f] \\
        &= \sum a_i \cdot f(\psi_i) \cdot [\psi(0,0,\dots,0,1)] \\
        &= (\sum a_i \cdot f(\psi_i)) \cdot e \\
        &= \underbrace{\Psi(\alpha)(\mu_0)}_{= 1} \cdot e = e
    \end{split}\]
    where \(e \in H_0(\IR, \IZ)\) is the geometric generator. So \(\mu_0 \cap \alpha\) generates \(H_0(\IR^n; \IZ)\).

    We can conclude:
    \[\begin{tikzcd}
        {H^n(\IR^n, \IR^n\setminus 0; \IZ)} \ar[r, "\lambda_{\set 0}", "\cong"'] \ar[dr, "\mu_0 \cap \_", two heads] & H^n_{comp}(\IR^n; \IZ) \ar[d, "D_{\IR^n}"] \\
        & H_0(\IR^n; \IZ) \\
    \end{tikzcd}\]
    So \(D_M\) is a surjection between free abelian groups of rank 1, hence an isomorphism!
\end{proof}

We now prooved Poincaré duality for \(\IR^n\). This is in itself of course not very exciting, but the first step for the bootstrapping we will do later. For that we will develop a Mayer-Vietoris sequence for compactly supported cohomology.

We now first need to show naturality of the duality map.

We let \(M\) be a \(n\)-manifold, \(\mu = \set{\mu_x}_{x \in X}\) an orientation and \(U\) an open subset of \(M\). Then \(\set{r_U^M(\mu_x)}_{x\in U}\) is an orientation of \(U\).

\begin{proposition}
    Let \((M, \mu)\) be an oriented \(n\)-manifold and \(U\) an open subset of \(M\) with induced orientation. Then the following diagram commutes:
    \[\begin{tikzcd}
        H^i_{comp}(M; \IZ) \ar[r, "D_M"] & H_{n-i}(M; \IZ) \\
        H^i_{comp}(U, \IZ) \ar[u, "\iota_U^M"] \ar[r, "D_U"] & H_{n-i}(U; \IZ) \ar[u, "\incl_*" ] \\
    \end{tikzcd}\]
\end{proposition}

\begin{proof}
    Let \(K\) be a compact susbset of \(U\).

    \textbf{Claim.} The map
    \[\incl_*\colon H_n(U, U\setminus K; \IZ) \to H_n(M, M\setminus K; \IZ)\]
    takes \(\mu_K^U\) to \(\mu_K^M\).

    Indeed for all \(x \in K\),
    \[r_x^M(\incl_*(\mu_K^U)) = r_x^U(\mu_K^U) = \mu_x^U = \mu_x\]
    Since this property characterizse \(\mu_K^M\), the claim holds.

    Consider the diagram
    \[\begin{tikzcd}
        H^i(U, U\setminus K; \IZ) \ar[r, "\lambda_K"] \ar[rr, bend left, "\mu_K^U \cap \_"] & H^i_{comp}(U; \IZ) \ar[d, "\iota_U^M"] \ar[r, "D_U"] & H_{n-i}(U; \IZ) \ar[d, "\incl_*"] \\
        H^i(M, M\setminus K; \IZ) \ar[u, "\incl^*", "\cong"'] \ar[rr, bend right, "\mu_K^M \cap \_"] \ar[r, "\lambda_K"] & H^i_{comp}(M; \IZ) \ar[r, "D_M"] & H_{n-i}(M; \IZ) \\
    \end{tikzcd}\]
    where we now the left square commutes but don't know about the right one yet.

    Every class in \(H^i_{comp}(U, \IZ)\) is of the form \(\lambda_K(\alpha)\) for some compact subset \(K\) of \(U\), some \(\alpha \in H^i(U, U\setminus K; \IZ)\).
    By excision there is a class \(\beta \in H^i(M, M\setminus K; \IZ)\), such that \(\incl^*(\beta) = \alpha\). So using the mixed functoriality of \(\cap\) and the definition of \(D_M\), we get
    \[\begin{split}
        \incl_*(D_U(\lambda_K(\alpha))) &\overset{\text{def. } D_U}= \incl_*(\mu_K^U \cap \alpha) \\
        &= \incl_*(\mu_K^U \cap \incl^*(\beta)) \\
        &= \mu_K^M \cap \beta \\
        &= D_M(\lambda_K(\beta)) \\
        &= D_M(\iota_U^M(\lambda_K(\incl^*(\beta)))) \\
        &= D_M(\iota_U^M(\lambda_K(\alpha))) \\
    \end{split}\]
    Since all classes in \(H_{comp}^i(U; \IZ)\) are of the form \(\lambda_K(\alpha)\), this prooves the proposition.
\end{proof}



\subsection{Mayer-Vietoris sequences for compactly supported cohomology}

\begin{lem}{Compact unions}{}
    Let \(X\) be a locally compact Hausdorff space. Let \(U,V\) be open subsets of \(X\), such that \(X = U \cup V\). Then every compact subset of \(X\) is of the form \(K \cup L\) for a compact subset \(K\) of \(U\) and a compact subset \(L\) of \(V\).
\end{lem}

\begin{proof}
    Let \(C\) be any compact subset of \(X = U \cup V\). Then every point \(x \in C\) is contained in \(U\) or in \(V\). If \(x \in U\), then \(U\) is a neighborhood of \(x\) in \(X\). Since \(x\) is locally compact, there is a compact neighborhood \(N_x\) of \(x\) in \(U\). If \(x \in V\), there is a compact neighborhood \(N_x\) of \(x\) in \(V\). Since \(C\) is compact, it is covered by finitely many \(N_{x_1} \cup \dots \cup N_{x_n}\) of these compact neighborhoods, with each \(N_{x_i}\) contained in \(U\) or \(V\).

    Set \(\bar K = \bigcup N_{x_i}\) such that \(N_{x_i} \subseteq U\) and \(\bar L = \bigcup N_{x_i}\) such that \(N_{x_i} \subseteq V\). So \(\bar K, \bar L\) are compact, \(\bar K \subseteq U, \bar L \subseteq V\). So \(K = C \cap \bar K\) is compact in \(U\) and \(L = C \cap \bar L\) is compact in \(V\). We see \(C = K \cup L\).
\end{proof}

\begin{construction}[Connecting homomorphism]
    Let \(X\) be a locally compact hausdorff space, \(U, V \subseteq X\) open, such that \(X = U \cup V\). Let \(A\) be any coefficient group, that we drop from the notation. We define
    \[\partial\colon H^i_{comp}(X) \to H_{comp}^{i+1}(U \cap V)\]
    For every compact subset \(C\) of \(X\), we define
    \[\partial_C\colon H^i(X, X\setminus C) \to H^{i+1}_{comp}(U\cap V)\]
    as follows. We choose compact subsets \(K\) of \(U\) and \(L\) of \(V\), such that \(C = K \cup L\) and define \(\partial_C\) as the composite
    \[H^i(X, X\setminus C) = H^i(X, X\setminus (C \cup L)) \xrightarrow{\partial} H^{i+1}(X,X\setminus(K \cap L)) \xrightarrow{\text{excision}} H^{i+1}(U \cap V, (U \cap V) \setminus (K \cap L)) \xrightarrow{\lambda_{K \cap L}} H_{comp}^{i+1}(U\cap V)\]
    using \(X\setminus (K \cap L) = (X\setminus K) \cup (X\setminus L)\) and \((X\setminus K) \cap (X\setminus L) = X \setminus (K \cup L)\)

    \textbf{Claim.}
    \begin{enumerate}
        \item the definition of \(\partial_C\) is independent of the choice of \(K,L\).
        \item If \(C \subseteq \bar C \subseteq X\) with \(C, \bar C\) compact, then
        \[\begin{tikzcd}
            H^i(X, X\setminus C) \ar[rd, "\partial_C"] \ar[d, "\iota_C^{\bar C}"] & \\
            H^i(X, X\setminus \bar C) \ar[r, "\partial_{\bar C}"] & H_{comp}^{i+1}(U\cap V) \\
        \end{tikzcd}\]
    \end{enumerate}
\end{construction}


\end{document}