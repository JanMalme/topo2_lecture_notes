\documentclass[language=english]{TemplateLecture}

\renewcommand{\ProfName}{Stefan Schwede}
\renewcommand{\LectureName}{Topology II}
\renewcommand{\Semester}{SoSe 2025}
\renewcommand{\mName}{Jan Malmström}

\begin{document}

\setcounter{chapter}{0}
\chapter{Cohomology}

\newLecture{07.04.2025}

\section{Last Term}

In last term, we discussed
\begin{itemize}
    \item CW-complexes
    \item higher homotopy groups
    \item Whitehead theorem 
    \item Singular homology
    \item cellular homology
\end{itemize}

In the very end, cohomology was started. Remeber
\[\begin{tikzcd}
    & & & \mathrm{Ab} \\
    \mathrm{TOP} \ar[rrru, bend left, "H_n(X;A)"] \ar[rrrd, bend right, "H^n(X;A)"] \ar[r, "\cS"] & (\mathrm{simpl. Sets}) \ar[r, "{C(\_, \IZ)}"] & (\mathrm{Chains}) \ar[ru, "H_n(\_ \otimes A)"] \ar[rd, "{H^n(\Hom(\_, A))}"] & \\
    & & & \mathrm{Ab}^{op} \\
\end{tikzcd}\]

\section{Cup-product}


Let \(X\) be a simplicial set, and \(R\)\footnote{A ring is not necessarily commutative, but has a unit} a ring\index{ring}.
\[C^n(X, R) = \mathrm{maps}(X_n, R)\]
is an abelian group under pointwise addition. There is a differential
\[d^n\colon C^n(X,R) \to C^{n+1}(X,R)\]
given by
\[d^n(f)(y) = \sum_{i = 0}^{n+1}(-1)^{i} f(d_i^*(y))\]
with \(f\colon X_n \to R, y \in X_{n+1}\)


\begin{construction}[Cup product/Alexander Whitney map]
    \index{cup-product}\index{Alexander Whitney map}
    The cup prodcut/Alexander Withney map
    \[\cup\colon C^n(X,R) \times C^m(X,R) \to C^{m+n}(X,R)\]
    with \(n,m \geq 0\) is defined by
    \[(f\cup g)(x)\coloneq f(d_{front}^*(x)) \cdot g(d_{back}^*(x))\]
    with \(f\colon X_n \to R, g\colon X_m \to R\), \(x \in X_{n+m}\).

    Where we use \([n+m] = \set{0,1,\dots, n+m}\) and \(d_{front} \colon [n] \to [n+m], d_{back}\colon [m]\to [n+m]\) are given by \(d_{front}(i) = i\), \(d_{back}(i) = n+i\). Note, that \(d_{front}\) and \(d_{back}\) respectively suppress in their noation \(n\) and \(m\).
\end{construction}

\begin{thm}{fundamental properties of cup product}{fpcp}
    The cup-product satisfies the following properties.
    \begin{enumerate}
        \item The AW-map is biadditive and satisfies a boundary formula:
        \[d(f\cup g) = (df) \cup g + (-1)^n f \cup (dg) \in C^{m+n+1}(X,R)\]
        \item Associativity: For \(h \in C^k(X, R)\), \((f \cup g) \cup h = f \cup (g\cup h) \in C^{n+m+k}(X,R)\).
        
        Let \(1 \in C^0(X,R)\) be the constant function \(1\colon X_0 \to R\) with value 1. Then \(1 \cup f = f\cup 1 = f\).
        \item Naturality: Let \(\alpha \colon Y \to X\) be a morphism of symplicial sets. Then
        \[\alpha^*(f \cup g) = \alpha^*(f) \cup \alpha^*(g), \quad \alpha^*(1) = 1.\]
        where \(\alpha^*\colon C^n(X,R) \to C^n(Y,R), \quad f \mapsto f\circ \alpha_n\).
    \end{enumerate}
\end{thm}

\begin{proof}\leavevmode
    \begin{itemize}
        \item Let \(d_{front}\colon [n] \to [n+m]\), \(d_{back}\colon [m] \to [n+m]\) be as in the definition of \(\cup\). Then
        \[d_i \circ d_{front} = \begin{cases}
            d_{front} \circ d_i & 0 \leq i \leq n+1 \\
            d_{front} & n+1 \leq i \leq n+m+1 \\
        \end{cases}\]
        and
        \[d_i \circ d_{back} = \begin{cases}
            d_{back} \circ d_i & 0 \leq i \leq n \\
            d_{back} \circ d_{i-n} & n \leq i \leq n+m+1 \\
        \end{cases}\]
        Note, that for \(n+1\) and \(n\) respectively the cases are the same.

        now
        \[\begin{split}
            d(f\cup g)(x) &= \sum_{i = 0}^{n+m+1}(-1)^{i} (f\cup g)(d_i^*(x)) \\
            &= \sum_{i = 0}^{n+m+1} (-1)^{i} \cdot f(d_{front}^*(x)) \cdot g(d_{back}^*(d_i^*(x))) \\
            = \sum_{i = 0}^{n} (-1)^{i} &\cdot f(d_{front}^*(d_i^*(x))) \cdot g(d_{back}^*(d_i^*(x))) + \sum_{j = 1}^{m+1} (-1)^{n+j} \cdot f(d_{front}^*(d_{j+n}^*(x))) \cdot g(d_{back}^*(d_{j+n}^*(x))) \\
            &= \sum_{i = 0}^{n+1}(-1)^{i} \cdot f(d_i^*(d_{front}^*(x))) \cdot g(d_{back}^*(x)) + \sum_{j = 0}^{m+1}(-1)^{n+j} f(d_{front}^*(x)) \cdot g(d_j^*(d_{back}^*(x))) \\
            &= d(f)(d_{front}^*(x)) \cdot g(d_{back}^*(x)) + (-1)^n \cdot f(d_{front}^*(x)) \cdot d(g)(d_{back}^*(x)) \\
            &= ((df) \cup g)(x) + (-1)^n \cdot (f\cup dg)(x)\\
            & = ((df) \cup g + (-1)^n \cdot f\cup (dg))(x)
        \end{split}\]

        \item For \(x \in X_{n+m+k}\) we see
        \[\begin{split}
            ((f\cup g) \cup h)(x) &= (f\cup g) (d_{front}^*(x)) \cdot h(d_{back}^*(x)) \\
            &= f(d_{front}^*(d_{front}^*(x))) \cdot g(d_{back}^*(d_{front}^*(x))) \cdot h(d_{back}^*(x))\\
            & = f(d_{front}^*(x)) \cdot g(d_{middle}^*(x)) \cdot h(d_{back}^*(x))
        \end{split}\]
        Note that we abuse that \(d_{front}\) suppresses the indices for which the map is the front map.
        We have in the last line
        \[d_{front}\colon[n] \to [n+m+k], d_{middle}\colon [m] \to [n+m+k], d_{back}\colon [k] \to [n+m+k]\]
        defined by
        \[d_{front}(i) = i, d_{middle}(i) = n+i, d_{back}(i) = n+m+i\]
        this is obviously associative in the inputs\footnote{for Schwede at least.}
        \item Naturality for \(\alpha\colon Y \to X\) we see
        \[\begin{split}
            (\alpha^*(f\cup g))(y) &= (f\cup g)(\alpha_{n+m}(y)) \\
            &= f(d_{front}^*(\alpha_{n+m}(y))) \cdot g(d_{back}^*(\alpha_{n+m}(y))) = f(\alpha_n(d_{front}^*(y))) \cdot g(\alpha_m(d_{back}^*(y))) \\
            &= \alpha^*(f)(d_{front}^*(y)) \cdot \alpha^*(g)(d_{back}^*(y)) \\
            &= (\alpha^*(f) \cup \alpha^*(g))(y).
        \end{split}\]
    \end{itemize}
\end{proof}

\begin{defi}{Differential graded ring}{}
    A differential graded ring (dg-ring)\index{differential graded ring}\index{graded ring} is a cochain-complex \(A = \set{A^n, d^n}_{n \in \IZ}\) equipped with biadditive maps
    \[\cdot\colon A^n \times A^m \to A^{n+m}, \quad n,m \in \IZ \]
    and a unit \(1 \in A^0\), such that;
    \begin{itemize}
        \item \(\cdot\) is associative and has \(1\) as a unit element.
        \item the Leibniz rule holds:
        \[d(a\cdot b) = (da) \cdot b + (-1)^n \cdot a \cdot (db)\]
        with \(a \in A^n, b \in A^m\).\footnote{The sign is somehow connected to a sign-rule I couldn't follow. The d moved past the a or something.}
    \end{itemize}
\end{defi}

\begin{example}
    Some Differential graded rings are:
    \begin{itemize}
        \item \(C^\cdot(X,R)\) for a simplicial set \(X\) and a ring \(R\).
        \item De Rham complex of a smooth manifold.
    \end{itemize}

\end{example}

\begin{construction}[Cup-Product on cohomology]\index{cup-product}
    Let \(A = (A^n, d, \cdot)\) be a dg-ring. We define a map
    \[\cdot \colon H^n(A) \times H^m(A) \to H^{n+m}(A), \quad [a] \cdot [b] = [a \cdot b]\]

    This is well defined:
    \[d(a\cdot b) = \underset{= 0}{(da)} \cdot b + (-1)^n @. a \cdot \underset{= 0}{(db)} = 0\]
    so \(a\cdot b\) is a cycle and we can take its homology class. Let \(x \in A^{n-1}\).
    \[(a+ dx) \cdot b = a\cdot b + (dx) \cdot b = a \cdot b + d(x\cdot b) = [(a+dx)\cdot b] = [a \cdot b]\]
    so it only depends on the cohomology class of \(a\), analogous for \(b\).

    The product on cohomology inherits associativity and unity with \(1 = [1] \in H^0(A)\). We need to see \(1\) is a cocycle:
    \[d(1) = d(1\cdot 1) = (d1) \cdot 1 + (-1)^0 1 \cdot (d1) = 2 \cdot d(1)\]
    and so \(d(1) = 0\).

    The cup product on the \(R\)-cohomology of a simplicial set \(X\) is the product induced by the cup product on \(C^*(X,R)\) in \(H^*(C(X,R)) = H^*(X,R)\).
\end{construction}

\begin{thm}{Properties of the cup-product on homology}{}
    Let \(X\) be a simplicial set and \(R\) a ring. Then
    \begin{itemize}
        \item The cup product on \(H^*(X,R)\) is associative and unital, with unit the cohomology class of the constant function \(1\colon X_0 \to R\).
        \item For a morphism of simplicial sets \(\alpha\colon Y \to X\), the relation
        \[\alpha^*([x] \cup [y]) = \alpha^*[X] \cup \alpha^*[y]\]
        holds for all \([x] \in H^n(X,R), [y] \in H^m(X,R)\).
    \end{itemize}
\end{thm}

\begin{remark}
    The cup product generalizes to relative cohomology: For \(A, B\) simplicial subsets of \(X\). We have
    \[C^n(X,A;R) = \set{f\colon X_n\to R \mid f(A_n) = \set{0}}\]
    The relative cup product is the restriciton of \(\cup\) on \(C^*(X,R)\) to
    \[C^n(X,A;R) \times C^m(X,B;R) \xrightarrow{u} C^{n+m}(X,A\cup B;R).\]
    Let \(x \in (A \cup B)_{n+m}\), then
    \[(f\cup g)(x) = f(d_{front}^*(x)) \cdot g(d_{back}^*(x))\]
    if \(x \in A_{n+m}\) then \(f(d_{front}^*(x)) = 0\) and analogous with \(B_{n+m}\), anyways the product is 0.

    This gives us biadditive well defined maps
    \[\cup \colon H^n(X,A;R) \times H^n(X,B;R) \to H^{n+m}(X,A\cup B; R)\]

    In particular for \(A = B\) we get
    \[\cup\colon H^n(X,A;R) \times H^n(X,A;R) \to H^{n+m}(X,A;R)\]
    which is well defined and associative, but not unital anymore.
\end{remark}

\section{Commutativity of the cup-product}

\begin{thm}{Commutativity of the cup-product}{}
    Let \(X\) be a simplicial set and \(R\) a commutative ring. Then for all \([x] \in H^n(X,R); [y] \in H^m(X,R)\) the realtion
    \[[x] \cup [y] = (-1)^{n\cdot m} \cdot [y] \cup [x]\]
    holds.
\end{thm}

Schwede points out, that the easy way doesn't work.
\textbf{Warning.} For \(f \in C^n(X,R), g \in C^m(Y,R)\), then in general \(f\cup g \neq (-1)^{n+m} (g\cup f)\) in \(C^{n+m}(X,R)\). The commutativity is a property we only get on homology.


\begin{construction}
    The \(\cup_1\)-product (spoken Cup-one)
    \[\cup_1\colon C^n(X,R) \times C^m(X,R) \to C^{n+m-1}(X,R)\]
    is defined by
    \[(f\cup_1 g)(x) = \sum_{i =0}^{n-1} (-1)^{(n-1)\cdot (m+1)} f((d_i^{out})^*(x)) \cdot g((d_i^{inner})^*(x))\]
    for \(f\in C^n\), \(g \in C^m\) and \(x \in X_{n+m-1}\).\footnote{There are also \(\cup_i\) for \(i \in \IN\). However, they are quite messy and combinatorical.}
    where \(d_i^{out}\colon [n] \to [n+m-1], d_i^{inner}\colon [m] \to [n+m-1]\) are the unique monotone injective maps with images \(\img(d_i^{out}) = \set{0, \dots, i} \cup \set{i+m, \dots, n+m-1}\) and \(\img(d_i^{inn}) = \set{i, \dots, i+m}\).
\end{construction}

\begin{thm}{\(\cup_1\)-Product}{}
    The \(\cup_1\)-product satisfies the following formula
    \[d(f\cup_1 g) = (df) \cup_1 g + (-1)^n \cdot f \cup_1 (dg) - (-1)^{n+m}(f\cup g) - (-1)^{n+1}{m+1}(g\cup f)\]
    for \(f \in C^n(X,R)\) and \(g \in C^m(X,R)\).
\end{thm}

\begin{remark}
    What we want to see, is that \(f\cup g\) and \(g\cup f\) are not the same but rather homotopic, and \(\cup_1\) wittnesses that homotopy.
\end{remark}

\begin{proof}
    This theorem will not be prooven, because it is quite messy. You should find a lecture-video for that.
\end{proof}

Now suppose that \(f\) and \(g\) are cocycles, i.e. \(df = 0\), \(dg = 0\). Then
\[d(f\cup_1 g) = -(-1)^{n+m}(f\cup g) - (-1)^{(n+1)(m+1)}(g\cup f) \]
and we get
\[(-1)^{n+m+1}\cdot d(f\cup_1 g) = f\cup g - (-1)^{n \cdot m} (g\cup f)\] and as such
\[0 = [(-1)^{n+m-1}] = [f] \cup [g] - (-1)^{n\cdot m} [g] \cup [f]\]

\begin{remark}
    Last term we discussed the tensor product of two chain complexes (in an exercise):
    \[(C\otimes D)_n = \bigoplus_{p+q = n} C_p \otimes D_q\]
    and differential
    \[d(x\otimes y) = (dx) \otimes y + (-1)^{\abs{x}} \cdot x \otimes (dy)\]
\end{remark}

\begin{remark}
    Reinterpretation of \(d(f \cup_1 g)\).
    The cup product yields a morphism of cochain complexes
    \[C^*(X,R) \otimes C^*(X,R) \to C^*(X,R)\]
    and we get a diagram
    \[\begin{tikzcd}
        x \otimes y \ar[d] & C^*(X,R) \otimes C^*(X,R) \ar[r, "\cup"] \ar[d] & C^*(X,R) \\
        y \otimes x & C^*(X,R) \otimes C^*(X,R) \ar[ru, "\cup"] & \\
    \end{tikzcd}\]
    that does not commute, however it does so up to cochain homotopy and \(\cup_1\) is exactly a cochain homotopy between the two maps.
\end{remark}


\newLecture{09.04.2025}


Only with the definition of the cup-product we cannot calculate a lot yet. Some methods to compute cup-products are:
\begin{itemize}
    \item directly from the definition
    \item cellular approximation of the diagonal (whatever that means, he gives a little intuition I failed to record.) (this might be used later)
    \item Group homology (one exapmle later today, something for AT I)
    \item Poincaré duality (later this term)
    \item Analysis on smooth manifolds together with De Rahm Cohomology 
\end{itemize}
The first two methods are not very practical.

\begin{example}
    Let \(X\) be a discrete space, Then \(\cS(X)\) is a constant simplicial set. The chain complex has the form
    \[\begin{tikzcd}
        \ar[r, "0"] & \IZ[X] \ar[r, "="] & \IZ[X] \ar[r, "0"] & \IZ[X]\\
    \end{tikzcd}\]
    And so \(H^n(X,R) = 0\) for \(n \geq 0\).
    And only for \(n = m = 0\) something nontrivial happens. for \(f\colon X_0 \to R, g\colon X_0 \to R\), we have \((f\cup g)(x) = f(d_{front}^*(x)) \cdot g(d_{back}^*(x)) = f(x) \cdot g(x)\)
    and so the cup product is just pointwise multiplication in dimension \(0\).

    More generally: \(H^0(X,R) = \mathrm{maps}(\pi_0(X),R)\) with \(\cup\)-prodcut pointwise multiplication
\end{example}

\begin{example}
    Let \(G\) be a group: Define a category \(\underline{G}\)\footnote{via geometric realization, these define interesting spaces, namely some (missed word)-Maclane spaces \(M(G,1)\), didn't catch it all} wit one object \(*\) and \(\Hom_{\underline{G}}(*,*) = G\). We then define
    \[BG = N(\underline{G})\]
    Where \(N\) is the Nerve-Functor \(\mathbf{CAT} \to \mathbf{Sset}\).
    Then
    \[(BG)_n = G^n, \quad d_i^*\colon G^n \to G^{n-1} (g_1, \dots, g_n) \mapsto \begin{cases}
        (g_2, \dots, g_n) & i = 0 \\
        (g_1, \dots, g_i \circ g_{i+1}, \dots, g_n) & 1 \leq i \leq n-1 \\
        (g_1, \dots , g_{n-1}) & i = n \\
    \end{cases}\]
    And \(s_i(g_1, \dots, g_n) = (g_1, \dots, g_i, 1, g_{i+1}, \dots, g_n)\).

    The general case of this is too hard to calculate. We take \(G = (\IF_2, +)\) and \(R = \IF_2\) and we calculate \(H^*(B\IF_2, \IF_2)\).
    We see
    \[\begin{tikzcd}
        {C^0(BG,A)} \ar[r, "d"] \ar[d, phantom, sloped, "= "] & C^1(BG,A) \ar[r, "d"] \ar[d, phantom, sloped, "="] & C^2(BG,A) \ar[r] \ar[d, phantom, sloped, "="] & \dots \\
        \mathrm{maps}(\{1\},A) \ar[r, "0"] \ar[d, sloped, phantom, "\cong"] & \mathrm{maps}(G,A) \ar[r] & \mathrm{maps}(G^2,A) & \\
        A &(f\colon G \to A) \ar[r]&(df)(g,h) & \\
    \end{tikzcd}\]
    And the map is defined by
    \[f(d_0^*(g,h)) - f(d_1^*(g,h)) + f(d_2^*(g,h)) = f(h) - f(g\cdot h) + f(g)\]
    and 
    \[df = 0 \Lra f(g,h) = f(g) + f(h)\]
    \(\implies\) 1-cocycles are the group homomorphisms from \(G\) to \(A\)
    \[H^1(BG,A) \cong \Hom(G,A)\]
    and  for \(G = (\IF_2, +)\), \(A = \IF_2\)
    
    We define
    \[0 \neq x \coloneq [\Id_{\IF_2}] \in H^1(B\IF_2, \IF_2).\]
    We will show that \(x^n = x \cup \dots \cup x\) (\(n\)-times) \(\in H^n(B\IF_2, \IF_2)\) is nonzero.
    \begin{Proposition}
        \(x^n \in H^n(B\IF_2, \IF_2)\) is represented by
        \[f_n \colon (\IF_2)^n \to \IF_2, f_n(\lambda_1, \dots, \lambda_n) = \lambda_1 \cdot \dots \cdot \lambda_n = \begin{cases}
            1 & \text{if } \lambda_1 = \lambda_2 = \dots = \lambda_n = 1 \\
            0 & \text{else}
        \end{cases}\]
    \end{Proposition}
    \begin{proof}
        By induction on \(n\). We checked for \(n = 1\).
        For \(n \geq 2\) we have
        \[\begin{split}
            x^n &= x^{n-1} \cup x = [f_{n-1}] \cup [\Id_{\IF_2}] \\
            &= [f_{n-1} \cup \Id] \\
        \end{split}\]
        Then
        \[\begin{split}
            (f_{n-1} \cup \Id)(\lambda_1 , \dots, \lambda_n) &= f_{n-1}(d_{front}^*(\lambda_1, \dots, \lambda_n)) \cdot \Id_(d_{back}^*(\lambda_1, \dots, \lambda_n)) \\
            &= f_{n-1}(\lambda_1, \dots, \lambda{n-1}) \cdot \Id(\lambda_n) \\
            &= (\lambda_1 \cdot \dots \cdot \lambda_{n-1}) \cdot \lambda_n \\
        \end{split}\]
    \end{proof}

    Claim: \(x^n \neq 0\). In the UCT for cohomology we used the evaluation pair
    \[\Phi\colon H^n(X,A) \to \Hom(H_n(X;\IZ);A), \quad [f_n\colon X_n \to A] \mapsto \set*{[\sum b_ix_i] \mapsto \sum b_i f(x_i)}\]
    for \(b_i \in \IZ, x_i \in X_n\).
    We can slightly variate that for ring coefficients:
    \[\Phi\colon H^n(X,R) \to \Hom(H_n(X,R), R)\]
    and \([f\colon X_n \to R] \mapsto \set{[\sum r_i \cdot x_i] \mapsto \sum r_i \cdot f(x_i)}\) with \(r_i \in R, x_i \in X_n\).

    With \(X = B\IF_2, R = \IF_2\), we consider
    \[y\coloneq \sum_{(\lambda_1 \dots, \lambda_n) \in (\IF_2)^n} 1 (\lambda_1, \dots, \lambda_n) \in \IF_2[(\IF_2)^n] = \IF_2[(B\IF_2)_n]\]
    Claim: \(y\) is an \(n\)-cycle in \(C_*(B\IF_2, \IF_2)\).
    \[\begin{split}
        dy &= \sum_{i = 0, \dots n} (-1)^{i} \cdot d_i^*(\sum_1\cdot (\lambda_1, \dots, \lambda_n)) \\
        &= \sum_{i = 0,\dots, n} \underbrace{\sum_{(\lambda_1, \dots, \lambda_n) \in \IF_2^n} (-1)^{i} \cdot d_i^*(\lambda_1, \dots, \lambda_n)}_{\text{cancel in pairs}} \\
        &= 0
    \end{split}\]
    Now
    \[d_0^*(0,\lambda_2, \dots, \lambda_n) = (\lambda_2, \dots \lambda_n) = d_0^*(1, \lambda_2, \dots, \lambda_n)\]
    So
    \[\Phi(x^n) \colon H_n(B\IF_2, \IF_2) \to \IF_2\]
    \[\Phi(x^n)[y] = \Phi[f_n][\sum_{(\lambda_1, \dots, \lambda_n) \in \IF_2^n}(\lambda_1, \dots, \lambda_n)] = \sum_{(\lambda_1, \dots, \lambda_n)} f_n(\lambda_1, \dots, \lambda_n) = \sum_{(\lambda_1, \dots, \lambda_n)} \lambda_1,\cdot \dots \cdot \lambda_n = 1 \neq 0\]
    and \([y] \neq 0\) in \(H_n(B\IF_2, \IF_2)\).

    We will later see, that in fact \(H^*(B\IF_2;\IF_2) = \IF_2[X]\).

    \begin{Remark}
        Let \(p\) be an odd prime. \(H^*(B\IF_p, \IF_p) = ?\).
        \[0 \neq x = [\Id_{\IF_p} \in H^1(B\IF_p; \IF_p)]\]
        still makes sense, but now there are more scalars and
        \[x^n = 0\]
        for \(n \geq 2\). The graded commutativity says:
        \[x\cup x = (-1)^{1\cdot 1} x \cup x = - x\cup x\]
        so if \(R\) is commutative, \(x \in H^n(X,R)\) and \(n\) is odd, then \(2\cdot(x\cup x) = 0\) in \(H^{2n}(X,R)\).
        And then \(2\cdot x^2 = 0 \Ra x^2 = 0\).

        Define \(h \colon \IF_p \times \IF_p \to \IF_p\) by 
        \[h(i,j) = \begin{cases}
            0 & \text{if } i + j < p \\
            1 & \text{if } i+ j \geq p \\
        \end{cases}\]
        where we write \(\IF_p = \set{0, \dots, p-1}\). Now \(h \in C^2(B\IF_p, \IF_p)\). Fact: \(dh = 0\) and \(0 \neq y \coloneq [h] \in H^2(B\IF_p, \IF_p)\).

        We then get (but do not proove)
        \[H^*(B\IF_p, \IF_p) = \Lambda(x) \otimes \IF_p[y]\] and
        \[H^{2n}(B\IF_p, \IF_p) = \IF_p \set{y^n}, \quad H^{2n+1}(B\IF_p, \IF_p) = \IF_p \set{xy^n}\]
    \end{Remark}
\end{example}

\section{Künneth theorem}
\index{Künneth theorem}
The Künneth theorem is an algebraic relationship between \(H^*_*(X,R), H^*_*(Y,R)\) and \(H^*_*(X\times Y, R)\)\footnote{\(H_*^*\) denotes, that Schwede was too lazy to write the statement for homology and cohomology separately}.

Here is a simplest version in homology with field coefficients:
\begin{thm}{Künneth, simple version}{}
    Let \(X\) and \(Y\) be spaces and \(k\) a field. Then
    \[H_n(X\times Y, k)\]
    is natural isomorphic to
    \[\bigoplus_{p+q = n} H_p(X,k) \otimes_k H_q(Y,k)\]
\end{thm}

\subsection{The Eilenberg-Zilber-theorem}

Let \(A,B\) be simplicial abelian groups. Then we get two natural chain homotopy equivalences
\[C_*(A) \otimes C_*(B) C_*(A\otimes B)\]
up Eilenberg Zilber map, bottom Alexander Whitney map %%help

\begin{defi}{Simplicial abelian group}{}
    A \emph{simplicial abelian group} \index{simplicial abelian group} is a functor \(A\colon \Delta^{Op} \to \mathbf{Ab.Groups}\).
\end{defi}

\begin{remark}
    Equivalently a simplicial abelian group is a colleciton of abelian groups \(A_n\), and homomorphisms \(\alpha^*\colon A_m \to A_n\) for all \(\alpha\colon [n] \to [m]\) in \(\Delta\), s.t. \((\alpha\circ \beta)^* = \beta^* \circ \alpha^*\).

    Equivalently a simplicial abelian group is a simplical set endorsed with abelian group structure on the sets of \(n\)-simplices, such that all \(\alpha^*\) are homomorphisms.
\end{remark}


\begin{example}
    Let \(X\) be a simplicial set and \(A\) an abelian group. Then the composite
    \[\begin{tikzcd}
        \Delta^{op} \ar[rr, bend left, "{A[X]}"]\ar[r, "X"] & (\mathbf{Sets}) \ar[r, "{{A[\_]}}"] & (\mathbf{ab.grps}) \\
    \end{tikzcd}\]
    is a simplicial abelian group.
\end{example}

\begin{construction}
    Let \(A\colon \Delta^{op} \to (\mathbf{ab.grps})\) be a simplicial abelian groups. Its \emph{chain complex} \(C_*(A)\) is the chain complex with \(C_n(A) = A_n\) with differential
    \[d\colon C_n(A) = A_n \to A_{n-1} = C_{n-1}(A), \quad d(a) = \sum_{i = 0, \dots, n} (-1)^{i} d_i^*(a)\]
    And one can easily check \(d\circ d = 0\).
\end{construction}

\textbf{Note.} The following commutes
\[\begin{tikzcd}
    (\mathbf{Ssets}) \ar[rr, "X \mapsto {C_*(X,A)}"] \ar[rd, "{A[_]}"] & & (\mathbf{Chains}) \\
    & (\mathbf{s.ab.grps}) \ar[ru, "C_*"] \\
\end{tikzcd}\]

\begin{remark}
    The tensor product of chain complexes \(C,D\) is
    \[(C\otimes D)_n \coloneq \bigoplus_{p+q = n} C_p \otimes D_q\]
    with differential
    \[d(x\otimes y) = (dx \otimes y) + (-1)^p x \otimes (dy)\]
    for \(x \in C_p, y \in D_q\).

    We can also form the tensor product of simplical abelian groups: \(A,B \colon \Delta^{op} \to (\mathbf{ab.grps})\) by
    \[(A\otimes B)_n = A_n \otimes B_n, \quad \alpha^*\colon (A\otimes B)_n \to (A\otimes B)_m\]
    for \(\alpha\colon [m] \to [n]\) isis defined as \(\alpha^*(a\otimes b) = \alpha^*(a) \otimes \alpha^*(b)\) and we write \(\alpha^*_{A\otimes B} \coloneq \alpha^*_A \otimes \alpha^*_B\).
    Or this can be equally described as
    \[\Delta^{op}\xrightarrow{(A,B)} (\mathbf{ab.grps})^2 \xrightarrow{\otimes} (\mathbf{ab.grps})\]

    \textbf{Warning.} For \(A,B \in (\mathbf{SAB}) = \) simplicial abelian groups
    \[C_*(A\otimes B) \neq C_*(A) \otimes C_*(B)\]
    Also he did this in dimension \(n\), but I lacked time to copy.

    The Eilenberg-Zilber theorem is a natural pair of chain homotopy equivalences between these two.
\end{remark}

\begin{construction}
    Let \(A,B\) be simplicial chain groups. The \emph{Alexander-Whitney map} is the chain map
    \[AW\colon C_*(A\otimes B) \to C_*(A) \otimes C_*(B)\]
    defined by
    \[\begin{tikzcd}
        C_n(A\otimes B) \ar[d, phantom, sloped, "="]\ar[r] & \bigoplus\limits_{p+q = n, p,q \geq 0} A_p \otimes B_q \ar[d, phantom, sloped, "="] \\
        A_n \otimes B_n & C_*(A) \otimes C_*(B)
    \end{tikzcd}\]
    \[AW_n(a\otimes b) = \sum_{p+q = n} d_{front}^*(a) \otimes d_{back}^*(b)\]
    Where \([p] \xrightarrow{d_{front}} [p+q] = [n] \xleftarrow{d_{back}} [q]\).

    You may check for yourself, that this is a chain map, however Schwede didn't do that.
\end{construction}


\end{document}